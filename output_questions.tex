\documentclass[a4paper,12pt]{article}
\usepackage{amsmath}
\usepackage{amsfonts}
\usepackage{amssymb}
\usepackage{geometry}
\geometry{a4paper, margin=1in}
\usepackage{polyglossia}
\setmainlanguage{vietnamese}
\setmainfont{Times New Roman}
\usepackage{tikz}
\usepackage{tkz-tab}
\usepackage{tkz-euclide}
\usetikzlibrary{calc,decorations.pathmorphing,decorations.pathreplacing}
\begin{document}
\title{Câu hỏi Tối ưu hóa}
\author{dev}
\maketitle

Câu 1: Cho hàm số \(y=f(x)\) có bảng xét dấu \(f'(x)\) như dưới đây:

\begin{tikzpicture}[>=stealth, scale=1]
	\tkzTabInit[lgt=2,espcl=2]
	{$x$/1,$f''(x)$/0.8,$f'(x)$/3}
	{$-\infty$,$-5$,$3$,$5$,$+\infty$}
	\tkzTabLine{,-,0,+,0,-,0,+,}
	\path
	(N12)node[shift={(0,-0.2)}](A){$+\infty$}
	(N23)node[shift={(0,0.2)}](B){$-8$}
	(N32)node[shift={(0,-1.5)}](C){$8$}
	(N43)node[shift={(0,0.2)}](D){$-4$}
	(N52)node[shift={(0,-0.2)}](E){$+\infty$};
	\foreach \X/\Y in {A/B,B/C,C/D,D/E} \draw[->](\X)--(\Y);
\end{tikzpicture}

Hàm số có bao nhiêu cực tiểu?

A. 1

*B. 2

C. 5

D. 3

Lời giải:

Dựa vào bảng xét dấu \(f'(x)\), ta xác định được: - Dấu của \(f'(x)\): âm trên \((-\infty; -5)\), dương trên \((-5; 3)\), âm trên \((3; 5)\), dương trên \((5; +\infty)\) - Hàm số có các điểm cực trị tại: \(x = -5, x = 3, x = 5\) - Điểm cực tiểu: \(x = -5\) và \(x = 5\) (chuyển từ giảm sang tăng) - Điểm cực đại: \(x = 3\) (chuyển từ tăng sang giảm) - Hàm số nghịch biến trên \((-\infty; -5) \cup (3; 5)\) - Hàm số đồng biến trên \((-5; 3) \cup (5; +\infty)\) **Kết luận:** Hàm số có 2 cực tiểu (tại x = -5 và x = 5).



Câu 2: Cho hàm số \(y=f(x)\) có bảng xét dấu \(f'(x)\) như dưới đây:

\begin{tikzpicture}[>=stealth, scale=1]
	\tkzTabInit[lgt=2,espcl=2]
	{$x$/1,$f''(x)$/0.8,$f'(x)$/3}
	{$-\infty$,$-4$,$0$,$6$,$+\infty$}
	\tkzTabLine{,+,0,-,0,+,0,-,}
	\path
	(N13)node[shift={(0,0.2)}](A){$+\infty$}
	(N22)node[shift={(0,-0.2)}](B){$1$}
	(N32)node[shift={(0,-1.5)}](C){$-1$}
	(N42)node[shift={(0,-0.2)}](D){$8$}
	(N53)node[shift={(0,0.2)}](E){$+\infty$};
	\foreach \X/\Y in {A/B,B/C,C/D,D/E} \draw[->](\X)--(\Y);
\end{tikzpicture}

Hàm số có bao nhiêu cực trị?

A. 5

B. 4

C. 1

*D. 3

Lời giải:

Dựa vào bảng xét dấu \(f'(x)\), ta xác định được: - Dấu của \(f'(x)\): dương trên \((-\infty; -4)\), âm trên \((-4; 0)\), dương trên \((0; 6)\), âm trên \((6; +\infty)\) - Hàm số có các điểm cực trị tại: \(x = -4, x = 0, x = 6\) - Điểm cực đại: \(x = -4\) và \(x = 6\) (chuyển từ tăng sang giảm) - Điểm cực tiểu: \(x = 0\) (chuyển từ giảm sang tăng) - Hàm số đồng biến trên \((-\infty; -4) \cup (0; 6)\) - Hàm số nghịch biến trên \((-4; 0) \cup (6; +\infty)\) **Kết luận:** Hàm số có 3 cực trị (tại x = -4, x = 0, x = 6).



Câu 3: Cho hàm số \(y=f(x)\) có bảng xét dấu \(f'(x)\) như dưới đây:

\begin{tikzpicture}[>=stealth, scale=1]
	\tkzTabInit[lgt=2,espcl=2]
	{$x$/1,$f''(x)$/0.8,$f'(x)$/3}
	{$-\infty$,$-3$,$3$,$5$,$+\infty$}
	\tkzTabLine{,+,0,-,0,+,0,-,}
	\path
	(N13)node[shift={(0,0.2)}](A){$+\infty$}
	(N22)node[shift={(0,-0.2)}](B){$1$}
	(N32)node[shift={(0,-1.5)}](C){$-1$}
	(N42)node[shift={(0,-0.2)}](D){$10$}
	(N53)node[shift={(0,0.2)}](E){$+\infty$};
	\foreach \X/\Y in {A/B,B/C,C/D,D/E} \draw[->](\X)--(\Y);
\end{tikzpicture}

Hàm số có bao nhiêu cực đại?

A. 5

B. 3

*C. 2

D. 0

Lời giải:

Dựa vào bảng xét dấu \(f'(x)\), ta xác định được: - Dấu của \(f'(x)\): dương trên \((-\infty; -3)\), âm trên \((-3; 3)\), dương trên \((3; 5)\), âm trên \((5; +\infty)\) - Hàm số có các điểm cực trị tại: \(x = -3, x = 3, x = 5\) - Điểm cực đại: \(x = -3\) và \(x = 5\) (chuyển từ tăng sang giảm) - Điểm cực tiểu: \(x = 3\) (chuyển từ giảm sang tăng) - Hàm số đồng biến trên \((-\infty; -3) \cup (3; 5)\) - Hàm số nghịch biến trên \((-3; 3) \cup (5; +\infty)\) **Kết luận:** Hàm số có 2 cực đại (tại x = -3 và x = 5).



Câu 4: Cho hàm số \(y=f(x)\) có bảng xét dấu \(f'(x)\) như dưới đây:

\begin{tikzpicture}[>=stealth, scale=1]
	\tkzTabInit[lgt=2,espcl=2]
	{$x$/1,$f''(x)$/0.8,$f'(x)$/3}
	{$-\infty$,$-2$,$1$,$5$,$+\infty$}
	\tkzTabLine{,+,0,-,0,+,0,-,}
	\path
	(N13)node[shift={(0,0.2)}](A){$+\infty$}
	(N22)node[shift={(0,-0.2)}](B){$2$}
	(N32)node[shift={(0,-1.5)}](C){$-5$}
	(N42)node[shift={(0,-0.2)}](D){$10$}
	(N53)node[shift={(0,0.2)}](E){$+\infty$};
	\foreach \X/\Y in {A/B,B/C,C/D,D/E} \draw[->](\X)--(\Y);
\end{tikzpicture}

Phương trình f'(x) = a có bao nhiêu nghiệm?

A. 2

B. 4

C. 5

*D. 3

Lời giải:

Dựa vào bảng xét dấu \(f'(x)\), ta xác định được: - Dấu của \(f'(x)\): dương trên \((-\infty; -2)\), âm trên \((-2; 1)\), dương trên \((1; 5)\), âm trên \((5; +\infty)\) - Hàm số có các điểm cực trị tại: \(x = -2, x = 1, x = 5\) - Điểm cực đại: \(x = -2\) và \(x = 5\) (chuyển từ tăng sang giảm) - Điểm cực tiểu: \(x = 1\) (chuyển từ giảm sang tăng) - Hàm số đồng biến trên \((-\infty; -2) \cup (1; 5)\) - Hàm số nghịch biến trên \((-2; 1) \cup (5; +\infty)\) **Kết luận:** Với a = 0, phương trình f'(x) = 0 có 3 nghiệm (tại x = -2, x = 1, x = 5).



Câu 5: Cho hàm số \(y=f(x)\) có bảng xét dấu \(f'(x)\) như dưới đây:

\begin{tikzpicture}[>=stealth, scale=1]
	\tkzTabInit[lgt=2,espcl=2]
	{$x$/1,$f''(x)$/0.8,$f'(x)$/3}
	{$-\infty$,$-5$,$1$,$6$,$+\infty$}
	\tkzTabLine{,+,0,-,0,+,0,-,}
	\path
	(N13)node[shift={(0,0.2)}](A){$+\infty$}
	(N22)node[shift={(0,-0.2)}](B){$2$}
	(N32)node[shift={(0,-1.5)}](C){$-2$}
	(N42)node[shift={(0,-0.2)}](D){$10$}
	(N53)node[shift={(0,0.2)}](E){$+\infty$};
	\foreach \X/\Y in {A/B,B/C,C/D,D/E} \draw[->](\X)--(\Y);
\end{tikzpicture}

Hàm số có bao nhiêu cực đại?

A. 0

B. 1

*C. 2

D. 4

Lời giải:

Dựa vào bảng xét dấu \(f'(x)\), ta xác định được: - Dấu của \(f'(x)\): dương trên \((-\infty; -5)\), âm trên \((-5; 1)\), dương trên \((1; 6)\), âm trên \((6; +\infty)\) - Hàm số có các điểm cực trị tại: \(x = -5, x = 1, x = 6\) - Điểm cực đại: \(x = -5\) và \(x = 6\) (chuyển từ tăng sang giảm) - Điểm cực tiểu: \(x = 1\) (chuyển từ giảm sang tăng) - Hàm số đồng biến trên \((-\infty; -5) \cup (1; 6)\) - Hàm số nghịch biến trên \((-5; 1) \cup (6; +\infty)\) **Kết luận:** Hàm số có 2 cực đại (tại x = -5 và x = 6).


\end{document}
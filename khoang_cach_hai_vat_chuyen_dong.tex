
\documentclass[a4paper,12pt]{article}
\usepackage{amsmath}
\usepackage{mathtools}
\usepackage{amsfonts}
\usepackage{amssymb}
\usepackage{geometry}
\geometry{a4paper, margin=1in}
\usepackage{polyglossia}
\setmainlanguage{vietnamese}
\setmainfont{Times New Roman}
\usepackage{tikz}
\usepackage{tkz-tab}
\usepackage{tkz-euclide}
\usetikzlibrary{calc,decorations.pathmorphing,decorations.pathreplacing}
\begin{document}
\title{Bài toán cực trị hình học}
\maketitle
Câu 1: 
Với một hình hộp chữ nhật \(ABCD.A'B'C'D'\) có \(AB=12\), \(AD=17\), \(AA'=10\) như hình vẽ. Ở cùng một thời điểm hai con kiến coi như bò chuyển động thẳng đều, con kiến \(M\) bò từ \(B'\) đến điểm \(A\) với tốc độ \(2.9\,\mathrm{cm/s}\) và con kiến \(N\) bò từ \(C\) đến \(D'\) với tốc độ bằng \(1.7\,\mathrm{cm/s}\). Hãy tính khoảng cách nhỏ nhất giữa hai con kiến theo đơn vị centimet (làm tròn kết quả đến hàng phần mười)?

% BEGIN TIKZ (không cần sửa phần này)
\begin{tikzpicture}[scale=1.0]
% Định nghĩa các đỉnh của hình hộp với phép chiếu đúng
\coordinate (B) at (0,0);
\coordinate (C) at (4,0);
\coordinate (A) at (1.5,1.5);
\coordinate (D) at (5.5,1.5);
\coordinate (B') at (0,3);
\coordinate (C') at (4,3);
\coordinate (A') at (1.5,4.5);
\coordinate (D') at (5.5,4.5);
% Vẽ các cạnh nhìn thấy của hình hộp
\draw (B) -- (C);
\draw (B') -- (C') -- (D') -- (A') -- cycle;
\draw (B) -- (B');
\draw (C) -- (C');
\draw (C) -- (D);
\draw (D) -- (D');
% Vẽ các cạnh ẩn bằng đường đứt nét
\draw[dashed] (B) -- (A);
\draw[dashed] (A) -- (D);
\draw[dashed] (A') -- (A);
% Định nghĩa điểm M trên đường chéo AB'
\coordinate (M) at (0.75,2.25);
% Định nghĩa điểm N trên cạnh CD'
\coordinate (N) at (4.75,2.25);
% Vẽ đường chéo AB' bằng đường đứt nét đỏ (nhưng dùng màu đen theo yêu cầu)
\draw[dashed, thick] (A) -- (B');
% Vẽ đường từ C đến N đến D' (đường xanh trong gốc, nhưng dùng màu đen)
\draw[thick] (C) -- (N) -- (D');
% Vẽ các mũi tên
\draw[->, thick] (0.1,2.7) -- (0.5,2.3);  % Mũi tên từ B' về phía M
\draw[->, thick] (4.17,0.3) -- (4.57,1.4);
% Đánh dấu các điểm
\fill (A) circle (1.2pt);
\fill (B) circle (1.2pt);
\fill (C) circle (1.2pt);
\fill (D) circle (1.2pt);
\fill (A') circle (1.2pt);
\fill (B') circle (1.2pt);
\fill (C') circle (1.2pt);
\fill (D') circle (1.2pt);
\fill (M) circle (1.2pt);
\fill (N) circle (1.2pt);
% Gắn nhãn cho các điểm
\node[below right] at (A) {\(A\)};
\node[below left] at (B) {\(B\)};
\node[below right] at (C) {\(C\)};
\node[below right] at (D) {\(D\)};
\node[above right] at (A') {\(A'\)};
\node[above left] at (B') {\(B'\)};
\node[above left] at (C') {\(C'\)};
\node[above right] at (D') {\(D'\)};
\node[right] at (M) {\(M\)};
\node[right] at (N) {\(N\)};
\end{tikzpicture}


A. 17.9 cm

B. 15.6 cm

*C. 17.3 cm

D. 18.7 cm

Lời giải:


Dữ kiện:
+ Hình hộp chữ nhật \(ABCD.A'B'C'D'\) có:
\[
AB = 12,\quad AD = 17,\quad AA' = 10
\]
+ Gán tọa độ:
\[
A = (0, 0, 0),\quad B = (12, 0, 0),\quad D = (0, 17, 0),\quad B' = (12, 0, 10),\quad D' = (0, 17, 10)
\]
+ Con kiến \(M\) bò từ \(B'\) đến \(A\) với tốc độ \(2.9\, \text{cm/s}\)

+ Con kiến \(N\) bò từ \(C\) đến \(D'\) với tốc độ \(1.7\, \text{cm/s}\)

Bước 1: Phương trình chuyển động

Con kiến \(M\):

Vector chỉ phương đường đi:
\[
\overrightarrow{B'A} = (0, 0, 0) - (12, 0, 10) = ( -12, 0, -10 )
\]
Chiều dài đoạn \(B'A\):
\[
|\overrightarrow{B'A}| = \sqrt{12^2 + 10^2} = \sqrt{144 + 100} = \sqrt{244}
\]

Trong một giây con kiến tại \(M\) đi được \(\frac{2.9}{\sqrt{244}}\) lần \(\overrightarrow{B'A}\)

Véctơ vận tốc của con kiến tại \(M\) là:
\[
\overrightarrow{v}_M = \frac{2.9}{\sqrt{244}} \cdot ( -12, 0, -10 )
\]

Vị trí tại thời điểm \(t\):
\[
M(t) = (12, 0, 10) + t \cdot \frac{2.9}{\sqrt{244}} \cdot ( -12, 0, -10 )
= \left( 12 - \frac{2.2t}{\sqrt{244}},\; 0,\; 10 - \frac{1.9t}{\sqrt{244}} \right)
\]

Con kiến \(N\):

Vị trí tại thời điểm \(t\):
\[
N(t) = (12, 17, 0) + t \cdot (0, 0, 1.7) = (12, 17, 1.7t)
\]

Bước 2: Tính khoảng cách giữa hai con kiến tại thời điểm \(t\)

\[
d(t) = \sqrt{\left(12 - \frac{2.2t}{\sqrt{244}}\right)^2 + 17^2 + \left(10 - \frac{1.9t}{\sqrt{244}} - 1.7t\right)^2}
\]

Bước 3: Tìm khoảng cách nhỏ nhất

\(d'(t)=0\Leftrightarrow t \approx 3.09 \Rightarrow d_{min}=17.3cm\)


\end{document}
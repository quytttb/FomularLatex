\documentclass[a4paper,12pt]{article}
\usepackage{amsmath}
\usepackage{amsfonts}
\usepackage{amssymb}
\usepackage{geometry}
\geometry{a4paper, margin=1in}
\usepackage{polyglossia}
\setmainlanguage{vietnamese}
\setmainfont{Times New Roman}
\usepackage{tikz}
\usepackage{tkz-tab}
\usepackage{tkz-euclide}
\usetikzlibrary{calc,decorations.pathmorphing,decorations.pathreplacing}
\begin{document}
\title{Câu hỏi Tối ưu hóa}
\author{dev}
\maketitle

Câu 1: Trong không gian \(Oxyz\), một vật có trọng lượng \(P=215N\) đặt trên một giá đỡ ba chân với điểm đặt là \(D(1; 2; 5)\). Ba điểm tiếp xúc với mặt đất \(A, B, C\) nằm trên mặt phẳng \((Ozx)\). Biết tọa độ các điểm \(A(-2; 0; 1), B(-4; 0; 1), C(a; b; c)\), tam giác \(ABC\) đều. 

Biết rằng trọng lực \(\overrightarrow{P}=(0; 0; -6)\) sẽ ép vào ba thanh \(DA, DB, DC\) các lực \(\overrightarrow{F}_1, \overrightarrow{F}_2, \overrightarrow{F}_3\) lần lượt hướng dọc theo các vectơ \(\overrightarrow{DA}, \overrightarrow{DB}, \overrightarrow{DC}\). 

Theo tính chất Vật Lý thì ta có: \(\overrightarrow{F_1}+\overrightarrow{F_2}+\overrightarrow{F_3}=\overrightarrow{P}\).

Hỏi trong các mệnh đề dưới đây, mệnh đề nào đúng, mệnh đề nào sai?

*A. a²+b²+c²=9.54; 1 đơn vị=35.83N; S=1.732; F₃=902.0N

B. a²+b²+c²=6.539999999999999; 1 đơn vị=55.83N; S=1.289; F₃=1176N

C. a²+b²+c²=8.54; 1 đơn vị=15.829999999999998N; S=1.626; F₃=1040N

D. a²+b²+c²=8.54; 1 đơn vị=25.83N; S=2.325; F₃=982N

Lời giải:


**a) Tính a² + b² + c²:**

Do C thuộc mặt phẳng Ozx nên y = 0, ta có C(-3.000; 0; -0.732).

Để tam giác ABC đều, cần: |AB| = |AC| = |BC|.

Khoảng cách AB = 2.000

Giải hệ phương trình:
- |AC|² = |AB|²  
- |BC|² = |AB|²

Từ đó suy ra: a² + b² + c² = 9.54

**b) Đơn vị lực:**

Ta có |P⃗| = 6 ứng với 215N
⟹ 1 đơn vị = 36N

**c) Diện tích tam giác ABC:**

Sử dụng tích có hướng:
- AB⃗ = (-2; 0; 0)
- AC⃗ = (-1.000; 0.000; -1.732)

Diện tích = ½|AB⃗ × AC⃗| = 1.732

**d) Độ lớn lực F₃:**

Giải hệ: x1DA⃗ + x2DB⃗ + x3DC⃗ = P⃗

Với:
- DA⃗ = (-3.0; -2.0; -4.0)
- DB⃗ = (-5.0; -2.0; -4.0)
- DC⃗ = (-4.000; -2.000; -5.732)

Nghiệm: x3 = 3.464102

|F₃| = |x3| × |DC⃗| × 36 = 902N




Câu 2: Trong không gian \(Oxyz\), một vật có trọng lượng \(P=292N\) đặt trên một giá đỡ ba chân với điểm đặt là \(D(2; 1; 1)\). Ba điểm tiếp xúc với mặt đất \(A, B, C\) nằm trên mặt phẳng \((Oyz)\). Biết tọa độ các điểm \(A(0; -3; -1), B(0; -1; 0), C(a; b; c)\), tam giác \(ABC\) đều. 

Biết rằng trọng lực \(\overrightarrow{P}=(0; 0; -6)\) sẽ ép vào ba thanh \(DA, DB, DC\) các lực \(\overrightarrow{F}_1, \overrightarrow{F}_2, \overrightarrow{F}_3\) lần lượt hướng dọc theo các vectơ \(\overrightarrow{DA}, \overrightarrow{DB}, \overrightarrow{DC}\). 

Theo tính chất Vật Lý thì ta có: \(\overrightarrow{F_1}+\overrightarrow{F_2}+\overrightarrow{F_3}=\overrightarrow{P}\).

Hỏi trong các mệnh đề dưới đây, mệnh đề nào đúng, mệnh đề nào sai?

A. a²+b²+c²=10.73; 1 đơn vị=18.67N; S=1.572; F₃=507N

*B. a²+b²+c²=9.73; 1 đơn vị=48.67N; S=2.165; F₃=588.0N

C. a²+b²+c²=6.73; 1 đơn vị=78.67N; S=1.651; F₃=747N

D. a²+b²+c²=8.73; 1 đơn vị=68.67N; S=2.672; F₃=649N

Lời giải:


**a) Tính a² + b² + c²:**

Do C thuộc mặt phẳng Oyz nên x = 0, ta có C(0; -2.866; 1.232).

Để tam giác ABC đều, cần: |AB| = |AC| = |BC|.

Khoảng cách AB = 2.236

Giải hệ phương trình:
- |AC|² = |AB|²  
- |BC|² = |AB|²

Từ đó suy ra: a² + b² + c² = 9.73

**b) Đơn vị lực:**

Ta có |P⃗| = 6 ứng với 292N
⟹ 1 đơn vị = 49N

**c) Diện tích tam giác ABC:**

Sử dụng tích có hướng:
- AB⃗ = (0; 2; 1)
- AC⃗ = (0.000; 0.134; 2.232)

Diện tích = ½|AB⃗ × AC⃗| = 2.165

**d) Độ lớn lực F₃:**

Giải hệ: x1DA⃗ + x2DB⃗ + x3DC⃗ = P⃗

Với:
- DA⃗ = (-2.0; -4.0; -2.0)
- DB⃗ = (-2.0; -2.0; -1.0)
- DC⃗ = (-2.000; -3.866; 0.232)

Nghiệm: x3 = -2.771281

|F₃| = |x3| × |DC⃗| × 49 = 588N




Câu 3: Trong không gian \(Oxyz\), một vật có trọng lượng \(P=232N\) đặt trên một giá đỡ ba chân với điểm đặt là \(D(1; 1; 3)\). Ba điểm tiếp xúc với mặt đất \(A, B, C\) nằm trên mặt phẳng \((Oxy)\). Biết tọa độ các điểm \(A(-1; -3; 0), B(2; -5; 0), C(a; b; c)\), tam giác \(ABC\) đều. 

Biết rằng trọng lực \(\overrightarrow{P}=(0; 0; -6)\) sẽ ép vào ba thanh \(DA, DB, DC\) các lực \(\overrightarrow{F}_1, \overrightarrow{F}_2, \overrightarrow{F}_3\) lần lượt hướng dọc theo các vectơ \(\overrightarrow{DA}, \overrightarrow{DB}, \overrightarrow{DC}\). 

Theo tính chất Vật Lý thì ta có: \(\overrightarrow{F_1}+\overrightarrow{F_2}+\overrightarrow{F_3}=\overrightarrow{P}\).

Hỏi trong các mệnh đề dưới đây, mệnh đề nào đúng, mệnh đề nào sai?

A. a²+b²+c²=48.05; 1 đơn vị=18.67N; S=5.476; F₃=857N

*B. a²+b²+c²=45.05; 1 đơn vị=38.67N; S=5.629; F₃=931.0N

C. a²+b²+c²=48.05; 1 đơn vị=8.670000000000002N; S=4.573; F₃=985N

D. a²+b²+c²=42.05; 1 đơn vị=18.67N; S=5.009; F₃=1269N

Lời giải:


**a) Tính a² + b² + c²:**

Do C thuộc mặt phẳng Oxy nên z = 0, ta có C(-1.232; -6.598; 0).

Để tam giác ABC đều, cần: |AB| = |AC| = |BC|.

Khoảng cách AB = 3.606

Giải hệ phương trình:
- |AC|² = |AB|²  
- |BC|² = |AB|²

Từ đó suy ra: a² + b² + c² = 45.05

**b) Đơn vị lực:**

Ta có |P⃗| = 6 ứng với 232N
⟹ 1 đơn vị = 39N

**c) Diện tích tam giác ABC:**

Sử dụng tích có hướng:
- AB⃗ = (3; -2; 0)
- AC⃗ = (-0.232; -3.598; 0.000)

Diện tích = ½|AB⃗ × AC⃗| = 5.629

**d) Độ lớn lực F₃:**

Giải hệ: x1DA⃗ + x2DB⃗ + x3DC⃗ = P⃗

Với:
- DA⃗ = (-2.0; -4.0; -3.0)
- DB⃗ = (1.0; -6.0; -3.0)
- DC⃗ = (-2.232; -7.598; -3.000)

Nghiệm: x3 = -2.842340

|F₃| = |x3| × |DC⃗| × 39 = 931N




Câu 4: Trong không gian \(Oxyz\), một vật có trọng lượng \(P=170N\) đặt trên một giá đỡ ba chân với điểm đặt là \(D(2; 3; 3)\). Ba điểm tiếp xúc với mặt đất \(A, B, C\) nằm trên mặt phẳng \((Ozx)\). Biết tọa độ các điểm \(A(0; 0; 1), B(-3; 0; 0), C(a; b; c)\), tam giác \(ABC\) đều. 

Biết rằng trọng lực \(\overrightarrow{P}=(0; 0; -4)\) sẽ ép vào ba thanh \(DA, DB, DC\) các lực \(\overrightarrow{F}_1, \overrightarrow{F}_2, \overrightarrow{F}_3\) lần lượt hướng dọc theo các vectơ \(\overrightarrow{DA}, \overrightarrow{DB}, \overrightarrow{DC}\). 

Theo tính chất Vật Lý thì ta có: \(\overrightarrow{F_1}+\overrightarrow{F_2}+\overrightarrow{F_3}=\overrightarrow{P}\).

Hỏi trong các mệnh đề dưới đây, mệnh đề nào đúng, mệnh đề nào sai?

*A. a²+b²+c²=15.2; 1 đơn vị=42.5N; S=4.33; F₃=312.0N

B. a²+b²+c²=13.2; 1 đơn vị=22.5N; S=5.334; F₃=372N

C. a²+b²+c²=13.2; 1 đơn vị=52.5N; S=4.465; F₃=390N

D. a²+b²+c²=13.2; 1 đơn vị=32.5N; S=6.452; F₃=350N

Lời giải:


**a) Tính a² + b² + c²:**

Do C thuộc mặt phẳng Ozx nên y = 0, ta có C(-2.366; 0; 3.098).

Để tam giác ABC đều, cần: |AB| = |AC| = |BC|.

Khoảng cách AB = 3.162

Giải hệ phương trình:
- |AC|² = |AB|²  
- |BC|² = |AB|²

Từ đó suy ra: a² + b² + c² = 15.20

**b) Đơn vị lực:**

Ta có |P⃗| = 4 ứng với 170N
⟹ 1 đơn vị = 42N

**c) Diện tích tam giác ABC:**

Sử dụng tích có hướng:
- AB⃗ = (-3; 0; -1)
- AC⃗ = (-2.366; 0.000; 2.098)

Diện tích = ½|AB⃗ × AC⃗| = 4.330

**d) Độ lớn lực F₃:**

Giải hệ: x1DA⃗ + x2DB⃗ + x3DC⃗ = P⃗

Với:
- DA⃗ = (-2.0; -3.0; -2.0)
- DB⃗ = (-5.0; -3.0; -3.0)
- DC⃗ = (-4.366; -3.000; 0.098)

Nghiệm: x3 = -1.385641

|F₃| = |x3| × |DC⃗| × 42 = 312N




Câu 5: Trong không gian \(Oxyz\), một vật có trọng lượng \(P=210N\) đặt trên một giá đỡ ba chân với điểm đặt là \(D(5; 2; 1)\). Ba điểm tiếp xúc với mặt đất \(A, B, C\) nằm trên mặt phẳng \((Oyz)\). Biết tọa độ các điểm \(A(0; 3; -3), B(0; 5; -6), C(a; b; c)\), tam giác \(ABC\) đều. 

Biết rằng trọng lực \(\overrightarrow{P}=(0; 0; -6)\) sẽ ép vào ba thanh \(DA, DB, DC\) các lực \(\overrightarrow{F}_1, \overrightarrow{F}_2, \overrightarrow{F}_3\) lần lượt hướng dọc theo các vectơ \(\overrightarrow{DA}, \overrightarrow{DB}, \overrightarrow{DC}\). 

Theo tính chất Vật Lý thì ta có: \(\overrightarrow{F_1}+\overrightarrow{F_2}+\overrightarrow{F_3}=\overrightarrow{P}\).

Hỏi trong các mệnh đề dưới đây, mệnh đề nào đúng, mệnh đề nào sai?

A. a²+b²+c²=37.8; 1 đơn vị=25.0N; S=3.962; F₃=273N

B. a²+b²+c²=37.8; 1 đơn vị=45.0N; S=4.229; F₃=459N

C. a²+b²+c²=37.8; 1 đơn vị=15.0N; S=4.161; F₃=416N

*D. a²+b²+c²=40.8; 1 đơn vị=35.0N; S=5.629; F₃=329.0N

Lời giải:


**a) Tính a² + b² + c²:**

Do C thuộc mặt phẳng Oyz nên x = 0, ta có C(0; 1.402; -6.232).

Để tam giác ABC đều, cần: |AB| = |AC| = |BC|.

Khoảng cách AB = 3.606

Giải hệ phương trình:
- |AC|² = |AB|²  
- |BC|² = |AB|²

Từ đó suy ra: a² + b² + c² = 40.80

**b) Đơn vị lực:**

Ta có |P⃗| = 6 ứng với 210N
⟹ 1 đơn vị = 35N

**c) Diện tích tam giác ABC:**

Sử dụng tích có hướng:
- AB⃗ = (0; 2; -3)
- AC⃗ = (0.000; -1.598; -3.232)

Diện tích = ½|AB⃗ × AC⃗| = 5.629

**d) Độ lớn lực F₃:**

Giải hệ: x1DA⃗ + x2DB⃗ + x3DC⃗ = P⃗

Với:
- DA⃗ = (-5.0; 1.0; -4.0)
- DB⃗ = (-5.0; 3.0; -7.0)
- DC⃗ = (-5.000; -0.598; -7.232)

Nghiệm: x3 = 1.065877

|F₃| = |x3| × |DC⃗| × 35 = 329N




Câu 6: Trong không gian \(Oxyz\), một vật có trọng lượng \(P=157N\) đặt trên một giá đỡ ba chân với điểm đặt là \(D(5; 4; 5)\). Ba điểm tiếp xúc với mặt đất \(A, B, C\) nằm trên mặt phẳng \((Oyz)\). Biết tọa độ các điểm \(A(0; -3; 0), B(0; -3; -1), C(a; b; c)\), tam giác \(ABC\) đều. 

Biết rằng trọng lực \(\overrightarrow{P}=(0; 0; -6)\) sẽ ép vào ba thanh \(DA, DB, DC\) các lực \(\overrightarrow{F}_1, \overrightarrow{F}_2, \overrightarrow{F}_3\) lần lượt hướng dọc theo các vectơ \(\overrightarrow{DA}, \overrightarrow{DB}, \overrightarrow{DC}\). 

Theo tính chất Vật Lý thì ta có: \(\overrightarrow{F_1}+\overrightarrow{F_2}+\overrightarrow{F_3}=\overrightarrow{P}\).

Hỏi trong các mệnh đề dưới đây, mệnh đề nào đúng, mệnh đề nào sai?

A. a²+b²+c²=17.2; 1 đơn vị=6.170000000000002N; S=0.559; F₃=0N

*B. a²+b²+c²=15.2; 1 đơn vị=26.17N; S=0.433; F₃=0.0N

C. a²+b²+c²=17.2; 1 đơn vị=56.17N; S=0.579; F₃=0N

D. a²+b²+c²=12.2; 1 đơn vị=6.170000000000002N; S=0.611; F₃=0N

Lời giải:


**a) Tính a² + b² + c²:**

Do C thuộc mặt phẳng Oyz nên x = 0, ta có C(0; -3.866; -0.500).

Để tam giác ABC đều, cần: |AB| = |AC| = |BC|.

Khoảng cách AB = 1.000

Giải hệ phương trình:
- |AC|² = |AB|²  
- |BC|² = |AB|²

Từ đó suy ra: a² + b² + c² = 15.20

**b) Đơn vị lực:**

Ta có |P⃗| = 6 ứng với 157N
⟹ 1 đơn vị = 26N

**c) Diện tích tam giác ABC:**

Sử dụng tích có hướng:
- AB⃗ = (0; 0; -1)
- AC⃗ = (0.000; -0.866; -0.500)

Diện tích = ½|AB⃗ × AC⃗| = 0.433

**d) Độ lớn lực F₃:**

Giải hệ: x1DA⃗ + x2DB⃗ + x3DC⃗ = P⃗

Với:
- DA⃗ = (-5.0; -7.0; -5.0)
- DB⃗ = (-5.0; -7.0; -6.0)
- DC⃗ = (-5.000; -7.866; -5.500)

Nghiệm: x3 = 0.000000

|F₃| = |x3| × |DC⃗| × 26 = 0N




Câu 7: Trong không gian \(Oxyz\), một vật có trọng lượng \(P=279N\) đặt trên một giá đỡ ba chân với điểm đặt là \(D(1; 4; 4)\). Ba điểm tiếp xúc với mặt đất \(A, B, C\) nằm trên mặt phẳng \((Oxy)\). Biết tọa độ các điểm \(A(-1; 3; 0), B(-4; 5; 0), C(a; b; c)\), tam giác \(ABC\) đều. 

Biết rằng trọng lực \(\overrightarrow{P}=(0; 0; -3)\) sẽ ép vào ba thanh \(DA, DB, DC\) các lực \(\overrightarrow{F}_1, \overrightarrow{F}_2, \overrightarrow{F}_3\) lần lượt hướng dọc theo các vectơ \(\overrightarrow{DA}, \overrightarrow{DB}, \overrightarrow{DC}\). 

Theo tính chất Vật Lý thì ta có: \(\overrightarrow{F_1}+\overrightarrow{F_2}+\overrightarrow{F_3}=\overrightarrow{P}\).

Hỏi trong các mệnh đề dưới đây, mệnh đề nào đúng, mệnh đề nào sai?

A. a²+b²+c²=17.88; 1 đơn vị=63.0N; S=5.747; F₃=338N

B. a²+b²+c²=16.88; 1 đơn vị=73.0N; S=6.851; F₃=292N

C. a²+b²+c²=21.88; 1 đơn vị=123.0N; S=7.351; F₃=396N

*D. a²+b²+c²=19.88; 1 đơn vị=93.0N; S=5.629; F₃=307.0N

Lời giải:


**a) Tính a² + b² + c²:**

Do C thuộc mặt phẳng Oxy nên z = 0, ta có C(-4.232; 1.402; 0).

Để tam giác ABC đều, cần: |AB| = |AC| = |BC|.

Khoảng cách AB = 3.606

Giải hệ phương trình:
- |AC|² = |AB|²  
- |BC|² = |AB|²

Từ đó suy ra: a² + b² + c² = 19.88

**b) Đơn vị lực:**

Ta có |P⃗| = 3 ứng với 279N
⟹ 1 đơn vị = 93N

**c) Diện tích tam giác ABC:**

Sử dụng tích có hướng:
- AB⃗ = (-3; 2; 0)
- AC⃗ = (-3.232; -1.598; 0.000)

Diện tích = ½|AB⃗ × AC⃗| = 5.629

**d) Độ lớn lực F₃:**

Giải hệ: x1DA⃗ + x2DB⃗ + x3DC⃗ = P⃗

Với:
- DA⃗ = (-2.0; -1.0; -4.0)
- DB⃗ = (-5.0; 1.0; -4.0)
- DC⃗ = (-5.232; -2.598; -4.000)

Nghiệm: x3 = -0.466321

|F₃| = |x3| × |DC⃗| × 93 = 307N




Câu 8: Trong không gian \(Oxyz\), một vật có trọng lượng \(P=148N\) đặt trên một giá đỡ ba chân với điểm đặt là \(D(4; 3; 4)\). Ba điểm tiếp xúc với mặt đất \(A, B, C\) nằm trên mặt phẳng \((Oyz)\). Biết tọa độ các điểm \(A(0; -3; 0), B(0; -5; -3), C(a; b; c)\), tam giác \(ABC\) đều. 

Biết rằng trọng lực \(\overrightarrow{P}=(0; 0; -5)\) sẽ ép vào ba thanh \(DA, DB, DC\) các lực \(\overrightarrow{F}_1, \overrightarrow{F}_2, \overrightarrow{F}_3\) lần lượt hướng dọc theo các vectơ \(\overrightarrow{DA}, \overrightarrow{DB}, \overrightarrow{DC}\). 

Theo tính chất Vật Lý thì ta có: \(\overrightarrow{F_1}+\overrightarrow{F_2}+\overrightarrow{F_3}=\overrightarrow{P}\).

Hỏi trong các mệnh đề dưới đây, mệnh đề nào đúng, mệnh đề nào sai?

*A. a²+b²+c²=43.59; 1 đơn vị=29.6N; S=5.629; F₃=291.0N

B. a²+b²+c²=45.59; 1 đơn vị=39.6N; S=6.556; F₃=284N

C. a²+b²+c²=41.59; 1 đơn vị=39.6N; S=8.085; F₃=260N

D. a²+b²+c²=40.59; 1 đơn vị=39.6N; S=6.73; F₃=434N

Lời giải:


**a) Tính a² + b² + c²:**

Do C thuộc mặt phẳng Oyz nên x = 0, ta có C(0; -6.598; 0.232).

Để tam giác ABC đều, cần: |AB| = |AC| = |BC|.

Khoảng cách AB = 3.606

Giải hệ phương trình:
- |AC|² = |AB|²  
- |BC|² = |AB|²

Từ đó suy ra: a² + b² + c² = 43.59

**b) Đơn vị lực:**

Ta có |P⃗| = 5 ứng với 148N
⟹ 1 đơn vị = 30N

**c) Diện tích tam giác ABC:**

Sử dụng tích có hướng:
- AB⃗ = (0; -2; -3)
- AC⃗ = (0.000; -3.598; 0.232)

Diện tích = ½|AB⃗ × AC⃗| = 5.629

**d) Độ lớn lực F₃:**

Giải hệ: x1DA⃗ + x2DB⃗ + x3DC⃗ = P⃗

Với:
- DA⃗ = (-4.0; -6.0; -4.0)
- DB⃗ = (-4.0; -8.0; -7.0)
- DC⃗ = (-4.000; -9.598; -3.768)

Nghiệm: x3 = -0.888231

|F₃| = |x3| × |DC⃗| × 30 = 291N




Câu 9: Trong không gian \(Oxyz\), một vật có trọng lượng \(P=124N\) đặt trên một giá đỡ ba chân với điểm đặt là \(D(5; 3; 5)\). Ba điểm tiếp xúc với mặt đất \(A, B, C\) nằm trên mặt phẳng \((Oxy)\). Biết tọa độ các điểm \(A(-3; 3; 0), B(-4; 3; 0), C(a; b; c)\), tam giác \(ABC\) đều. 

Biết rằng trọng lực \(\overrightarrow{P}=(0; 0; -6)\) sẽ ép vào ba thanh \(DA, DB, DC\) các lực \(\overrightarrow{F}_1, \overrightarrow{F}_2, \overrightarrow{F}_3\) lần lượt hướng dọc theo các vectơ \(\overrightarrow{DA}, \overrightarrow{DB}, \overrightarrow{DC}\). 

Theo tính chất Vật Lý thì ta có: \(\overrightarrow{F_1}+\overrightarrow{F_2}+\overrightarrow{F_3}=\overrightarrow{P}\).

Hỏi trong các mệnh đề dưới đây, mệnh đề nào đúng, mệnh đề nào sai?

A. a²+b²+c²=14.8; 1 đơn vị=40.67N; S=0.325; F₃=0N

B. a²+b²+c²=18.8; 1 đơn vị=40.67N; S=0.323; F₃=0N

C. a²+b²+c²=13.8; 1 đơn vị=50.67N; S=0.62; F₃=0N

*D. a²+b²+c²=16.8; 1 đơn vị=20.67N; S=0.433; F₃=0.0N

Lời giải:


**a) Tính a² + b² + c²:**

Do C thuộc mặt phẳng Oxy nên z = 0, ta có C(-3.500; 2.134; 0).

Để tam giác ABC đều, cần: |AB| = |AC| = |BC|.

Khoảng cách AB = 1.000

Giải hệ phương trình:
- |AC|² = |AB|²  
- |BC|² = |AB|²

Từ đó suy ra: a² + b² + c² = 16.80

**b) Đơn vị lực:**

Ta có |P⃗| = 6 ứng với 124N
⟹ 1 đơn vị = 21N

**c) Diện tích tam giác ABC:**

Sử dụng tích có hướng:
- AB⃗ = (-1; 0; 0)
- AC⃗ = (-0.500; -0.866; 0.000)

Diện tích = ½|AB⃗ × AC⃗| = 0.433

**d) Độ lớn lực F₃:**

Giải hệ: x1DA⃗ + x2DB⃗ + x3DC⃗ = P⃗

Với:
- DA⃗ = (-8.0; 0.0; -5.0)
- DB⃗ = (-9.0; 0.0; -5.0)
- DC⃗ = (-8.500; -0.866; -5.000)

Nghiệm: x3 = 0.000000

|F₃| = |x3| × |DC⃗| × 21 = 0N



\end{document}
\documentclass[a4paper,12pt]{article}
\usepackage{amsmath}
\usepackage{amsfonts}
\usepackage{amssymb}
\usepackage{geometry}
\geometry{a4paper, margin=1in}
\usepackage{polyglossia}
\setmainlanguage{vietnamese}
\setmainfont{Times New Roman}
\usepackage{tikz}
\usepackage{tkz-tab}
\usepackage{tkz-euclide}
\usetikzlibrary{calc,decorations.pathmorphing,decorations.pathreplacing}
\begin{document}
\title{Câu hỏi Tối ưu hóa}
\maketitle

Câu 1: Cho đồ thị hàm số \(y=f(x)\) có đồ thị như hình vẽ dưới đây:


\begin{tikzpicture}[scale=1, font=\footnotesize, line join=round, line cap=round, >=stealth]
\draw[->] (-2.5,0)--(3.5,0) node[below] {\(x\)};
\draw[->] (0,-3.5)--(0,2.5) node[left] {\(y\)};
\draw[fill=black] (0,0) circle (1pt) node[below left=-2pt] {\(O\)};

% Đánh dấu các điểm cực trị - VỊ TRÍ TRÊN HÌNH VẼ CỐ ĐỊNH
\draw[fill=black] (-1,0) circle (1pt) node[below] {\(-2\)};
\draw[fill=black] (1,0) circle (1pt) node[below] {\(5\)};
\draw[fill=black] (0,1) circle (1pt) node[above left] {\(1\)};
\draw[fill=black] (0,-3) circle (1pt) node[below left] {\(-4\)};

% Đường kẻ phụ - VỊ TRÍ TRÊN HÌNH VẼ CỐ ĐỊNH
\draw[dashed] (-1,0)--(-1,-3)--(2,-3)--(2,0);
\draw[dashed] (-2,0)--(-2,1)--(1,1)--(1,0);

% Vẽ đường cong hàm số bậc 3 loại 2 (ngược)
\begin{scope}
\clip (-2.5,-3.5) rectangle (3.5,2.5);
\draw[smooth,samples=100,domain=-2.5:3.5] plot(\x,{-1*(\x)^3+3*(\x)-1});
\end{scope}
\end{tikzpicture}


Hàm số đạt cực đại tại điểm nào?

A. \(y=4\)

*B. \(x=5\)

C. \(y=-2\)

D. \(x=-5\)

Lời giải:


            Quan sát đồ thị hàm số, ta thấy:

- Hàm số có các điểm cực trị tại: \(x = -2, x = 5\)

- Đỉnh cao nhất (cực đại) tại điểm \((5, 1)\)

- Đỉnh thấp nhất (cực tiểu) tại điểm \((-2, -4)\)

- Các giá trị số nguyên tương ứng trên đồ thị

Từ đó suy ra đáp án cho câu hỏi đã cho.




Câu 2: Cho hàm số \(y=f(x)\) có bảng biến thiên như dưới đây:


\begin{tikzpicture}[>=stealth, scale=1]
\tkzTabInit[lgt=2,espcl=2]
{\(x\)/1,\(f'(x)\)/0.8,\(f(x)\)/3}
{\(-\infty\),\( -1 \),\( 1 \),\( 3 \),\(+\infty\)}
\tkzTabLine{-,0,+,0,-,0,+}
\path
(N12)node[shift={(0,-0.2)}](A){\(+\infty\)}
(N23)node[shift={(0,0.2)}](B){\(-3\)}
(N32)node[shift={(0,-1.5)}](C){\(1\)}
(N43)node[shift={(0,0.2)}](D){\(-1\)}
(N52)node[shift={(0,-0.2)}](E){\(+\infty\)};
\foreach\X/\Y in{A/B,B/C,C/D,D/E}\draw[->](\X)--(\Y);
\end{tikzpicture}


Đồ thị hàm số có điểm cực đại là điểm nào?

A. \(y=-1\)

B. \((1,1)\)

C. \((-3,1)\)

*D. \((-1,-3)\)

Lời giải:


Dựa vào bảng biến thiên, ta xác định được:

- Hàm số có các điểm cực trị tại: \(x = -1, x = 1, x = 3\)

- Giá trị cực đại: \(y = -3\) tại \(x = -1\)

- Giá trị cực tiểu: \(y = 1\) tại \(x = 1\)

- Điểm cực đại: \((-1, -3)\)

- Điểm cực tiểu: \((1, 1)\)

Từ đó suy ra đáp án cho câu hỏi đã cho.




Câu 3: Cho đồ thị hàm số \(y=f(x)\) có đồ thị như hình vẽ dưới đây:


\begin{tikzpicture}[scale=1, font=\footnotesize, line join=round, line cap=round, >=stealth]
\draw[->] (-2.5,0)--(3.5,0) node[below] {\(x\)};
\draw[->] (0,-3.5)--(0,2.5) node[left] {\(y\)};
\draw[fill=black] (0,0) circle (1pt) node[below left=-2pt] {\(O\)};

% Đánh dấu các điểm cực trị - VỊ TRÍ TRÊN HÌNH VẼ CỐ ĐỊNH
\draw[fill=black] (-1,0) circle (1pt) node[below] {\(-1\)};
\draw[fill=black] (1,0) circle (1pt) node[below] {\(3\)};
\draw[fill=black] (0,1) circle (1pt) node[above left] {\(3\)};
\draw[fill=black] (0,-3) circle (1pt) node[below left] {\(-2\)};

% Đường kẻ phụ - VỊ TRÍ TRÊN HÌNH VẼ CỐ ĐỊNH
\draw[dashed] (-1,0)--(-1,-3)--(2,-3)--(2,0);
\draw[dashed] (-2,0)--(-2,1)--(1,1)--(1,0);

% Vẽ đường cong hàm số bậc 3 loại 2 (ngược)
\begin{scope}
\clip (-2.5,-3.5) rectangle (3.5,2.5);
\draw[smooth,samples=100,domain=-2.5:3.5] plot(\x,{-1*(\x)^3+3*(\x)-1});
\end{scope}
\end{tikzpicture}


Hàm số có cực đại là giá trị nào?

A. \((1,1)\)

*B. \(y=3\)

C. \(x=1\)

D. \((-1,-4)\)

Lời giải:


            Quan sát đồ thị hàm số, ta thấy:

- Hàm số có các điểm cực trị tại: \(x = -1, x = 3\)

- Đỉnh cao nhất (cực đại) tại điểm \((3, 3)\)

- Đỉnh thấp nhất (cực tiểu) tại điểm \((-1, -2)\)

- Các giá trị số nguyên tương ứng trên đồ thị

Từ đó suy ra đáp án cho câu hỏi đã cho.



\end{document}
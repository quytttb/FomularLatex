\documentclass[a4paper,12pt]{article}
\usepackage{amsmath}
\usepackage{amsfonts}
\usepackage{amssymb}
\usepackage{geometry}
\geometry{a4paper, margin=1in}
\usepackage{polyglossia}
\setmainlanguage{vietnamese}
\setmainfont{Times New Roman}
\usepackage{tikz}
\usepackage{tkz-tab}
\usepackage{tkz-euclide}
\usetikzlibrary{calc,decorations.pathmorphing,decorations.pathreplacing}
\begin{document}
\title{Câu hỏi Tối ưu hóa}
\maketitle

Câu 1: Cho hàm số \(y=f(x)\) có bảng biến thiên như dưới đây:


\begin{tikzpicture}[>=stealth, scale=1]
\tkzTabInit[lgt=2,espcl=2]
{\(x\)/1,\(f'(x)\)/0.8,\(f(x)\)/3}
{\(-\infty\),\( -2 \),\( -1 \),\( 1 \),\(+\infty\)}
\tkzTabLine{-,0,+,0,-,0,+}
\path
(N12)node[shift={(0,-0.2)}](A){\(+\infty\)}
(N23)node[shift={(0,0.2)}](B){\(-4\)}
(N32)node[shift={(0,-1.5)}](C){\(-1\)}
(N43)node[shift={(0,0.2)}](D){\(3\)}
(N52)node[shift={(0,-0.2)}](E){\(+\infty\)};
\foreach\X/\Y in{A/B,B/C,C/D,D/E}\draw[->](\X)--(\Y);
\end{tikzpicture}


Hàm số đạt cực trị tại điểm nào?

A. \((-1,-1)\)

B. \(x=-1\)

C. \(x=3\)

*D. \(x=-2\) hoặc \(x=-1\) hoặc \(x=1\)

Lời giải:


Dựa vào bảng biến thiên, ta xác định được:

- Hàm số có các điểm cực trị tại: \(x = -2, x = -1, x = 1\)

- Giá trị cực đại: \(y = -4\) tại \(x = -2\)

- Giá trị cực tiểu: \(y = -1\) tại \(x = -1\)

- Điểm cực đại: \((-2, -4)\)

- Điểm cực tiểu: \((-1, -1)\)

Từ đó suy ra đáp án cho câu hỏi đã cho.




Câu 2: Cho hàm số \(y=f(x)\) có bảng biến thiên như dưới đây:


\begin{tikzpicture}[>=stealth, scale=1]
\tkzTabInit[lgt=2,espcl=2]
{\(x\)/1,\(f'(x)\)/0.8,\(f(x)\)/3}
{\(-\infty\),\( -3 \),\( 2 \),\( 3 \),\(+\infty\)}
\tkzTabLine{-,0,+,0,-,0,+}
\path
(N12)node[shift={(0,-0.2)}](A){\(+\infty\)}
(N23)node[shift={(0,0.2)}](B){\(4\)}
(N32)node[shift={(0,-1.5)}](C){\(-2\)}
(N43)node[shift={(0,0.2)}](D){\(3\)}
(N52)node[shift={(0,-0.2)}](E){\(+\infty\)};
\foreach\X/\Y in{A/B,B/C,C/D,D/E}\draw[->](\X)--(\Y);
\end{tikzpicture}


Hàm số đạt cực đại tại điểm nào?

*A. \(x=-3\)

B. \(y=-3\)

C. \((3,2)\)

D. \(x=2\)

Lời giải:


Dựa vào bảng biến thiên, ta xác định được:

- Hàm số có các điểm cực trị tại: \(x = -3, x = 2, x = 3\)

- Giá trị cực đại: \(y = 4\) tại \(x = -3\)

- Giá trị cực tiểu: \(y = -2\) tại \(x = 2\)

- Điểm cực đại: \((-3, 4)\)

- Điểm cực tiểu: \((2, -2)\)

Từ đó suy ra đáp án cho câu hỏi đã cho.




Câu 3: Cho hàm số \(y=f(x)\) có bảng biến thiên như dưới đây:


\begin{tikzpicture}[>=stealth, scale=1]
\tkzTabInit[lgt=2,espcl=2]
{\(x\)/1,\(f'(x)\)/0.8,\(f(x)\)/3}
{\(-\infty\),\( -3 \),\( -1 \),\( 1 \),\(+\infty\)}
\tkzTabLine{-,0,+,0,-,0,+}
\path
(N12)node[shift={(0,-0.2)}](A){\(+\infty\)}
(N23)node[shift={(0,0.2)}](B){\(2\)}
(N32)node[shift={(0,-1.5)}](C){\(1\)}
(N43)node[shift={(0,0.2)}](D){\(4\)}
(N52)node[shift={(0,-0.2)}](E){\(+\infty\)};
\foreach\X/\Y in{A/B,B/C,C/D,D/E}\draw[->](\X)--(\Y);
\end{tikzpicture}


Hàm số có cực đại là giá trị nào?

A. \(y=5\)

B. \(x=2\)

C. \((1,-1)\)

*D. \(y=2\)

Lời giải:


Dựa vào bảng biến thiên, ta xác định được:

- Hàm số có các điểm cực trị tại: \(x = -3, x = -1, x = 1\)

- Giá trị cực đại: \(y = 2\) tại \(x = -3\)

- Giá trị cực tiểu: \(y = 1\) tại \(x = -1\)

- Điểm cực đại: \((-3, 2)\)

- Điểm cực tiểu: \((-1, 1)\)

Từ đó suy ra đáp án cho câu hỏi đã cho.




Câu 4: Cho đồ thị hàm số \(y=f(x)\) như hình vẽ dưới đây:


\begin{tikzpicture}[line join=round, line cap=round,>=stealth,scale=1]
\tikzset{label style/.style={font=\footnotesize}}
\draw[->] (-2.1,0)--(2.5,0) node[below right] {\(x\)};
\draw[->] (0,-3.1)--(0,2.1) node[below left] {\(y\)};
\draw (0,0) node [below right] {\(O\)}circle(1.5pt);

% Đánh dấu các điểm cực trị - VỊ TRÍ TRÊN HÌNH VẼ CỐ ĐỊNH
\draw[dashed,thin](-1,0)--(-1,1)--(0,1);
\draw[dashed,thin](1,0)--(1,-3)--(0,-3);
\draw (1,0) node[above]{\( 4 \)}; 
\draw (-1,0) node[below]{\( -3 \)};
\draw (0,-3) node[left]{\( -3 \)};
\draw (0,1) node[right]{\( 1 \)};

% Vẽ đường cong hàm số bậc 3 loại 1
\begin{scope}
\clip (-2,-3) rectangle (2,2);
\draw[samples=200,domain=-2:2,smooth,variable=\x] plot (\x,{(\x)^3-3*(\x)-1});
\end{scope}
\end{tikzpicture}


Hàm số có cực đại là giá trị nào?

A. \(x=-2\)

*B. \(y=1\)

C. \((2,-3)\)

D. \((1,-3)\)

Lời giải:


            Quan sát đồ thị hàm số, ta thấy:

- Hàm số có các điểm cực trị tại: \(x = -3, x = 4\)

- Đỉnh cao nhất (cực đại) tại điểm \((-3, 1)\)

- Đỉnh thấp nhất (cực tiểu) tại điểm \((4, -3)\)

- Các giá trị số nguyên tương ứng trên đồ thị

Từ đó suy ra đáp án cho câu hỏi đã cho.




Câu 5: Cho đồ thị hàm số \(y=f(x)\) như hình vẽ dưới đây:


\begin{tikzpicture}[scale=1, font=\footnotesize, line join=round, line cap=round, >=stealth]
\draw[->] (-2.5,0)--(3.5,0) node[below] {\(x\)};
\draw[->] (0,-3.5)--(0,2.5) node[left] {\(y\)};
\draw[fill=black] (0,0) circle (1pt) node[below left=-2pt] {\(O\)};

% Đánh dấu các điểm cực trị - VỊ TRÍ TRÊN HÌNH VẼ CỐ ĐỊNH
\draw[fill=black] (-1,0) circle (1pt) node[below] {\(-3\)};
\draw[fill=black] (1,0) circle (1pt) node[below] {\(6\)};
\draw[fill=black] (0,1) circle (1pt) node[above left] {\(3\)};
\draw[fill=black] (0,-3) circle (1pt) node[below left] {\(-3\)};

% Đường kẻ phụ - VỊ TRÍ TRÊN HÌNH VẼ CỐ ĐỊNH
\draw[dashed] (-1,0)--(-1,-3)--(2,-3)--(2,0);
\draw[dashed] (-2,0)--(-2,1)--(1,1)--(1,0);

% Vẽ đường cong hàm số bậc 3 loại 2 (ngược)
\begin{scope}
\clip (-2.5,-3.5) rectangle (3.5,2.5);
\draw[smooth,samples=100,domain=-2.5:3.5] plot(\x,{-1*(\x)^3+3*(\x)-1});
\end{scope}
\end{tikzpicture}


Đồ thị hàm số có điểm cực đại là điểm nào?

A. \((1,3)\)

*B. \((6,3)\)

C. \((3,1)\)

D. \(x=5\)

Lời giải:


            Quan sát đồ thị hàm số, ta thấy:

- Hàm số có các điểm cực trị tại: \(x = -3, x = 6\)

- Đỉnh cao nhất (cực đại) tại điểm \((6, 3)\)

- Đỉnh thấp nhất (cực tiểu) tại điểm \((-3, -3)\)

- Các giá trị số nguyên tương ứng trên đồ thị

Từ đó suy ra đáp án cho câu hỏi đã cho.




Câu 6: Cho đồ thị hàm số \(y=f(x)\) như hình vẽ dưới đây:


\begin{tikzpicture}[line join=round, line cap=round,>=stealth,scale=1]
\tikzset{label style/.style={font=\footnotesize}}
\draw[->] (-2.1,0)--(2.5,0) node[below right] {\(x\)};
\draw[->] (0,-3.1)--(0,2.1) node[below left] {\(y\)};
\draw (0,0) node [below right] {\(O\)}circle(1.5pt);

% Đánh dấu các điểm cực trị - VỊ TRÍ TRÊN HÌNH VẼ CỐ ĐỊNH
\draw[dashed,thin](-1,0)--(-1,1)--(0,1);
\draw[dashed,thin](1,0)--(1,-3)--(0,-3);
\draw (1,0) node[above]{\( 5 \)}; 
\draw (-1,0) node[below]{\( -3 \)};
\draw (0,-3) node[left]{\( -3 \)};
\draw (0,1) node[right]{\( 4 \)};

% Vẽ đường cong hàm số bậc 3 loại 1
\begin{scope}
\clip (-2,-3) rectangle (2,2);
\draw[samples=200,domain=-2:2,smooth,variable=\x] plot (\x,{(\x)^3-3*(\x)-1});
\end{scope}
\end{tikzpicture}


Đồ thị hàm số có điểm cực đại là điểm nào?

*A. \((-3,4)\)

B. \((2,1)\)

C. \(y=-5\)

D. \((-3,-1)\)

Lời giải:


            Quan sát đồ thị hàm số, ta thấy:

- Hàm số có các điểm cực trị tại: \(x = -3, x = 5\)

- Đỉnh cao nhất (cực đại) tại điểm \((-3, 4)\)

- Đỉnh thấp nhất (cực tiểu) tại điểm \((5, -3)\)

- Các giá trị số nguyên tương ứng trên đồ thị

Từ đó suy ra đáp án cho câu hỏi đã cho.




Câu 7: Cho hàm số \(y=f(x)\) có bảng biến thiên như dưới đây:


\begin{tikzpicture}[>=stealth, scale=1]
\tkzTabInit[lgt=2,espcl=2]
{\(x\)/1,\(f'(x)\)/0.8,\(f(x)\)/3}
{\(-\infty\),\( -3 \),\( -2 \),\( -1 \),\(+\infty\)}
\tkzTabLine{-,0,+,0,-,0,+}
\path
(N12)node[shift={(0,-0.2)}](A){\(+\infty\)}
(N23)node[shift={(0,0.2)}](B){\(-3\)}
(N32)node[shift={(0,-1.5)}](C){\(-4\)}
(N43)node[shift={(0,0.2)}](D){\(2\)}
(N52)node[shift={(0,-0.2)}](E){\(+\infty\)};
\foreach\X/\Y in{A/B,B/C,C/D,D/E}\draw[->](\X)--(\Y);
\end{tikzpicture}


Hàm số có cực đại là giá trị nào?

A. \((-2,-4)\)

B. \(x=-3\)

*C. \(y=-3\)

D. \((-3,2)\)

Lời giải:


Dựa vào bảng biến thiên, ta xác định được:

- Hàm số có các điểm cực trị tại: \(x = -3, x = -2, x = -1\)

- Giá trị cực đại: \(y = -3\) tại \(x = -3\)

- Giá trị cực tiểu: \(y = -4\) tại \(x = -2\)

- Điểm cực đại: \((-3, -3)\)

- Điểm cực tiểu: \((-2, -4)\)

Từ đó suy ra đáp án cho câu hỏi đã cho.




Câu 8: Cho đồ thị hàm số \(y=f(x)\) như hình vẽ dưới đây:


\begin{tikzpicture}[scale=1, font=\footnotesize, line join=round, line cap=round, >=stealth]
\draw[->] (-3,0)--(3,0) node[below] {\(x\)};
\draw[->] (0,-3.5)--(0,2.5) node[left] {\(y\)};
\draw[fill=black] (0,0) circle (1pt) node[above left=-2pt] {\(O\)};

% Đánh dấu các điểm cực trị - VỊ TRÍ TRÊN HÌNH VẼ CỐ ĐỊNH
\draw[fill=black] (-1,0) circle (1pt) node[below] {\(-2\)};
\draw[fill=black] (1,0) circle (1pt) node[below] {\(3\)};
\draw[fill=black] (0,-2) circle (1pt) node[above left] {\(-2\)};
\draw[fill=black] (0,-3) circle (1pt);
\draw[fill=black] (0,-3.12) node[above left] {\(-4\)};

% Đường kẻ phụ - VỊ TRÍ TRÊN HÌNH VẼ CỐ ĐỊNH
\draw[dashed] (-1,0)--(-1,-3)--(1,-3)--(1,0);

% Vẽ đường cong hàm số bậc 4
\begin{scope}
\clip (-2,-3.5) rectangle (2,3.25);
\draw[smooth,samples=100,domain=-1.8:1.8] plot(\x,{(\x)^4-2*(\x)^2-2});
\end{scope}
\end{tikzpicture}


Hàm số đạt cực đại tại điểm nào?

A. \((-3,3)\)

B. \((2,-2)\)

*C. \(x=0\)

D. \((-4,-2)\)

Lời giải:


            Quan sát đồ thị hàm số, ta thấy:

- Hàm số có các điểm cực trị tại: \(x = -2, x = 0, x = 3\)

- Đỉnh cao nhất (cực đại) tại điểm \((0, -2)\)

- Đỉnh thấp nhất (cực tiểu) tại điểm \((-2, -4)\) và \((3, -4)\)

- Các giá trị số nguyên tương ứng trên đồ thị

Từ đó suy ra đáp án cho câu hỏi đã cho.




Câu 9: Cho đồ thị hàm số \(y=f(x)\) như hình vẽ dưới đây:


\begin{tikzpicture}[scale=1, font=\footnotesize, line join=round, line cap=round, >=stealth]
\draw[->] (-3,0)--(3,0) node[below] {\(x\)};
\draw[->] (0,-3.5)--(0,2.5) node[left] {\(y\)};
\draw[fill=black] (0,0) circle (1pt) node[above left=-2pt] {\(O\)};

% Đánh dấu các điểm cực trị - VỊ TRÍ TRÊN HÌNH VẼ CỐ ĐỊNH
\draw[fill=black] (-1,0) circle (1pt) node[below] {\(-1\)};
\draw[fill=black] (1,0) circle (1pt) node[below] {\(2\)};
\draw[fill=black] (0,-2) circle (1pt) node[above left] {\(-2\)};
\draw[fill=black] (0,-3) circle (1pt);
\draw[fill=black] (0,-3.12) node[above left] {\(-4\)};

% Đường kẻ phụ - VỊ TRÍ TRÊN HÌNH VẼ CỐ ĐỊNH
\draw[dashed] (-1,0)--(-1,-3)--(1,-3)--(1,0);

% Vẽ đường cong hàm số bậc 4
\begin{scope}
\clip (-2,-3.5) rectangle (2,3.25);
\draw[smooth,samples=100,domain=-1.8:1.8] plot(\x,{(\x)^4-2*(\x)^2-2});
\end{scope}
\end{tikzpicture}


Đồ thị hàm số có điểm cực đại là điểm nào?

A. \(y=-1\)

B. \((-4,-2)\)

*C. \((0,-2)\)

D. \(x=2\)

Lời giải:


            Quan sát đồ thị hàm số, ta thấy:

- Hàm số có các điểm cực trị tại: \(x = -1, x = 0, x = 2\)

- Đỉnh cao nhất (cực đại) tại điểm \((0, -2)\)

- Đỉnh thấp nhất (cực tiểu) tại điểm \((-1, -4)\) và \((2, -4)\)

- Các giá trị số nguyên tương ứng trên đồ thị

Từ đó suy ra đáp án cho câu hỏi đã cho.




Câu 10: Cho đồ thị hàm số \(y=f(x)\) như hình vẽ dưới đây:


\begin{tikzpicture}[line join=round, line cap=round,>=stealth,scale=1]
\tikzset{label style/.style={font=\footnotesize}}
\draw[->] (-2.1,0)--(2.5,0) node[below right] {\(x\)};
\draw[->] (0,-3.1)--(0,2.1) node[below left] {\(y\)};
\draw (0,0) node [below right] {\(O\)}circle(1.5pt);

% Đánh dấu các điểm cực trị - VỊ TRÍ TRÊN HÌNH VẼ CỐ ĐỊNH
\draw[dashed,thin](-1,0)--(-1,1)--(0,1);
\draw[dashed,thin](1,0)--(1,-3)--(0,-3);
\draw (1,0) node[above]{\( 7 \)}; 
\draw (-1,0) node[below]{\( -3 \)};
\draw (0,-3) node[left]{\( -1 \)};
\draw (0,1) node[right]{\( 4 \)};

% Vẽ đường cong hàm số bậc 3 loại 1
\begin{scope}
\clip (-2,-3) rectangle (2,2);
\draw[samples=200,domain=-2:2,smooth,variable=\x] plot (\x,{(\x)^3-3*(\x)-1});
\end{scope}
\end{tikzpicture}


Hàm số đạt cực trị tại điểm nào?

A. \((2,-1)\)

B. \((-3,-2)\)

C. \(x=4\)

*D. \(x=-3\) hoặc \(x=7\)

Lời giải:


            Quan sát đồ thị hàm số, ta thấy:

- Hàm số có các điểm cực trị tại: \(x = -3, x = 7\)

- Đỉnh cao nhất (cực đại) tại điểm \((-3, 4)\)

- Đỉnh thấp nhất (cực tiểu) tại điểm \((7, -1)\)

- Các giá trị số nguyên tương ứng trên đồ thị

Từ đó suy ra đáp án cho câu hỏi đã cho.



\end{document}
\documentclass[a4paper,12pt]{article}
\usepackage{amsmath}
\usepackage{amsfonts}
\usepackage{amssymb}
\usepackage{geometry}
\geometry{a4paper, margin=1in}
\usepackage{polyglossia}
\setmainlanguage{vietnamese}
\setmainfont{Times New Roman}
\begin{document}

Câu hỏi Đúng/Sai về Tam Giác

Câu 1: Cho \( \triangle SKL \) với \( S(-2, 0, -1) \), \( K(-6, 1, 1) \), \( L(11, 2, -5) \). Chọn các lựa chọn đúng:

a) Tọa độ chân đường phân giác kẻ từ S xuống KL là \( D(\frac{-3}{4}, \frac{5}{4}, \frac{-1}{2}) \).

b) Độ dài đường cao kẻ từ L trong \( \triangle SKL \) = \( \frac{23\sqrt{5}}{2\sqrt{21}} \).

c) \( \triangle SKL \) có góc \( \widehat{SLK} \) = \( 0.7° \).

*d) Bốn điểm S, K, L, D(5; 4; 3m + 0) đồng phẳng khi m = \(-\frac{3}{5}\).

Lời giải cho mệnh đề a):

\(\left. \begin{array}{l}
SK = \sqrt{21} \\
SL = 3\sqrt{21} \\
\text{Do SD là đường phân giác của } \widehat{KSL}
\end{array} \right\} \Rightarrow \frac{SK}{SL} = \frac{KD}{DL} = \frac{1}{3}\)

\(\Rightarrow \overrightarrow{KD} = \frac{1}{3}\overrightarrow{DL} \Leftrightarrow D - K = \frac{1}{3}(L - D) \Leftrightarrow \frac{4}{3}D = \frac{1}{3}L + K\)

\(\Leftrightarrow D = \frac{\frac{1}{3}L + K}{\frac{4}{3}} = \frac{\frac{1}{3}(11, 2, -5) + (-6, 1, 1)}{\frac{4}{3}} = (\frac{-7}{4}, \frac{5}{4}, \frac{-1}{2})\)

Lời giải cho mệnh đề b):

\([\overrightarrow{SK}, \overrightarrow{SL}] = (-8, 10, -21) \Rightarrow |[\overrightarrow{SK}, \overrightarrow{SL}]| = 11\sqrt{5}\)

\(\Rightarrow S_{\triangle SKL} = \frac{1}{2} |[\overrightarrow{SK}, \overrightarrow{SL}]| = \frac{11\sqrt{5}}{2}\)

\(SK = \sqrt{21}\).

Ta có: \({}S_{\triangle SKL} = \frac{1}{2} LI \cdot SK \Leftrightarrow LI = \frac{2 \cdot S_{\triangle SKL}}{SK} = \frac{11\sqrt{5}}{\sqrt{21}}\)

Lời giải cho mệnh đề c):

\( \cos(\widehat{SLK}) = \frac{\overrightarrow{LS} \cdot \overrightarrow{LK}}{|\overrightarrow{LS}| \cdot |\overrightarrow{LK}|} = \frac{247}{14 \cdot 18} = \frac{247}{252} \) \( \Rightarrow \widehat{SLK} = 5.7° \).

Lời giải cho mệnh đề d):

\([\overrightarrow{SK}, \overrightarrow{SL}] = (-8; 10; -21)\)

\(\overrightarrow{SD} = (7; 4; 3m + 1)\)

\(S, K, L, D \text{ đồng phẳng } \Leftrightarrow [\overrightarrow{SK}, \overrightarrow{SL}] \cdot \overrightarrow{SD} = 0\)

\(\Leftrightarrow -8 \cdot (7) + 10 \cdot (4) - 21 \cdot (3m + 1) = 0\)

\(\Leftrightarrow -56 + 40 + -21 \cdot (3m) + -21 \cdot 1 = 0\)

\(\Leftrightarrow -56 + 40 - 63m - 21 = 0\)

\(\Leftrightarrow -37 - 63m = 0\)

\(\Leftrightarrow m = -\frac{37}{63}\)



Câu 2: Cho \( \triangle BIH \) với \( B(-1, 0, 3) \), \( I(-1, 4, 0) \), \( H(7, -6, 3) \). Chọn các lựa chọn đúng:

*a) Tọa độ chân đường phân giác kẻ từ B xuống IH là \( D(\frac{5}{3}, \frac{2}{3}, 1) \).

*b) Độ dài đường cao kẻ từ I trong \( \triangle BIH \) = \( \frac{\sqrt{481}}{5} \).

*c) \( \triangle BIH \) có góc \( \widehat{BHI} \) = \( 19.5° \).

*d) Bốn điểm B, I, H, D(-1; -3; -3m + 1) đồng phẳng khi m = \(-\frac{7}{5}\).

Lời giải cho mệnh đề a):

\(\left. \begin{array}{l}
BI = 5 \\
BH = 10 \\
\text{Do BD là đường phân giác của } \widehat{IBH}
\end{array} \right\} \Rightarrow \frac{BI}{BH} = \frac{ID}{DH} = \frac{1}{2}\)

\(\Rightarrow \overrightarrow{ID} = \frac{1}{2}\overrightarrow{DH} \Leftrightarrow D - I = \frac{1}{2}(H - D) \Leftrightarrow \frac{3}{2}D = \frac{1}{2}H + I\)

\(\Leftrightarrow D = \frac{\frac{1}{2}H + I}{\frac{3}{2}} = \frac{\frac{1}{2}(7, -6, 3) + (-1, 4, 0)}{\frac{3}{2}} = (\frac{5}{3}, \frac{2}{3}, 1)\)

Lời giải cho mệnh đề b):

\([\overrightarrow{BI}, \overrightarrow{BH}] = (-18, -24, -32) \Rightarrow |[\overrightarrow{BI}, \overrightarrow{BH}]| = 2\sqrt{481}\)

\(\Rightarrow S_{\triangle BIH} = \frac{1}{2} |[\overrightarrow{BI}, \overrightarrow{BH}]| = \sqrt{481}\)

\(BH = 10\).

Ta có: \({}S_{\triangle BIH} = \frac{1}{2} IK \cdot BH \Leftrightarrow IK = \frac{2 \cdot S_{\triangle BIH}}{BH} = \frac{\sqrt{481}}{5}\)

Lời giải cho mệnh đề c):

\( \cos(\widehat{BHI}) = \frac{\overrightarrow{HB} \cdot \overrightarrow{HI}}{|\overrightarrow{HB}| \cdot |\overrightarrow{HI}|} = \frac{124}{10 \cdot 13} = \frac{62}{65} \) \( \Rightarrow \widehat{BHI} = 19.5° \).

Lời giải cho mệnh đề d):

\([\overrightarrow{BI}, \overrightarrow{BH}] = (-18; -24; -32)\)

\(\overrightarrow{BD} = (0; -3; -3m - 2)\)

\(B, I, H, D \text{ đồng phẳng } \Leftrightarrow [\overrightarrow{BI}, \overrightarrow{BH}] \cdot \overrightarrow{BD} = 0\)

\(\Leftrightarrow -18 \cdot (0) - 24 \cdot (-3) - 32 \cdot (-3m - 2) = 0\)

\(\Leftrightarrow 0 + 72 + -32 \cdot (-3m) - -32 \cdot 2 = 0\)

\(\Leftrightarrow 0 + 72 + 96m + 64 = 0\)

\(\Leftrightarrow 136 + 96m = 0\)

\(\Leftrightarrow m = -\frac{17}{12}\)



\end{document}
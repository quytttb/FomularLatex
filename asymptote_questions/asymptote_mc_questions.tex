\documentclass[a4paper,12pt]{article}
\usepackage{amsmath}
\usepackage{amsfonts}
\usepackage{amssymb}
\usepackage{geometry}
\geometry{a4paper, margin=1in}
\usepackage{polyglossia}
\setmainlanguage{vietnamese}
\setmainfont{Times New Roman}
\begin{document}

\section*{Câu hỏi Trắc nghiệm về Tiệm cận xiên}

Câu 1: Cho hàm số \(y = \frac{-3x^{2} + 5x + 5}{-x - 1}\).

Phương trình đường tiệm cận xiên của đồ thị hàm số này là:

*A. \(y = 3x - 8\)

B. \(y = \frac{5}{2}x - 8\)

C. \(y = 3x - 9\)

D. \(y = 3x - 6\)

Lời giải:

\textbf{Giải:}

Ta có: \(y = \frac{-3x^{2} + 5x + 5}{-x - 1} = 3x - 8 - \frac{3}{-x - 1}\)

\(\Rightarrow \displaystyle\lim_{x \to +\infty} \left(\left(\frac{-3x^{2} + 5x + 5}{-x - 1}\right) - \left(3x - 8 - \frac{3}{-x - 1}\right)\right) = \displaystyle\lim_{x \to +\infty} \frac{-3}{-x - 1} = \displaystyle\lim_{x \to +\infty} \frac{\frac{-3}{x}}{-1 + \frac{-1}{x}} = 0\)

\(\Rightarrow\) Tiệm cận xiên: \(y = 3x - 8\)


Câu 2: Cho hàm số \(y = \frac{x^{2} - 4x - 4}{2x + 1}\).

Phương trình đường tiệm cận xiên của đồ thị hàm số này là:

A. \(y = \frac{1}{2}x + \frac{7}{4}\)

B. \(y = \frac{3}{2}x - \frac{9}{4}\)

C. \(y = \frac{1}{2}x - \frac{5}{4}\)

*D. \(y = \frac{1}{2}x - \frac{9}{4}\)

Lời giải:

\textbf{Giải:}

Ta có: \(y = \frac{x^{2} - 4x - 4}{2x + 1} = \frac{1}{2}x - \frac{9}{4} - \frac{\frac{7}{4}}{2x + 1}\)

\(\Rightarrow \displaystyle\lim_{x \to +\infty} \left(\left(\frac{x^{2} - 4x - 4}{2x + 1}\right) - \left(\frac{1}{2}x - \frac{9}{4} - \frac{\frac{7}{4}}{2x + 1}\right)\right) = \displaystyle\lim_{x \to +\infty} \frac{\frac{-7}{4}}{2x + 1} = \displaystyle\lim_{x \to +\infty} \frac{\frac{\frac{-7}{4}}{x}}{2 + \frac{1}{x}} = 0\)

\(\Rightarrow\) Tiệm cận xiên: \(y = \frac{1}{2}x - \frac{9}{4}\)


Câu 3: Cho hàm số \(y = \frac{x^{2} + 4x + 6}{-3x - 5}\).

Phương trình đường tiệm cận xiên của đồ thị hàm số này là:

*A. \(y = \frac{-1}{3}x - \frac{7}{9}\)

B. \(y = \frac{-1}{3}x + \frac{11}{9}\)

C. \(y = \frac{2}{3}x - \frac{7}{9}\)

D. \(y = \frac{-1}{3}x + \frac{47}{9}\)

Lời giải:

\textbf{Giải:}

Ta có: \(y = \frac{x^{2} + 4x + 6}{-3x - 5} = \frac{-1}{3}x - \frac{7}{9} + \frac{\frac{19}{9}}{-3x - 5}\)

\(\Rightarrow \displaystyle\lim_{x \to +\infty} \left(\left(\frac{x^{2} + 4x + 6}{-3x - 5}\right) - \left(\frac{-1}{3}x - \frac{7}{9} + \frac{\frac{19}{9}}{-3x - 5}\right)\right) = \displaystyle\lim_{x \to +\infty} \frac{\frac{19}{9}}{-3x - 5} = \displaystyle\lim_{x \to +\infty} \frac{\frac{\frac{19}{9}}{x}}{-3 + \frac{-5}{x}} = 0\)

\(\Rightarrow\) Tiệm cận xiên: \(y = \frac{-1}{3}x - \frac{7}{9}\)


Câu 4: Cho hàm số \(y = \frac{x^{2} + x - 5}{-x - 4}\).

Phương trình đường tiệm cận xiên của đồ thị hàm số này là:

A. \(y = -x + 5\)

B. \(y = -2x + 3\)

*C. \(y = -x + 3\)

D. \(y = -x + 2\)

Lời giải:

\textbf{Giải:}

Ta có: \(y = \frac{x^{2} + x - 5}{-x - 4} = -x + 3 + \frac{7}{-x - 4}\)

\(\Rightarrow \displaystyle\lim_{x \to +\infty} \left(\left(\frac{x^{2} + x - 5}{-x - 4}\right) - \left(-x + 3 + \frac{7}{-x - 4}\right)\right) = \displaystyle\lim_{x \to +\infty} \frac{7}{-x - 4} = \displaystyle\lim_{x \to +\infty} \frac{\frac{7}{x}}{-1 + \frac{-4}{x}} = 0\)

\(\Rightarrow\) Tiệm cận xiên: \(y = -x + 3\)


\end{document}
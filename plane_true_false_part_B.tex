\documentclass[a4paper,12pt]{article}
\usepackage{amsmath,amssymb}
\usepackage{geometry}
\geometry{a4paper, margin=1in}
\usepackage{polyglossia}
\setmainlanguage{vietnamese}
\setmainfont{Times New Roman}
\begin{document}

\section*{Các bài toán về viết phương trình mặt phẳng - Đúng/Sai}

Câu 1: Chọn các mệnh đề đúng.

*a) Mặt phẳng đi qua M(1;2;3) cắt ba trục tọa độ sao cho thể tích tứ diện OABC nhỏ nhất có phương trình $6x+3y+2z-18=0$.

*b) Với M(1;2;3), mặt phẳng (P) để $T=\dfrac{1}{OA^2}+\dfrac{1}{OB^2}+\dfrac{1}{OC^2}$ nhỏ nhất có dạng $x+ay+bz+c=0$.

*c) Mặt phẳng đi qua M(1;2;3) cắt ba trục tọa độ sao cho thể tích tứ diện OABC nhỏ nhất có phương trình $6x+3y+2z-18=0$.

*d) Mặt phẳng đi qua M(1;2;3) cắt ba trục tọa độ sao cho thể tích tứ diện OABC nhỏ nhất có phương trình $6x+3y+2z-18=0$.



Câu 2: Chọn các mệnh đề đúng.

*a) Với điểm M(1;2;3), gọi A, B, C lần lượt là hình chiếu của M trên các trục tọa độ. Khi đó phương trình mặt phẳng (ABC) là $6x+3y+2z-6=0$.

*b) Phương trình mặt phẳng qua A(1;0;0), B(0;-2;0), C(0;0;3) là $(P): -6x + 3y - 2z + 6 = 0$.

*c) Với điểm M(1;2;3), gọi A, B, C lần lượt là hình chiếu của M trên các trục tọa độ. Khi đó phương trình mặt phẳng (ABC) là $6x+3y+2z-6=0$.

*d) Với điểm M(1;2;3), gọi A, B, C lần lượt là hình chiếu của M trên các trục tọa độ. Khi đó phương trình mặt phẳng (ABC) là $6x+3y+2z-6=0$.



Câu 3: Chọn các mệnh đề đúng.

*a) Với điểm M(1;2;3), gọi A, B, C lần lượt là hình chiếu của M trên các trục tọa độ. Khi đó phương trình mặt phẳng (ABC) là $6x+3y+2z-6=0$.

b) Với điểm M(1;2;3), gọi A, B, C lần lượt là hình chiếu của M trên các trục tọa độ. Khi đó phương trình mặt phẳng (ABC) là $2x+3y+6z-6=0$.

c) Phương trình mặt phẳng qua A(1;0;0), B(0;-2;0), C(0;0;3) là $(P): -6x + 3y - 2z + 7 = 0$.

d) Phương trình mặt phẳng qua A(1;0;0), B(0;-2;0), C(0;0;3) là $(P): -6x + 3y - 2z + 7 = 0$.



\end{document}
\documentclass[a4paper,12pt]{article}
\usepackage{amsmath,amssymb}
\usepackage{geometry}
\geometry{a4paper, margin=1in}
\usepackage{polyglossia}
\setmainlanguage{vietnamese}
\setmainfont{Times New Roman}
\begin{document}

\section*{Các bài toán về viết phương trình mặt phẳng - Đúng/Sai}

Câu 1: Chọn các mệnh đề đúng.

a) Với M(1;2;3), mặt phẳng (P) để $T$ nhỏ nhất có dạng $ax+by+cz+d=0$ với $a=b=c=0$.

b) Cho điểm M(-3;2;4). Mặt phẳng song song với (ABC) có phương trình $3x-6y-4z+12=0$.

*c) Phương trình mặt phẳng qua A(1;0;0), B(0;-2;0), C(0;0;3) là $(P): -6x + 3y - 2z + 6 = 0$.

d) Với điểm M(1;2;3), gọi A, B, C lần lượt là hình chiếu của M trên các trục tọa độ. Khi đó phương trình mặt phẳng (ABC) là $2x+3y+6z-6=0$.



Câu 2: Chọn các mệnh đề đúng.

a) Cho điểm M(-3;2;4). Mặt phẳng song song với (ABC) có phương trình $3x-6y-4z+12=0$.

*b) Với điểm M(1;2;3), gọi A, B, C lần lượt là hình chiếu của M trên các trục tọa độ. Khi đó phương trình mặt phẳng (ABC) là $6x+3y+2z-6=0$.

c) Phương trình mặt phẳng qua A(1;0;0), B(0;-2;0), C(0;0;3) là $(P): -6x + 3y - 2z + 7 = 0$.

d) Với M(1;2;3), mặt phẳng (P) để $T$ nhỏ nhất có dạng $ax+by+cz+d=0$ với $a=b=c=0$.



\end{document}
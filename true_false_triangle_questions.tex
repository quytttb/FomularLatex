\documentclass[a4paper,12pt]{article}
\usepackage{amsmath}
\usepackage{amsfonts}
\usepackage{amssymb}
\usepackage{geometry}
\geometry{a4paper, margin=1in}
\usepackage{polyglossia}
\setmainlanguage{vietnamese}
\setmainfont{Times New Roman}
\begin{document}

Câu hỏi Trắc nghiệm về Tam Giác

Câu 1: Cho \( \triangle FLR \) với \( F(-2, 3, 3) \), \( L(2, 5, -3) \), \( R(0, 2, 6) \). Bốn điểm F, L, R, D(4; -2; -3m - 4) đồng phẳng khi m

A. \( \(-\frac{63}{10}\) \)

*B. \( \(-\frac{73}{10}\) \)

C. \( \(-\frac{93}{10}\) \)

D. \( \(-\frac{83}{10}\) \)

Lời giải:

\([\overrightarrow{FL}, \overrightarrow{FR}] = (0; -24; -8)\)

\(\overrightarrow{FD} = (6; -5; -3m - 7)\)

\(F, L, R, D \text{ đồng phẳng } \Leftrightarrow [\overrightarrow{FL}, \overrightarrow{FR}] \cdot \overrightarrow{FD} = 0\)

\(\Leftrightarrow 0 \cdot (6) + -24 \cdot (-5) + -8 \cdot (-3m - 7) = 0\)

\(\Leftrightarrow 0 + 120 + -8 \cdot (-3m) + -8 \cdot (-7) = 0\)

\(\Leftrightarrow 0 + 120 + 24m + 56 = 0\)

\(\Leftrightarrow 176 + 24m = 0\)

\(\Leftrightarrow m = -\frac{22}{3}\)

Câu 2: Cho \( \triangle GQF \) với \( G(3, 0, -2) \), \( Q(5, -4, -1) \), \( F(-1, 13, -4) \). Bốn điểm G, Q, F, D(3; 3; m + 4) đồng phẳng khi m

A. \( -5 \)

B. \( -7 \)

*C. \( -6 \)

D. \( -8 \)

Lời giải:

\([\overrightarrow{GQ}, \overrightarrow{GF}] = (-5; 0; 10)\)

\(\overrightarrow{GD} = (0; 3; m + 6)\)

\(G, Q, F, D \text{ đồng phẳng } \Leftrightarrow [\overrightarrow{GQ}, \overrightarrow{GF}] \cdot \overrightarrow{GD} = 0\)

\(\Leftrightarrow -5 \cdot (0) + 0 \cdot (3) + 10 \cdot (m + 6) = 0\)

\(\Leftrightarrow 0 + 0 + 10 \cdot m + 10 \cdot 6 = 0\)

\(\Leftrightarrow 0 + 0 + 10m + 60 = 0\)

\(\Leftrightarrow 60 + 10m = 0\)

\(\Leftrightarrow m = -6\)

Câu 3: Cho \( \triangle ICH \) với \( I(-1, 0, -2) \), \( C(-1, 3, 0) \), \( H(0, -10, 2) \). Bốn điểm I, C, H, D(4; 5; -2m + 8) đồng phẳng khi m

A. \( \(-\frac{119}{5}\) \)

*B. \( \(-\frac{233}{10}\) \)

C. \( \(-\frac{114}{5}\) \)

D. \( \(-\frac{253}{10}\) \)

Lời giải:

\([\overrightarrow{IC}, \overrightarrow{IH}] = (32; 2; -3)\)

\(\overrightarrow{ID} = (5; 5; -2m + 10)\)

\(I, C, H, D \text{ đồng phẳng } \Leftrightarrow [\overrightarrow{IC}, \overrightarrow{IH}] \cdot \overrightarrow{ID} = 0\)

\(\Leftrightarrow 32 \cdot (5) + 2 \cdot (5) + -3 \cdot (-2m + 10) = 0\)

\(\Leftrightarrow 160 + 10 + -3 \cdot (-2m) + -3 \cdot 10 = 0\)

\(\Leftrightarrow 160 + 10 + 6m + -30 = 0\)

\(\Leftrightarrow 140 + 6m = 0\)

\(\Leftrightarrow m = -\frac{70}{3}\)

Câu 4: Cho \( \triangle NQB \) với \( N(1, -3, 3) \), \( Q(7, -5, 7) \), \( B(-2, -4, 5) \). Độ dài đường cao kẻ từ Q trong \( \triangle NQB \)

*A. \( \frac{12\sqrt{5}}{\sqrt{14}} \)

B. \( \frac{12\sqrt{7}}{\sqrt{14}} \)

C. \( \frac{25\sqrt{5}}{2\sqrt{14}} \)

D. \( \frac{24\sqrt{5}}{\sqrt{14}} \)

Lời giải:

\([\overrightarrow{NQ}, \overrightarrow{NB}] = (0, -24, -12) \Rightarrow |[\overrightarrow{NQ}, \overrightarrow{NB}]| = 12\sqrt{5}\)

\(\Rightarrow S_{\triangle NQB} = \frac{1}{2} |[\overrightarrow{NQ}, \overrightarrow{NB}]| = 6\sqrt{5}\)

\(NB = \sqrt{14}\).

Ta có: \({}S_{\triangle NQB} = \frac{1}{2} QK \cdot NB \Leftrightarrow QK = \frac{2 \cdot S_{\triangle NQB}}{NB} = \frac{12\sqrt{5}}{\sqrt{14}}\)

Câu 5: Cho \( \triangle QRC \) với \( Q(-2, -3, -3) \), \( R(-1, -7, -5) \), \( C(0, 10, -7) \). Tọa độ chân đường phân giác kẻ từ Q xuống RC

A. \( D(\frac{-3}{4}, \frac{-11}{4}, \frac{-15}{2}) \)

*B. \( D(\frac{-3}{4}, \frac{-11}{4}, \frac{-11}{2}) \)

C. \( D(\frac{5}{4}, \frac{-11}{4}, \frac{-11}{2}) \)

D. \( D(\frac{-3}{4}, \frac{-3}{4}, \frac{-11}{2}) \)

Lời giải:

\(\left. \begin{array}{l}
QR = \sqrt{21} \\
QC = 3\sqrt{21} \\
\text{Do QD là đường phân giác của } \widehat{RQC}
\end{array} \right\} \Rightarrow \frac{QR}{QC} = \frac{RD}{DC} = \frac{1}{3}\)

\(\Rightarrow \overrightarrow{RD} = \frac{1}{3}\overrightarrow{DC} \Leftrightarrow D - R = \frac{1}{3}(C - D) \Leftrightarrow \frac{4}{3}D = \frac{1}{3}C + R\)

\(\Leftrightarrow D = \frac{\frac{1}{3}C + R}{\frac{4}{3}} = \frac{\frac{1}{3}(0, 10, -7) + (-1, -7, -5)}{\frac{4}{3}} = (\frac{-3}{4}, \frac{-11}{4}, \frac{-11}{2})\)

Câu 6: Cho \( \triangle KSP \) với \( K(0, -3, -2) \), \( S(2, -3, -5) \), \( P(-4, -4, -12) \). Độ dài đường cao kẻ từ K trong \( \triangle KSP \)

*A. \( \frac{\sqrt{1037}}{\sqrt{86}} \)

B. \( \frac{2\sqrt{1037}}{\sqrt{86}} \)

C. \( \frac{2\sqrt{1037}}{\sqrt{86}} \)

D. \( \frac{\sqrt{1038}}{\sqrt{86}} \)

Lời giải:

\([\overrightarrow{KS}, \overrightarrow{KP}] = (-3, 32, -2) \Rightarrow |[\overrightarrow{KS}, \overrightarrow{KP}]| = \sqrt{1037}\)

\(\Rightarrow S_{\triangle KSP} = \frac{1}{2} |[\overrightarrow{KS}, \overrightarrow{KP}]| = \frac{\sqrt{1037}}{2}\)

\(SP = \sqrt{86}\).

Ta có: \({}S_{\triangle KSP} = \frac{1}{2} KH \cdot SP \Leftrightarrow KH = \frac{2 \cdot S_{\triangle KSP}}{SP} = \frac{\sqrt{1037}}{\sqrt{86}}\)

Câu 7: Cho \( \triangle SCJ \) với \( S(-1, -1, -3) \), \( C(3, -3, -9) \), \( J(1, 0, -6) \). Tọa độ chân đường phân giác kẻ từ S xuống CJ

A. \( D(\frac{5}{3}, -1, -6) \)

B. \( D(\frac{5}{3}, -3, -7) \)

C. \( D(\frac{2}{3}, -1, -7) \)

*D. \( D(\frac{5}{3}, -1, -7) \)

Lời giải:

\(\left. \begin{array}{l}
SC = 2\sqrt{14} \\
SJ = \sqrt{14} \\
\text{Do SD là đường phân giác của } \widehat{CSJ}
\end{array} \right\} \Rightarrow \frac{SC}{SJ} = \frac{CD}{DJ} = 2\)

\(\Rightarrow \overrightarrow{CD} = 2\overrightarrow{DJ} \Leftrightarrow D - C = 2(J - D) \Leftrightarrow 3D = 2J + C\)

\(\Leftrightarrow D = \frac{2J + C}{3} = \frac{2(1, 0, -6) + (3, -3, -9)}{3} = (\frac{5}{3}, -1, -7)\)

Câu 8: Cho \( \triangle MDI \) với \( M(0, 0, 0) \), \( D(4, 0, 3) \), \( I(6, 8, 0) \). \( \triangle MDI \) có góc \( \widehat{MID} \)

A. \( 25.0° \)

B. \( 45.0° \)

*C. \( 30.0° \)

D. \( 35.0° \)

Lời giải:

\( \cos(\widehat{MID}) = \frac{\overrightarrow{IM} \cdot \overrightarrow{ID}}{|\overrightarrow{IM}| \cdot |\overrightarrow{ID}|} = \frac{76}{10 \cdot 9} = \frac{38}{45} \) \( \Rightarrow \widehat{MID} = 30.0° \).

\end{document}
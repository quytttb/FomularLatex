\documentclass[a4paper,12pt]{article}
\usepackage{amsmath,amssymb}
\usepackage{geometry}
\geometry{a4paper, margin=1in}
\usepackage{polyglossia}
\setmainlanguage{vietnamese}
\setmainfont{Times New Roman}
\begin{document}

\section*{Các bài toán về viết phương trình mặt phẳng - Đúng/Sai (Câu 7-10)}

Câu 7: Chọn các mệnh đề đúng.

*a) Mặt phẳng đi qua M(1;3;2) cắt ba trục tọa độ sao cho thể tích tứ diện OABC nhỏ nhất có phương trình \(6x + 2y + 3z - 18 = 0\).

b) Với M(3;1;3), mặt phẳng (P) để \(T\) nhỏ nhất có dạng \(ax+by+cz+d=0\) với \(a=b=c=0\).

*c) Mặt phẳng đi qua M(1;2;4) cắt ba trục tọa độ sao cho tứ diện OABC có thể tích nhỏ nhất. Thể tích nhỏ nhất đó bằng \(36\).

*d) Cho hai điểm C(0;0;1) và M(1;2;0). Mặt phẳng qua C, M đồng thời chắn trên các nửa trục dương Ox, Oy các đoạn thẳng bằng nhau có phương trình được xác định theo điều kiện toán học.



Câu 8: Chọn các mệnh đề đúng.

a) Mặt phẳng đi qua M(2;5;4) và cắt các trục tọa độ Ox, Oy, Oz lần lượt tại A, B, C sao cho M là trực tâm của tam giác ABC có phương trình \(2x + 5y + 4z - 49 = 0\).

b) Mặt phẳng đi qua điểm M(1;2;2) và cắt trục tọa độ Ox, Oy, Oz tại A, B, C sao cho M là trực tâm tam giác ABC có phương trình \(x + 2y + 2z - 6 = 0\).

c) Mặt phẳng song song với (Q): 2x + 2y + z = 0 và cách điểm M(0;1;1) khoảng 3 có phương trình \(2x + 2y + z - 13 = 0\).

*d) Mặt phẳng qua M(4;1;2) và qua giao tuyến của (α): 3x - 2y + z - 2 = 0 và (β): 2x - y + 2z - 1 = 0 có phương trình \(x - y - z - 1 = 0\).



Câu 9: Chọn các mệnh đề đúng.

a) Mặt phẳng đi qua M(6;3;2) và cắt các trục tọa độ tại A, B, C sao cho M là trực tâm tam giác ABC có phương trình \(6x + 3y + 2z - 52 = 0\).

*b) Mặt phẳng qua M(2;1;-1) và cắt các tia Ox, Oy, Oz tại A, B, C sao cho 4OA = 2OB = OC có phương trình \(4x + 2y + z - 9 = 0\).

*c) Mặt phẳng đi qua điểm G(2;-2;1) và cắt các trục tọa độ tại các điểm A, B, C sao cho G là trọng tâm của tam giác ABC có phương trình \(-2x + 2y - 4z + 12 = 0\).

d) Mặt phẳng song song với (Q): 2x + 1 = 0 và cách (Q) khoảng 2 có phương trình \(2x + 6 = 0\).



Câu 10: Chọn các mệnh đề đúng.

*a) Mặt phẳng vuông góc với (α): x + y - 2z - 1 = 0 và (β): 2x - y + z - 1 = 0, cách điểm O(1;2;-1) một khoảng bằng 3. Một trong hai mặt phẳng đó có phương trình \(-x - 5y - 3z + 8 - 3\sqrt{35} = 0\).

b) Mặt phẳng đi qua M(1;1;3) cắt ba trục tọa độ sao cho thể tích tứ diện OABC nhỏ nhất có phương trình \(3x + 3y + z - 7 = 0\).

c) Mặt phẳng đi qua M(1;1;1) cắt ba trục tọa độ sao cho thể tích tứ diện OABC nhỏ nhất có phương trình \(4x + y + z - 3 = 0\).

d) Với M(1;3;2), mặt phẳng (P) để \(T\) nhỏ nhất có dạng \(ax+by+cz+d=0\) với \(a=b=c=0\).



\end{document}
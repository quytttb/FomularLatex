\documentclass[a4paper,12pt]{article}
\usepackage{amsmath}
\usepackage{amsfonts}
\usepackage{amssymb}
\usepackage{geometry}
\geometry{a4paper, margin=1in}
\usepackage{polyglossia}
\setmainlanguage{vietnamese}
\usepackage{tikz}
\usepackage{tkz-tab}
\usepackage{tkz-euclide}
\usetikzlibrary{calc,decorations.pathmorphing,decorations.pathreplacing}
\begin{document}
\title{Câu hỏi Tối ưu hóa}
\author{dev}
\maketitle

Câu 1: Cho hàm số \(y=f(x)\) có bảng xét dấu \(f'(x)\) như dưới đây:

\begin{tikzpicture}[>=stealth, scale=1]
	\tkzTabInit[lgt=2,espcl=2]
	{$x$/1,$f''(x)$/0.8,$f'(x)$/3}
	{$-\infty$,$-2$,$1$,$5$,$+\infty$}
	\tkzTabLine{,+,0,-,0,+,0,-,}
	\path
	(N13)node[shift={(0,0.2)}](A){$+\infty$}
	(N22)node[shift={(0,-0.2)}](B){$2$}
	(N32)node[shift={(0,-1.5)}](C){$-4$}
	(N42)node[shift={(0,-0.2)}](D){$8$}
	(N53)node[shift={(0,0.2)}](E){$+\infty$};
	\foreach \X/\Y in {A/B,B/C,C/D,D/E} \draw[->](\X)--(\Y);
\end{tikzpicture}

Hàm số đạt cực trị tại điểm nào?

A. \((1,2)\)

B. \(y=-2\)

C. \(x=2\)

*D. \(x=5\)

Lời giải:


Dựa vào bảng xét dấu \(f'(x)\), ta xác định được:

- Dấu của \(f'(x)\): dương trên \((-\infty, -2)\), âm trên \((-2, 1)\), dương trên \((1, 5)\), âm trên \((5, +\infty)\)

- Hàm số có các điểm cực trị tại: \(x = -2, x = 1, x = 5\)

- Điểm cực đại: \(x = -2\) và \(x = 5\) (chuyển từ tăng sang giảm)

- Điểm cực tiểu: \(x = 1\) (chuyển từ giảm sang tăng)

- Hàm số đồng biến trên \((-\infty, -2) \cup (1, 5)\)

- Hàm số nghịch biến trên \((-2, 1) \cup (5, +\infty)\)

Từ đó suy ra đáp án cho câu hỏi đã cho.




Câu 2: Cho hàm số \(y=f(x)\) có bảng xét dấu \(f'(x)\) như dưới đây:

\begin{tikzpicture}[>=stealth, scale=1]
	\tkzTabInit[lgt=2,espcl=2]
	{$x$/1,$f''(x)$/0.8,$f'(x)$/3}
	{$-\infty$,$-5$,$1$,$6$,$+\infty$}
	\tkzTabLine{,-,0,+,0,-,0,+,}
	\path
	(N12)node[shift={(0,-0.2)}](A){$+\infty$}
	(N23)node[shift={(0,0.2)}](B){$-8$}
	(N32)node[shift={(0,-1.5)}](C){$8$}
	(N43)node[shift={(0,0.2)}](D){$-2$}
	(N52)node[shift={(0,-0.2)}](E){$+\infty$};
	\foreach \X/\Y in {A/B,B/C,C/D,D/E} \draw[->](\X)--(\Y);
\end{tikzpicture}

Hàm số đạt cực đại tại điểm nào?

A. \((3,-1)\)

B. \((-8,-2)\)

C. \(x=-4\)

*D. \(x=1\)

Lời giải:


Dựa vào bảng xét dấu \(f'(x)\), ta xác định được:

- Dấu của \(f'(x)\): âm trên \((-\infty, -5)\), dương trên \((-5, 1)\), âm trên \((1, 6)\), dương trên \((6, +\infty)\)

- Hàm số có các điểm cực trị tại: \(x = -5, x = 1, x = 6\)

- Điểm cực tiểu: \(x = -5\) và \(x = 6\) (chuyển từ giảm sang tăng)

- Điểm cực đại: \(x = 1\) (chuyển từ tăng sang giảm)

- Hàm số nghịch biến trên \((-\infty, -5) \cup (1, 6)\)

- Hàm số đồng biến trên \((-5, 1) \cup (6, +\infty)\)

Từ đó suy ra đáp án cho câu hỏi đã cho.



\end{document}
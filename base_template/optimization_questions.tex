\documentclass[a4paper,12pt]{article}
\usepackage{amsmath}
\usepackage{amsfonts}
\usepackage{amssymb}
\usepackage{geometry}
\geometry{a4paper, margin=1in}
\usepackage{polyglossia}
\setmainlanguage{vietnamese}
\setmainfont{Times New Roman}
\usepackage{tikz}
\usepackage{tkz-tab}
\usepackage{tkz-euclide}
\usetikzlibrary{calc,decorations.pathmorphing,decorations.pathreplacing}
\begin{document}
\title{Câu hỏi Tối ưu hóa}
\maketitle

Câu 1: Cho đồ thị hàm số \(y=f(x)\) như hình vẽ dưới đây:


\begin{tikzpicture}[line join=round, line cap=round,>=stealth,scale=1]
\tikzset{label style/.style={font=\footnotesize}}
\draw[->] (-2.1,0)--(2.5,0) node[below right] {\(x\)};
\draw[->] (0,-3.1)--(0,2.1) node[below left] {\(y\)};
\draw (0,0) node [below right] {\(O\)}circle(1.5pt);

% Đánh dấu các điểm cực trị
\draw[dashed,thin](-3,0)--(-3,-3)--(0,-3);
\draw[dashed,thin](2,0)--(2,3)--(0,3);
\draw (2,0) node[above]{\( 2+m \)}; 
\draw (-3,0) node[below]{\( -3 \)};
\draw (0,3) node[left]{\( 3 \)};
\draw (0,-3) node[right]{\( -3 \)};

% Vẽ đường cong hàm số bậc 3 loại 1
\begin{scope}
\clip (-2,-3) rectangle (2,2);
\draw[samples=200,domain=-2:2,smooth,variable=\x] plot (\x,{(\x)^3-3*(\x)-1});
\end{scope}
\end{tikzpicture}


Đồ thị hàm số có điểm cực tiểu là điểm nào?

A. \((3,2)\)

B. \(x=-3\)

*C. \((2,3)\)

D. \((-3,4)\)

Lời giải:


            Quan sát đồ thị hàm số, ta thấy:

- Hàm số có các điểm cực trị tại: \(x = -3, x = 2, x = 3\)

- Đỉnh cao nhất (cực đại) tại điểm \((-3, -3)\)

- Đỉnh thấp nhất (cực tiểu) tại điểm \((2, 3)\)

- Các giá trị số nguyên tương ứng trên đồ thị

Từ đó suy ra đáp án cho câu hỏi đã cho.




Câu 2: Cho đồ thị hàm số \(y=f(x)\) như hình vẽ dưới đây:


\begin{tikzpicture}[scale=1, font=\footnotesize, line join=round, line cap=round, >=stealth]
\draw[->] (-3,0)--(3,0) node[below] {\(x\)};
\draw[->] (0,-3.5)--(0,2.5) node[left] {\(y\)};
\draw[fill=black] (0,0) circle (1pt) node[above left=-2pt] {\(O\)};

% Đánh dấu các điểm cực trị
\draw[fill=black] (-1,0) circle (1pt) node[below] {\(-1\)};
\draw[fill=black] (2,0) circle (1pt) node[below] {\(2\)};
\draw[fill=black] (0,1) circle (1pt) node[above left] {\(1\)};
\draw[fill=black] (0,-3) circle (1pt);
\draw[fill=black] (0,-3.12) node[above left] {\(-4\)};

% Đường kẻ phụ  
\draw[dashed] (-1,0)--(-1,-4)--(2,-4)--(2,0);

% Vẽ đường cong hàm số bậc 4
\begin{scope}
\clip (-2,-3.5) rectangle (2,3.25);
\draw[smooth,samples=100,domain=-1.8:1.8] plot(\x,{(\x)^4-2*(\x)^2-2});
\end{scope}
\end{tikzpicture}


Hàm số đạt cực đại tại điểm nào?

A. \(y=2\)

B. \(y=-1\)

*C. \(x=-1\)

D. \((-4,1)\)

Lời giải:


            Quan sát đồ thị hàm số, ta thấy:

- Hàm số có các điểm cực trị tại: \(x = -1, x = 1, x = 2\)

- Đỉnh cao nhất (cực đại) tại điểm \((-1, -4)\)

- Đỉnh thấp nhất (cực tiểu) tại điểm \((1, 1)\)

- Các giá trị số nguyên tương ứng trên đồ thị

Từ đó suy ra đáp án cho câu hỏi đã cho.




Câu 3: Cho hàm số \(y=f(x)\) có bảng biến thiên như dưới đây:


\begin{tikzpicture}[>=stealth, scale=1]
\tkzTabInit[lgt=2,espcl=2]
{\(x\)/1,\(f'(x)\)/0.8,\(f(x)\)/3}
{\(-\infty\),\( -1 \),\( 1 \),\( 3 \),\(+\infty\)}
\tkzTabLine{-,0,+,0,-,0,+}
\path
(N12)node[shift={(0,-0.2)}](A){\(+\infty\)}
(N23)node[shift={(0,0.2)}](B){\(-1\)}
(N32)node[shift={(0,-1.5)}](C){\(4\)}
(N43)node[shift={(0,0.2)}](D){\(3\)}
(N52)node[shift={(0,-0.2)}](E){\(+\infty\)};
\foreach\X/\Y in{A/B,B/C,C/D,D/E}\draw[->](\X)--(\Y);
\end{tikzpicture}


Hàm số đạt cực trị tại điểm nào?

A. \(y=-1\)

B. \(x=-4\)

*C. \(x=-1\) hoặc \(x=1\) hoặc \(x=3\)

D. \((3,1)\)

Lời giải:


Dựa vào bảng biến thiên, ta xác định được:

- Hàm số có các điểm cực trị tại: \(x = -1, x = 1, x = 3\)

- Giá trị cực đại: \(y = -1\) tại \(x = -1\)

- Giá trị cực tiểu: \(y = 4\) tại \(x = 1\)

- Điểm cực đại: \((-1, -1)\)

- Điểm cực tiểu: \((1, 4)\)

Từ đó suy ra đáp án cho câu hỏi đã cho.




Câu 4: Cho đồ thị hàm số \(y=f(x)\) như hình vẽ dưới đây:


\begin{tikzpicture}[scale=1, font=\footnotesize, line join=round, line cap=round, >=stealth]
\draw[->] (-3,0)--(3,0) node[below] {\(x\)};
\draw[->] (0,-3.5)--(0,2.5) node[left] {\(y\)};
\draw[fill=black] (0,0) circle (1pt) node[above left=-2pt] {\(O\)};

% Đánh dấu các điểm cực trị
\draw[fill=black] (-3,0) circle (1pt) node[below] {\(-3\)};
\draw[fill=black] (-1,0) circle (1pt) node[below] {\(-1\)};
\draw[fill=black] (0,-3) circle (1pt) node[above left] {\(-3\)};
\draw[fill=black] (0,-3) circle (1pt);
\draw[fill=black] (0,-3.12) node[above left] {\(2\)};

% Đường kẻ phụ  
\draw[dashed] (-3,0)--(-3,2)--(-1,2)--(-1,0);

% Vẽ đường cong hàm số bậc 4
\begin{scope}
\clip (-2,-3.5) rectangle (2,3.25);
\draw[smooth,samples=100,domain=-1.8:1.8] plot(\x,{(\x)^4-2*(\x)^2-2});
\end{scope}
\end{tikzpicture}


Hàm số đạt cực đại tại điểm nào?

*A. \(x=-3\)

B. \((-1,-2)\)

C. \(x=2\)

D. \(x=5\)

Lời giải:


            Quan sát đồ thị hàm số, ta thấy:

- Hàm số có các điểm cực trị tại: \(x = -3, x = -2, x = -1\)

- Đỉnh cao nhất (cực đại) tại điểm \((-3, 2)\)

- Đỉnh thấp nhất (cực tiểu) tại điểm \((-2, -3)\)

- Các giá trị số nguyên tương ứng trên đồ thị

Từ đó suy ra đáp án cho câu hỏi đã cho.




Câu 5: Cho đồ thị hàm số \(y=f(x)\) như hình vẽ dưới đây:


\begin{tikzpicture}[line join=round, line cap=round,>=stealth,scale=1]
\tikzset{label style/.style={font=\footnotesize}}
\draw[->] (-2.1,0)--(2.5,0) node[below right] {\(x\)};
\draw[->] (0,-3.1)--(0,2.1) node[below left] {\(y\)};
\draw (0,0) node [below right] {\(O\)}circle(1.5pt);

% Đánh dấu các điểm cực trị
\draw[dashed,thin](-1,0)--(-1,1)--(0,1);
\draw[dashed,thin](1,0)--(1,-3)--(0,-3);
\draw (1,0) node[above]{\( -1 \)}; 
\draw (-1,0) node[below]{\( -2 \)};
\draw (0,-3) node[left]{\( -4 \)};
\draw (0,1) node[right]{\( -2 \)};

% Vẽ đường cong hàm số (ví dụ đơn giản)
\begin{scope}
\clip (-2,-3) rectangle (2,2);
\draw[samples=200,domain=-2:2,smooth,variable=\x] plot (\x,{(\x)^3-3*(\x)-1});
\end{scope}
\end{tikzpicture}


Đồ thị hàm số có điểm cực đại là điểm nào?

*A. \((-2,-2)\)

B. \((-2,-4)\)

C. \((-2,2)\)

D. \(y=-3\)

Lời giải:


            Quan sát đồ thị hàm số, ta thấy:

- Hàm số có các điểm cực trị tại: \(x = -2, x = -1, x = 1\)

- Đỉnh cao nhất (cực đại) tại điểm \((-2, -2)\)

- Đỉnh thấp nhất (cực tiểu) tại điểm \((-1, -4)\)

- Các giá trị số nguyên tương ứng trên đồ thị

Từ đó suy ra đáp án cho câu hỏi đã cho.




Câu 6: Cho đồ thị hàm số \(y=f(x)\) như hình vẽ dưới đây:


\begin{tikzpicture}[scale=1, font=\footnotesize, line join=round, line cap=round, >=stealth]
\draw[->] (-3,0)--(3,0) node[below] {\(x\)};
\draw[->] (0,-3.5)--(0,2.5) node[left] {\(y\)};
\draw[fill=black] (0,0) circle (1pt) node[above left=-2pt] {\(O\)};

% Đánh dấu các điểm cực trị
\draw[fill=black] (-1,0) circle (1pt) node[below] {\(-1\)};
\draw[fill=black] (3,0) circle (1pt) node[below] {\(3\)};
\draw[fill=black] (0,4) circle (1pt) node[above left] {\(4\)};
\draw[fill=black] (0,-3) circle (1pt);
\draw[fill=black] (0,-3.12) node[above left] {\(2\)};

% Đường kẻ phụ  
\draw[dashed] (-1,0)--(-1,2)--(3,2)--(3,0);

% Vẽ đường cong hàm số bậc 4
\begin{scope}
\clip (-2,-3.5) rectangle (2,3.25);
\draw[smooth,samples=100,domain=-1.8:1.8] plot(\x,{(\x)^4-2*(\x)^2-2});
\end{scope}
\end{tikzpicture}


Hàm số có cực tiểu là giá trị nào?

A. \(y=5\)

*B. \(y=4\)

C. \(x=2\)

D. \(y=-1\)

Lời giải:


            Quan sát đồ thị hàm số, ta thấy:

- Hàm số có các điểm cực trị tại: \(x = -1, x = 1, x = 3\)

- Đỉnh cao nhất (cực đại) tại điểm \((-1, 2)\)

- Đỉnh thấp nhất (cực tiểu) tại điểm \((1, 4)\)

- Các giá trị số nguyên tương ứng trên đồ thị

Từ đó suy ra đáp án cho câu hỏi đã cho.




Câu 7: Cho đồ thị hàm số \(y=f(x)\) như hình vẽ dưới đây:


\begin{tikzpicture}[scale=1, font=\footnotesize, line join=round, line cap=round, >=stealth]
\draw[->] (-3,0)--(3,0) node[below] {\(x\)};
\draw[->] (0,-3.5)--(0,2.5) node[left] {\(y\)};
\draw[fill=black] (0,0) circle (1pt) node[above left=-2pt] {\(O\)};

% Đánh dấu các điểm cực trị
\draw[fill=black] (-3,0) circle (1pt) node[below] {\(-3\)};
\draw[fill=black] (3,0) circle (1pt) node[below] {\(3\)};
\draw[fill=black] (0,-4) circle (1pt) node[above left] {\(-4\)};
\draw[fill=black] (0,-3) circle (1pt);
\draw[fill=black] (0,-3.12) node[above left] {\(-3\)};

% Đường kẻ phụ  
\draw[dashed] (-3,0)--(-3,-3)--(3,-3)--(3,0);

% Vẽ đường cong hàm số bậc 4
\begin{scope}
\clip (-2,-3.5) rectangle (2,3.25);
\draw[smooth,samples=100,domain=-1.8:1.8] plot(\x,{(\x)^4-2*(\x)^2-2});
\end{scope}
\end{tikzpicture}


Hàm số đồng biến/nghịch biến trên khoảng nào dưới đây?

A. \(x=-4\)

*B. \((-3;2)\)

C. \((-3,-1)\)

D. \(y=-3\)

Lời giải:


            Quan sát đồ thị hàm số, ta thấy:

- Hàm số có các điểm cực trị tại: \(x = -3, x = 2, x = 3\)

- Đỉnh cao nhất (cực đại) tại điểm \((-3, -3)\)

- Đỉnh thấp nhất (cực tiểu) tại điểm \((2, -4)\)

- Các giá trị số nguyên tương ứng trên đồ thị

Từ đó suy ra đáp án cho câu hỏi đã cho.




Câu 8: Cho hàm số \(y=f(x)\) có bảng biến thiên như dưới đây:


\begin{tikzpicture}[>=stealth, scale=1]
\tkzTabInit[lgt=2,espcl=2]
{\(x\)/1,\(f'(x)\)/0.8,\(f(x)\)/3}
{\(-\infty\),\( -3 \),\( -1 \),\( 2 \),\(+\infty\)}
\tkzTabLine{-,0,+,0,-,0,+}
\path
(N12)node[shift={(0,-0.2)}](A){\(+\infty\)}
(N23)node[shift={(0,0.2)}](B){\(-2\)}
(N32)node[shift={(0,-1.5)}](C){\(4\)}
(N43)node[shift={(0,0.2)}](D){\(2\)}
(N52)node[shift={(0,-0.2)}](E){\(+\infty\)};
\foreach\X/\Y in{A/B,B/C,C/D,D/E}\draw[->](\X)--(\Y);
\end{tikzpicture}


Hàm số đồng biến/nghịch biến trên khoảng nào dưới đây?

A. \((3,3)\)

B. \(x=-2\)

C. \((-2,4)\)

*D. \((-3;-1)\)

Lời giải:


Dựa vào bảng biến thiên, ta xác định được:

- Hàm số có các điểm cực trị tại: \(x = -3, x = -1, x = 2\)

- Giá trị cực đại: \(y = -2\) tại \(x = -3\)

- Giá trị cực tiểu: \(y = 4\) tại \(x = -1\)

- Điểm cực đại: \((-3, -2)\)

- Điểm cực tiểu: \((-1, 4)\)

Từ đó suy ra đáp án cho câu hỏi đã cho.




Câu 9: Cho đồ thị hàm số \(y=f(x)\) như hình vẽ dưới đây:


\begin{tikzpicture}[line join=round, line cap=round,>=stealth,scale=1]
\tikzset{label style/.style={font=\footnotesize}}
\draw[->] (-2.1,0)--(2.5,0) node[below right] {\(x\)};
\draw[->] (0,-3.1)--(0,2.1) node[below left] {\(y\)};
\draw (0,0) node [below right] {\(O\)}circle(1.5pt);

% Đánh dấu các điểm cực trị
\draw[dashed,thin](-1,0)--(-1,1)--(0,1);
\draw[dashed,thin](1,0)--(1,-3)--(0,-3);
\draw (1,0) node[above]{\( 1 \)}; 
\draw (-1,0) node[below]{\( -1 \)};
\draw (0,-3) node[left]{\( -4 \)};
\draw (0,1) node[right]{\( -1 \)};

% Vẽ đường cong hàm số (ví dụ đơn giản)
\begin{scope}
\clip (-2,-3) rectangle (2,2);
\draw[samples=200,domain=-2:2,smooth,variable=\x] plot (\x,{(\x)^3-3*(\x)-1});
\end{scope}
\end{tikzpicture}


Hàm số đạt cực trị tại điểm nào?

A. \(x=-1\)

*B. \(x=-1\) hoặc \(x=1\) hoặc \(x=2\)

C. \((2,1)\)

D. \(x=5\)

Lời giải:


            Quan sát đồ thị hàm số, ta thấy:

- Hàm số có các điểm cực trị tại: \(x = -1, x = 1, x = 2\)

- Đỉnh cao nhất (cực đại) tại điểm \((-1, -1)\)

- Đỉnh thấp nhất (cực tiểu) tại điểm \((1, -4)\)

- Các giá trị số nguyên tương ứng trên đồ thị

Từ đó suy ra đáp án cho câu hỏi đã cho.




Câu 10: Cho đồ thị hàm số \(y=f(x)\) như hình vẽ dưới đây:


\begin{tikzpicture}[line join=round, line cap=round,>=stealth,scale=1]
\tikzset{label style/.style={font=\footnotesize}}
\draw[->] (-2.1,0)--(2.5,0) node[below right] {\(x\)};
\draw[->] (0,-3.1)--(0,2.1) node[below left] {\(y\)};
\draw (0,0) node [below right] {\(O\)}circle(1.5pt);

% Đánh dấu các điểm cực trị
\draw[dashed,thin](-1,0)--(-1,1)--(0,1);
\draw[dashed,thin](1,0)--(1,-3)--(0,-3);
\draw (1,0) node[above]{\( -1 \)}; 
\draw (-1,0) node[below]{\( -2 \)};
\draw (0,-3) node[left]{\( 4 \)};
\draw (0,1) node[right]{\( -1 \)};

% Vẽ đường cong hàm số (ví dụ đơn giản)
\begin{scope}
\clip (-2,-3) rectangle (2,2);
\draw[samples=200,domain=-2:2,smooth,variable=\x] plot (\x,{(\x)^3-3*(\x)-1});
\end{scope}
\end{tikzpicture}


Hàm số đạt cực đại tại điểm nào?

*A. \(x=-2\)

B. \((1,-4)\)

C. \((-2,1)\)

D. \(x=-1\)

Lời giải:


            Quan sát đồ thị hàm số, ta thấy:

- Hàm số có các điểm cực trị tại: \(x = -2, x = -1, x = 3\)

- Đỉnh cao nhất (cực đại) tại điểm \((-2, -1)\)

- Đỉnh thấp nhất (cực tiểu) tại điểm \((-1, 4)\)

- Các giá trị số nguyên tương ứng trên đồ thị

Từ đó suy ra đáp án cho câu hỏi đã cho.



\end{document}
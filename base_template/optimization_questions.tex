\documentclass[a4paper,12pt]{article}
\usepackage{amsmath}
\usepackage{amsfonts}
\usepackage{amssymb}
\usepackage{geometry}
\geometry{a4paper, margin=1in}
\usepackage{polyglossia}
\setmainlanguage{vietnamese}
\setmainfont{Times New Roman}
\usepackage{tikz}
\usepackage{tkz-tab}
\usepackage{tkz-euclide}
\usetikzlibrary{calc,decorations.pathmorphing,decorations.pathreplacing}
\begin{document}
\title{Câu hỏi Tối ưu hóa}
\maketitle

Câu 1: Cho đồ thị hàm số \(y=f(x)\) như hình vẽ dưới đây:


\begin{tikzpicture}[scale=1, font=\footnotesize, line join=round, line cap=round, >=stealth]
\draw[->] (-2.5,0)--(3.5,0) node[below] {\(x\)};
\draw[->] (0,-3.5)--(0,2.5) node[left] {\(y\)};
\draw[fill=black] (0,0) circle (1pt) node[below left=-2pt] {\(O\)};

% Đánh dấu các điểm cực trị
\draw[fill=black] (-2,0) circle (1pt) node[below] {\(-2\)};
\draw[fill=black] (1,0) circle (1pt) node[below] {\(1\)};
\draw[fill=black] (0,3) circle (1pt) node[above left] {\(3\)};
\draw[fill=black] (0,1) circle (1pt) node[below left] {\(1\)};

% Đường kẻ phụ
\draw[dashed] (-2,0)--(-2,1)--(1,1)--(1,0);
\draw[dashed] (-2,0)--(-2,3)--(-1,3)--(-1,0);

% Vẽ đường cong hàm số bậc 3 loại 2 (ngược)
\begin{scope}
\clip (-2.5,-3.5) rectangle (3.5,2.5);
\draw[smooth,samples=100,domain=-2.5:3.5] plot(\x,{-1*(\x)^3+3*(\x)-1});
\end{scope}
\end{tikzpicture}


Hàm số có cực tiểu là giá trị nào?

A. \((-5,-4)\)

B. \(x=1\)

*C. \(y=1\)

D. \(x=1\)

Lời giải:


            Quan sát đồ thị hàm số, ta thấy:

- Hàm số có các điểm cực trị tại: \(x = -2, x = 1\)

- Đỉnh cao nhất (cực đại) tại điểm \((1, 3)\)

- Đỉnh thấp nhất (cực tiểu) tại điểm \((-2, 1)\)

- Các giá trị số nguyên tương ứng trên đồ thị

Từ đó suy ra đáp án cho câu hỏi đã cho.




Câu 2: Cho đồ thị hàm số \(y=f(x)\) như hình vẽ dưới đây:


\begin{tikzpicture}[line join=round, line cap=round,>=stealth,scale=1]
\tikzset{label style/.style={font=\footnotesize}}
\draw[->] (-2.1,0)--(2.5,0) node[below right] {\(x\)};
\draw[->] (0,-3.1)--(0,2.1) node[below left] {\(y\)};
\draw (0,0) node [below right] {\(O\)}circle(1.5pt);

% Đánh dấu các điểm cực trị
\draw[dashed,thin](-3,0)--(-3,4)--(0,4);
\draw[dashed,thin](2,0)--(2,1)--(0,1);
\draw (2,0) node[above]{\( 2 \)}; 
\draw (-3,0) node[below]{\( -3 \)};
\draw (0,1) node[left]{\( 1 \)};
\draw (0,4) node[right]{\( 4 \)};

% Vẽ đường cong hàm số bậc 3 loại 1
\begin{scope}
\clip (-2,-3) rectangle (2,2);
\draw[samples=200,domain=-2:2,smooth,variable=\x] plot (\x,{(\x)^3-3*(\x)-1});
\end{scope}
\end{tikzpicture}


Đồ thị hàm số có điểm cực đại là điểm nào?

A. \(x=4\)

B. \((2,1)\)

*C. \((-3,4)\)

D. \((4,1)\)

Lời giải:


            Quan sát đồ thị hàm số, ta thấy:

- Hàm số có các điểm cực trị tại: \(x = -3, x = 2\)

- Đỉnh cao nhất (cực đại) tại điểm \((-3, 4)\)

- Đỉnh thấp nhất (cực tiểu) tại điểm \((2, 1)\)

- Các giá trị số nguyên tương ứng trên đồ thị

Từ đó suy ra đáp án cho câu hỏi đã cho.




Câu 3: Cho đồ thị hàm số \(y=f(x)\) như hình vẽ dưới đây:


\begin{tikzpicture}[scale=1, font=\footnotesize, line join=round, line cap=round, >=stealth]
\draw[->] (-2.5,0)--(3.5,0) node[below] {\(x\)};
\draw[->] (0,-3.5)--(0,2.5) node[left] {\(y\)};
\draw[fill=black] (0,0) circle (1pt) node[below left=-2pt] {\(O\)};

% Đánh dấu các điểm cực trị
\draw[fill=black] (-3,0) circle (1pt) node[below] {\(-3\)};
\draw[fill=black] (1,0) circle (1pt) node[below] {\(1\)};
\draw[fill=black] (0,3) circle (1pt) node[above left] {\(3\)};
\draw[fill=black] (0,-1) circle (1pt) node[below left] {\(-1\)};

% Đường kẻ phụ
\draw[dashed] (-3,0)--(-3,-1)--(1,-1)--(1,0);
\draw[dashed] (-2,0)--(-2,3)--(-1,3)--(-1,0);

% Vẽ đường cong hàm số bậc 3 loại 2 (ngược)
\begin{scope}
\clip (-2.5,-3.5) rectangle (3.5,2.5);
\draw[smooth,samples=100,domain=-2.5:3.5] plot(\x,{-1*(\x)^3+3*(\x)-1});
\end{scope}
\end{tikzpicture}


Hàm số đạt cực đại tại điểm nào?

A. \(x=-1\)

*B. \(x=1\)

C. \((-2,-4)\)

D. \(y=4\)

Lời giải:


            Quan sát đồ thị hàm số, ta thấy:

- Hàm số có các điểm cực trị tại: \(x = -3, x = 1\)

- Đỉnh cao nhất (cực đại) tại điểm \((1, 3)\)

- Đỉnh thấp nhất (cực tiểu) tại điểm \((-3, -1)\)

- Các giá trị số nguyên tương ứng trên đồ thị

Từ đó suy ra đáp án cho câu hỏi đã cho.



\end{document}
Câu 1: Vận tốc của một vật thể chuyển động được mô tả bởi hàm số \(f(t) = 2t^{3}- t^{2}+ 4\) (m/s) sau \(t\) giây. 
        Hỏi vận tốc của vật thể giảm trong khoảng thời gian nào trong 20 giây đầu?

A. từ giây thứ 0 đến 2

*B. không có khoảng thời gian nào

C. không có khoảng thời gian nào

D. từ giây thứ 0 đến 20

Lời giải:

Đề bài: vận tốc của vật thể: \(f(t) = 2t^{3}- t^{2}+ 4\) (m/s).

Tập xác định: \(D = \mathbb{R}\)
Với điều kiện Thời gian t \(\geq\) 0.

Bước 1: Tính đạo hàm
\[
f'(t) = 6t^{2}- 2t
\]

Bước 2: Giải phương trình đạo hàm
\[
f'(t) = 0 \Rightarrow t = 0; \quad t = \frac{1}{3}
\]



Bước 3: Xét dấu của đạo hàm
Xét dấu của \(f'(t)\):

\(f'(t) > 0\) trên các khoảng: \((\frac{1}{3}, 20)\)
  \(\Rightarrow\) vận tốc của vật thể tăng trên các khoảng này

\(f'(t) < 0\) trên các khoảng: \((0, \frac{1}{3})\)
  \(\Rightarrow\) vận tốc của vật thể giảm trên các khoảng này
  Nhận xét quan trọng: Khoảng \((0, \frac{1}{3})\) có độ dài \(\frac{1}{3}\), quá hẹp để có ý nghĩa thực tế đáng kể.
  \(\Rightarrow\) Trong thực tế, khoảng này không đủ rõ ràng để kết luận vận tốc của vật thể giảm trên một khoảng thời gian có ý nghĩa.

Khi đạo hàm f'(t) < 0, vận tốc giảm có nghĩa là vật thể chuyển động chậm dần.

Kết luận: Vậy vận tốc của vật thể giảm trong khoảng không có khoảng thời gian nào \(\rightarrow\) Chọn đáp án tương ứng.


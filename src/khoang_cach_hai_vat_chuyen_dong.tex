
\documentclass[a4paper,12pt]{article}
\usepackage{amsmath}
\usepackage{mathtools}
\usepackage{amsfonts}
\usepackage{amssymb}
\usepackage{geometry}
\geometry{a4paper, margin=1in}
\usepackage{polyglossia}
\setmainlanguage{vietnamese}
\setmainfont{Times New Roman}
\usepackage{tikz}
\usepackage{tkz-tab}
\usepackage{tkz-euclide}
\usetikzlibrary{calc,decorations.pathmorphing,decorations.pathreplacing}
\begin{document}
\title{Bài toán}
\maketitle
Câu 1: 
Với một hình hộp chữ nhật \(ABCD.A'B'C'D'\) có \(AB=24\), \(AD=29\), \(AA'=9\) như hình vẽ. Ở cùng một thời điểm hai con kiến coi như bò chuyển động thẳng đều, con kiến \(M\) bò từ \(B'\) đến điểm \(A\) với tốc độ \(1.3\,\mathrm{cm/s}\) và con kiến \(N\) bò từ \(C\) đến \(D'\) với tốc độ bằng \(2.0\,\mathrm{cm/s}\). Hãy tính khoảng cách nhỏ nhất giữa hai con kiến theo đơn vị centimet (làm tròn kết quả đến hàng phần mười)?

\begin{tikzpicture}[scale=1.0]
% Định nghĩa các đỉnh của hình hộp với phép chiếu đúng
\coordinate (B) at (0,0);
\coordinate (C) at (4,0);
\coordinate (A) at (1.5,1.5);
\coordinate (D) at (5.5,1.5);
\coordinate (B') at (0,3);
\coordinate (C') at (4,3);
\coordinate (A') at (1.5,4.5);
\coordinate (D') at (5.5,4.5);
% Vẽ các cạnh nhìn thấy của hình hộp
\draw (B) -- (C);
\draw (B') -- (C') -- (D') -- (A') -- cycle;
\draw (B) -- (B');
\draw (C) -- (C');
\draw (C) -- (D);
\draw (D) -- (D');
% Vẽ các cạnh ẩn bằng đường đứt nét
\draw[dashed] (B) -- (A);
\draw[dashed] (A) -- (D);
\draw[dashed] (A') -- (A);
% Định nghĩa điểm M trên đường chéo AB'
\coordinate (M) at (0.75,2.25);
% Định nghĩa điểm N trên cạnh CD'
\coordinate (N) at (4.75,2.25);
% Vẽ đường chéo AB' bằng đường đứt nét đỏ (nhưng dùng màu đen theo yêu cầu)
\draw[dashed, thick] (A) -- (B');
% Vẽ đường từ C đến N đến D' (đường xanh trong gốc, nhưng dùng màu đen)
\draw[thick] (C) -- (N) -- (D');
% Vẽ các mũi tên
\draw[->, thick] (0.1,2.7) -- (0.5,2.3);  % Mũi tên từ B' về phía M
\draw[->, thick] (4.17,0.3) -- (4.57,1.4);
% Đánh dấu các điểm
\fill (A) circle (1.2pt);
\fill (B) circle (1.2pt);
\fill (C) circle (1.2pt);
\fill (D) circle (1.2pt);
\fill (A') circle (1.2pt);
\fill (B') circle (1.2pt);
\fill (C') circle (1.2pt);
\fill (D') circle (1.2pt);
\fill (M) circle (1.2pt);
\fill (N) circle (1.2pt);
% Gắn nhãn cho các điểm
\node[below right] at (A) {\(A\)};
\node[below left] at (B) {\(B\)};
\node[below right] at (C) {\(C\)};
\node[below right] at (D) {\(D\)};
\node[above right] at (A') {\(A'\)};
\node[above left] at (B') {\(B'\)};
\node[above left] at (C') {\(C'\)};
\node[above right] at (D') {\(D'\)};
\node[right] at (M) {\(M\)};
\node[right] at (N) {\(N\)};
\end{tikzpicture}


Lời giải:


Dữ kiện:\\
+ Hình hộp chữ nhật \(ABCD.A'B'C'D'\) có:\\
\[
AB = 24,\quad AD = 29,\quad AA' = 9
\]
+ Gán tọa độ:
\[
A = (0, 0, 0),\quad B = (24, 0, 0),\quad D = (0, 29, 0),\quad B' = (24, 0, 9),\quad D' = (0, 29, 9)
\]
+ Con kiến \(M\) bò từ \(B'\) đến \(A\) với tốc độ \(1.3\, \text{cm/s}\)\\
+ Con kiến \(N\) bò từ \(C\) đến \(D'\) với tốc độ \(2.0\, \text{cm/s}\)\\
Bước 1: Phương trình chuyển động\\
Con kiến \(M\):\\
Vector chỉ phương đường đi:
\[
\overrightarrow{B'A} = (0, 0, 0) - (24, 0, 9) = ( -24, 0, -9 )
\]
Chiều dài đoạn \(B'A\):
\[
|\overrightarrow{B'A}| = \sqrt{24^2 + 9^2} = \sqrt{576 + 81} = \sqrt{657}
\]
Trong một giây con kiến tại \(M\) đi được \(\frac{1.3}{\sqrt{657}}\) lần \(\overrightarrow{B'A}\)\\
Véctơ vận tốc của con kiến tại \(M\) là:
\[
\overrightarrow{v}_M = \frac{1.3}{\sqrt{657}} \cdot ( -24, 0, -9 )
\]
Vị trí tại thời điểm \(t\):
\[
M(t) = (24, 0, 9) + t \cdot \frac{1.3}{\sqrt{657}} \cdot ( -24, 0, -9 )
= \left( 24 - \frac{1.2t}{\sqrt{657}},\; 0,\; 9 - \frac{0.5t}{\sqrt{657}} \right)
\]
Con kiến \(N\): Làm tương tự ta có:\\
Vị trí tại thời điểm \(t\):
\[
N(t) = (24, 29, 0) + t \cdot \frac{2.0}{\sqrt{657}} \cdot (-24, 0, 9)
= \left( 24 - \frac{1.9t}{\sqrt{657}},\; 29,\; \frac{0.7t}{\sqrt{657}} \right)
\]
Bước 2: Tính khoảng cách giữa hai con kiến tại thời điểm \(t\)
\[
d(t) = \sqrt{\left(24 - \frac{1.2t}{\sqrt{657}}\right)^2 + 29^2 + \left(9 - \frac{0.5t}{\sqrt{657}} - 2.0t\right)^2}
\]
Bước 3: Tìm khoảng cách nhỏ nhất
\[
d'(t)=0\Leftrightarrow t \approx 5.88 \Rightarrow d_{min}=29.3cm\
\]




Câu 2: 
Với một hình hộp chữ nhật \(ABCD.A'B'C'D'\) có \(AB=11\), \(AD=15\), \(AA'=16\) như hình vẽ. Ở cùng một thời điểm hai con kiến coi như bò chuyển động thẳng đều, con kiến \(M\) bò từ \(B'\) đến điểm \(A\) với tốc độ \(2.8\,\mathrm{cm/s}\) và con kiến \(N\) bò từ \(C\) đến \(D'\) với tốc độ bằng \(3.3\,\mathrm{cm/s}\). Hãy tính khoảng cách nhỏ nhất giữa hai con kiến theo đơn vị centimet (làm tròn kết quả đến hàng phần mười)?

\begin{tikzpicture}[scale=1.0]
% Định nghĩa các đỉnh của hình hộp với phép chiếu đúng
\coordinate (B) at (0,0);
\coordinate (C) at (4,0);
\coordinate (A) at (1.5,1.5);
\coordinate (D) at (5.5,1.5);
\coordinate (B') at (0,3);
\coordinate (C') at (4,3);
\coordinate (A') at (1.5,4.5);
\coordinate (D') at (5.5,4.5);
% Vẽ các cạnh nhìn thấy của hình hộp
\draw (B) -- (C);
\draw (B') -- (C') -- (D') -- (A') -- cycle;
\draw (B) -- (B');
\draw (C) -- (C');
\draw (C) -- (D);
\draw (D) -- (D');
% Vẽ các cạnh ẩn bằng đường đứt nét
\draw[dashed] (B) -- (A);
\draw[dashed] (A) -- (D);
\draw[dashed] (A') -- (A);
% Định nghĩa điểm M trên đường chéo AB'
\coordinate (M) at (0.75,2.25);
% Định nghĩa điểm N trên cạnh CD'
\coordinate (N) at (4.75,2.25);
% Vẽ đường chéo AB' bằng đường đứt nét đỏ (nhưng dùng màu đen theo yêu cầu)
\draw[dashed, thick] (A) -- (B');
% Vẽ đường từ C đến N đến D' (đường xanh trong gốc, nhưng dùng màu đen)
\draw[thick] (C) -- (N) -- (D');
% Vẽ các mũi tên
\draw[->, thick] (0.1,2.7) -- (0.5,2.3);  % Mũi tên từ B' về phía M
\draw[->, thick] (4.17,0.3) -- (4.57,1.4);
% Đánh dấu các điểm
\fill (A) circle (1.2pt);
\fill (B) circle (1.2pt);
\fill (C) circle (1.2pt);
\fill (D) circle (1.2pt);
\fill (A') circle (1.2pt);
\fill (B') circle (1.2pt);
\fill (C') circle (1.2pt);
\fill (D') circle (1.2pt);
\fill (M) circle (1.2pt);
\fill (N) circle (1.2pt);
% Gắn nhãn cho các điểm
\node[below right] at (A) {\(A\)};
\node[below left] at (B) {\(B\)};
\node[below right] at (C) {\(C\)};
\node[below right] at (D) {\(D\)};
\node[above right] at (A') {\(A'\)};
\node[above left] at (B') {\(B'\)};
\node[above left] at (C') {\(C'\)};
\node[above right] at (D') {\(D'\)};
\node[right] at (M) {\(M\)};
\node[right] at (N) {\(N\)};
\end{tikzpicture}


Lời giải:


Dữ kiện:\\
+ Hình hộp chữ nhật \(ABCD.A'B'C'D'\) có:\\
\[
AB = 11,\quad AD = 15,\quad AA' = 16
\]
+ Gán tọa độ:
\[
A = (0, 0, 0),\quad B = (11, 0, 0),\quad D = (0, 15, 0),\quad B' = (11, 0, 16),\quad D' = (0, 15, 16)
\]
+ Con kiến \(M\) bò từ \(B'\) đến \(A\) với tốc độ \(2.8\, \text{cm/s}\)\\
+ Con kiến \(N\) bò từ \(C\) đến \(D'\) với tốc độ \(3.3\, \text{cm/s}\)\\
Bước 1: Phương trình chuyển động\\
Con kiến \(M\):\\
Vector chỉ phương đường đi:
\[
\overrightarrow{B'A} = (0, 0, 0) - (11, 0, 16) = ( -11, 0, -16 )
\]
Chiều dài đoạn \(B'A\):
\[
|\overrightarrow{B'A}| = \sqrt{11^2 + 16^2} = \sqrt{121 + 256} = \sqrt{377}
\]
Trong một giây con kiến tại \(M\) đi được \(\frac{2.8}{\sqrt{377}}\) lần \(\overrightarrow{B'A}\)\\
Véctơ vận tốc của con kiến tại \(M\) là:
\[
\overrightarrow{v}_M = \frac{2.8}{\sqrt{377}} \cdot ( -11, 0, -16 )
\]
Vị trí tại thời điểm \(t\):
\[
M(t) = (11, 0, 16) + t \cdot \frac{2.8}{\sqrt{377}} \cdot ( -11, 0, -16 )
= \left( 11 - \frac{1.6t}{\sqrt{377}},\; 0,\; 16 - \frac{2.3t}{\sqrt{377}} \right)
\]
Con kiến \(N\): Làm tương tự ta có:\\
Vị trí tại thời điểm \(t\):
\[
N(t) = (11, 15, 0) + t \cdot \frac{3.3}{\sqrt{377}} \cdot (-11, 0, 16)
= \left( 11 - \frac{1.9t}{\sqrt{377}},\; 15,\; \frac{2.7t}{\sqrt{377}} \right)
\]
Bước 2: Tính khoảng cách giữa hai con kiến tại thời điểm \(t\)
\[
d(t) = \sqrt{\left(11 - \frac{1.6t}{\sqrt{377}}\right)^2 + 15^2 + \left(16 - \frac{2.3t}{\sqrt{377}} - 3.3t\right)^2}
\]
Bước 3: Tìm khoảng cách nhỏ nhất
\[
d'(t)=0\Leftrightarrow t \approx 3.17 \Rightarrow d_{min}=15.0cm\
\]




Câu 3: 
Với một hình hộp chữ nhật \(ABCD.A'B'C'D'\) có \(AB=18\), \(AD=15\), \(AA'=15\) như hình vẽ. Ở cùng một thời điểm hai con kiến coi như bò chuyển động thẳng đều, con kiến \(M\) bò từ \(B'\) đến điểm \(A\) với tốc độ \(2.4\,\mathrm{cm/s}\) và con kiến \(N\) bò từ \(C\) đến \(D'\) với tốc độ bằng \(2.0\,\mathrm{cm/s}\). Hãy tính khoảng cách nhỏ nhất giữa hai con kiến theo đơn vị centimet (làm tròn kết quả đến hàng phần mười)?

\begin{tikzpicture}[scale=1.0]
% Định nghĩa các đỉnh của hình hộp với phép chiếu đúng
\coordinate (B) at (0,0);
\coordinate (C) at (4,0);
\coordinate (A) at (1.5,1.5);
\coordinate (D) at (5.5,1.5);
\coordinate (B') at (0,3);
\coordinate (C') at (4,3);
\coordinate (A') at (1.5,4.5);
\coordinate (D') at (5.5,4.5);
% Vẽ các cạnh nhìn thấy của hình hộp
\draw (B) -- (C);
\draw (B') -- (C') -- (D') -- (A') -- cycle;
\draw (B) -- (B');
\draw (C) -- (C');
\draw (C) -- (D);
\draw (D) -- (D');
% Vẽ các cạnh ẩn bằng đường đứt nét
\draw[dashed] (B) -- (A);
\draw[dashed] (A) -- (D);
\draw[dashed] (A') -- (A);
% Định nghĩa điểm M trên đường chéo AB'
\coordinate (M) at (0.75,2.25);
% Định nghĩa điểm N trên cạnh CD'
\coordinate (N) at (4.75,2.25);
% Vẽ đường chéo AB' bằng đường đứt nét đỏ (nhưng dùng màu đen theo yêu cầu)
\draw[dashed, thick] (A) -- (B');
% Vẽ đường từ C đến N đến D' (đường xanh trong gốc, nhưng dùng màu đen)
\draw[thick] (C) -- (N) -- (D');
% Vẽ các mũi tên
\draw[->, thick] (0.1,2.7) -- (0.5,2.3);  % Mũi tên từ B' về phía M
\draw[->, thick] (4.17,0.3) -- (4.57,1.4);
% Đánh dấu các điểm
\fill (A) circle (1.2pt);
\fill (B) circle (1.2pt);
\fill (C) circle (1.2pt);
\fill (D) circle (1.2pt);
\fill (A') circle (1.2pt);
\fill (B') circle (1.2pt);
\fill (C') circle (1.2pt);
\fill (D') circle (1.2pt);
\fill (M) circle (1.2pt);
\fill (N) circle (1.2pt);
% Gắn nhãn cho các điểm
\node[below right] at (A) {\(A\)};
\node[below left] at (B) {\(B\)};
\node[below right] at (C) {\(C\)};
\node[below right] at (D) {\(D\)};
\node[above right] at (A') {\(A'\)};
\node[above left] at (B') {\(B'\)};
\node[above left] at (C') {\(C'\)};
\node[above right] at (D') {\(D'\)};
\node[right] at (M) {\(M\)};
\node[right] at (N) {\(N\)};
\end{tikzpicture}


Lời giải:


Dữ kiện:\\
+ Hình hộp chữ nhật \(ABCD.A'B'C'D'\) có:\\
\[
AB = 18,\quad AD = 15,\quad AA' = 15
\]
+ Gán tọa độ:
\[
A = (0, 0, 0),\quad B = (18, 0, 0),\quad D = (0, 15, 0),\quad B' = (18, 0, 15),\quad D' = (0, 15, 15)
\]
+ Con kiến \(M\) bò từ \(B'\) đến \(A\) với tốc độ \(2.4\, \text{cm/s}\)\\
+ Con kiến \(N\) bò từ \(C\) đến \(D'\) với tốc độ \(2.0\, \text{cm/s}\)\\
Bước 1: Phương trình chuyển động\\
Con kiến \(M\):\\
Vector chỉ phương đường đi:
\[
\overrightarrow{B'A} = (0, 0, 0) - (18, 0, 15) = ( -18, 0, -15 )
\]
Chiều dài đoạn \(B'A\):
\[
|\overrightarrow{B'A}| = \sqrt{18^2 + 15^2} = \sqrt{324 + 225} = \sqrt{549}
\]
Trong một giây con kiến tại \(M\) đi được \(\frac{2.4}{\sqrt{549}}\) lần \(\overrightarrow{B'A}\)\\
Véctơ vận tốc của con kiến tại \(M\) là:
\[
\overrightarrow{v}_M = \frac{2.4}{\sqrt{549}} \cdot ( -18, 0, -15 )
\]
Vị trí tại thời điểm \(t\):
\[
M(t) = (18, 0, 15) + t \cdot \frac{2.4}{\sqrt{549}} \cdot ( -18, 0, -15 )
= \left( 18 - \frac{1.8t}{\sqrt{549}},\; 0,\; 15 - \frac{1.5t}{\sqrt{549}} \right)
\]
Con kiến \(N\): Làm tương tự ta có:\\
Vị trí tại thời điểm \(t\):
\[
N(t) = (18, 15, 0) + t \cdot \frac{2.0}{\sqrt{549}} \cdot (-18, 0, 15)
= \left( 18 - \frac{1.5t}{\sqrt{549}},\; 15,\; \frac{1.3t}{\sqrt{549}} \right)
\]
Bước 2: Tính khoảng cách giữa hai con kiến tại thời điểm \(t\)
\[
d(t) = \sqrt{\left(18 - \frac{1.8t}{\sqrt{549}}\right)^2 + 15^2 + \left(15 - \frac{1.5t}{\sqrt{549}} - 2.0t\right)^2}
\]
Bước 3: Tìm khoảng cách nhỏ nhất
\[
d'(t)=0\Leftrightarrow t \approx 5.26 \Rightarrow d_{min}=15.1cm\
\]




Câu 4: 
Với một hình hộp chữ nhật \(ABCD.A'B'C'D'\) có \(AB=20\), \(AD=21\), \(AA'=13\) như hình vẽ. Ở cùng một thời điểm hai con kiến coi như bò chuyển động thẳng đều, con kiến \(M\) bò từ \(B'\) đến điểm \(A\) với tốc độ \(1.8\,\mathrm{cm/s}\) và con kiến \(N\) bò từ \(C\) đến \(D'\) với tốc độ bằng \(3.3\,\mathrm{cm/s}\). Hãy tính khoảng cách nhỏ nhất giữa hai con kiến theo đơn vị centimet (làm tròn kết quả đến hàng phần mười)?

\begin{tikzpicture}[scale=1.0]
% Định nghĩa các đỉnh của hình hộp với phép chiếu đúng
\coordinate (B) at (0,0);
\coordinate (C) at (4,0);
\coordinate (A) at (1.5,1.5);
\coordinate (D) at (5.5,1.5);
\coordinate (B') at (0,3);
\coordinate (C') at (4,3);
\coordinate (A') at (1.5,4.5);
\coordinate (D') at (5.5,4.5);
% Vẽ các cạnh nhìn thấy của hình hộp
\draw (B) -- (C);
\draw (B') -- (C') -- (D') -- (A') -- cycle;
\draw (B) -- (B');
\draw (C) -- (C');
\draw (C) -- (D);
\draw (D) -- (D');
% Vẽ các cạnh ẩn bằng đường đứt nét
\draw[dashed] (B) -- (A);
\draw[dashed] (A) -- (D);
\draw[dashed] (A') -- (A);
% Định nghĩa điểm M trên đường chéo AB'
\coordinate (M) at (0.75,2.25);
% Định nghĩa điểm N trên cạnh CD'
\coordinate (N) at (4.75,2.25);
% Vẽ đường chéo AB' bằng đường đứt nét đỏ (nhưng dùng màu đen theo yêu cầu)
\draw[dashed, thick] (A) -- (B');
% Vẽ đường từ C đến N đến D' (đường xanh trong gốc, nhưng dùng màu đen)
\draw[thick] (C) -- (N) -- (D');
% Vẽ các mũi tên
\draw[->, thick] (0.1,2.7) -- (0.5,2.3);  % Mũi tên từ B' về phía M
\draw[->, thick] (4.17,0.3) -- (4.57,1.4);
% Đánh dấu các điểm
\fill (A) circle (1.2pt);
\fill (B) circle (1.2pt);
\fill (C) circle (1.2pt);
\fill (D) circle (1.2pt);
\fill (A') circle (1.2pt);
\fill (B') circle (1.2pt);
\fill (C') circle (1.2pt);
\fill (D') circle (1.2pt);
\fill (M) circle (1.2pt);
\fill (N) circle (1.2pt);
% Gắn nhãn cho các điểm
\node[below right] at (A) {\(A\)};
\node[below left] at (B) {\(B\)};
\node[below right] at (C) {\(C\)};
\node[below right] at (D) {\(D\)};
\node[above right] at (A') {\(A'\)};
\node[above left] at (B') {\(B'\)};
\node[above left] at (C') {\(C'\)};
\node[above right] at (D') {\(D'\)};
\node[right] at (M) {\(M\)};
\node[right] at (N) {\(N\)};
\end{tikzpicture}


Lời giải:


Dữ kiện:\\
+ Hình hộp chữ nhật \(ABCD.A'B'C'D'\) có:\\
\[
AB = 20,\quad AD = 21,\quad AA' = 13
\]
+ Gán tọa độ:
\[
A = (0, 0, 0),\quad B = (20, 0, 0),\quad D = (0, 21, 0),\quad B' = (20, 0, 13),\quad D' = (0, 21, 13)
\]
+ Con kiến \(M\) bò từ \(B'\) đến \(A\) với tốc độ \(1.8\, \text{cm/s}\)\\
+ Con kiến \(N\) bò từ \(C\) đến \(D'\) với tốc độ \(3.3\, \text{cm/s}\)\\
Bước 1: Phương trình chuyển động\\
Con kiến \(M\):\\
Vector chỉ phương đường đi:
\[
\overrightarrow{B'A} = (0, 0, 0) - (20, 0, 13) = ( -20, 0, -13 )
\]
Chiều dài đoạn \(B'A\):
\[
|\overrightarrow{B'A}| = \sqrt{20^2 + 13^2} = \sqrt{400 + 169} = \sqrt{569}
\]
Trong một giây con kiến tại \(M\) đi được \(\frac{1.8}{\sqrt{569}}\) lần \(\overrightarrow{B'A}\)\\
Véctơ vận tốc của con kiến tại \(M\) là:
\[
\overrightarrow{v}_M = \frac{1.8}{\sqrt{569}} \cdot ( -20, 0, -13 )
\]
Vị trí tại thời điểm \(t\):
\[
M(t) = (20, 0, 13) + t \cdot \frac{1.8}{\sqrt{569}} \cdot ( -20, 0, -13 )
= \left( 20 - \frac{1.5t}{\sqrt{569}},\; 0,\; 13 - \frac{1.0t}{\sqrt{569}} \right)
\]
Con kiến \(N\): Làm tương tự ta có:\\
Vị trí tại thời điểm \(t\):
\[
N(t) = (20, 21, 0) + t \cdot \frac{3.3}{\sqrt{569}} \cdot (-20, 0, 13)
= \left( 20 - \frac{2.8t}{\sqrt{569}},\; 21,\; \frac{1.8t}{\sqrt{569}} \right)
\]
Bước 2: Tính khoảng cách giữa hai con kiến tại thời điểm \(t\)
\[
d(t) = \sqrt{\left(20 - \frac{1.5t}{\sqrt{569}}\right)^2 + 21^2 + \left(13 - \frac{1.0t}{\sqrt{569}} - 3.3t\right)^2}
\]
Bước 3: Tìm khoảng cách nhỏ nhất
\[
d'(t)=0\Leftrightarrow t \approx 3.88 \Rightarrow d_{min}=21.7cm\
\]




Câu 5: 
Với một hình hộp chữ nhật \(ABCD.A'B'C'D'\) có \(AB=24\), \(AD=27\), \(AA'=12\) như hình vẽ. Ở cùng một thời điểm hai con kiến coi như bò chuyển động thẳng đều, con kiến \(M\) bò từ \(B'\) đến điểm \(A\) với tốc độ \(2.0\,\mathrm{cm/s}\) và con kiến \(N\) bò từ \(C\) đến \(D'\) với tốc độ bằng \(2.1\,\mathrm{cm/s}\). Hãy tính khoảng cách nhỏ nhất giữa hai con kiến theo đơn vị centimet (làm tròn kết quả đến hàng phần mười)?

\begin{tikzpicture}[scale=1.0]
% Định nghĩa các đỉnh của hình hộp với phép chiếu đúng
\coordinate (B) at (0,0);
\coordinate (C) at (4,0);
\coordinate (A) at (1.5,1.5);
\coordinate (D) at (5.5,1.5);
\coordinate (B') at (0,3);
\coordinate (C') at (4,3);
\coordinate (A') at (1.5,4.5);
\coordinate (D') at (5.5,4.5);
% Vẽ các cạnh nhìn thấy của hình hộp
\draw (B) -- (C);
\draw (B') -- (C') -- (D') -- (A') -- cycle;
\draw (B) -- (B');
\draw (C) -- (C');
\draw (C) -- (D);
\draw (D) -- (D');
% Vẽ các cạnh ẩn bằng đường đứt nét
\draw[dashed] (B) -- (A);
\draw[dashed] (A) -- (D);
\draw[dashed] (A') -- (A);
% Định nghĩa điểm M trên đường chéo AB'
\coordinate (M) at (0.75,2.25);
% Định nghĩa điểm N trên cạnh CD'
\coordinate (N) at (4.75,2.25);
% Vẽ đường chéo AB' bằng đường đứt nét đỏ (nhưng dùng màu đen theo yêu cầu)
\draw[dashed, thick] (A) -- (B');
% Vẽ đường từ C đến N đến D' (đường xanh trong gốc, nhưng dùng màu đen)
\draw[thick] (C) -- (N) -- (D');
% Vẽ các mũi tên
\draw[->, thick] (0.1,2.7) -- (0.5,2.3);  % Mũi tên từ B' về phía M
\draw[->, thick] (4.17,0.3) -- (4.57,1.4);
% Đánh dấu các điểm
\fill (A) circle (1.2pt);
\fill (B) circle (1.2pt);
\fill (C) circle (1.2pt);
\fill (D) circle (1.2pt);
\fill (A') circle (1.2pt);
\fill (B') circle (1.2pt);
\fill (C') circle (1.2pt);
\fill (D') circle (1.2pt);
\fill (M) circle (1.2pt);
\fill (N) circle (1.2pt);
% Gắn nhãn cho các điểm
\node[below right] at (A) {\(A\)};
\node[below left] at (B) {\(B\)};
\node[below right] at (C) {\(C\)};
\node[below right] at (D) {\(D\)};
\node[above right] at (A') {\(A'\)};
\node[above left] at (B') {\(B'\)};
\node[above left] at (C') {\(C'\)};
\node[above right] at (D') {\(D'\)};
\node[right] at (M) {\(M\)};
\node[right] at (N) {\(N\)};
\end{tikzpicture}


Lời giải:


Dữ kiện:\\
+ Hình hộp chữ nhật \(ABCD.A'B'C'D'\) có:\\
\[
AB = 24,\quad AD = 27,\quad AA' = 12
\]
+ Gán tọa độ:
\[
A = (0, 0, 0),\quad B = (24, 0, 0),\quad D = (0, 27, 0),\quad B' = (24, 0, 12),\quad D' = (0, 27, 12)
\]
+ Con kiến \(M\) bò từ \(B'\) đến \(A\) với tốc độ \(2.0\, \text{cm/s}\)\\
+ Con kiến \(N\) bò từ \(C\) đến \(D'\) với tốc độ \(2.1\, \text{cm/s}\)\\
Bước 1: Phương trình chuyển động\\
Con kiến \(M\):\\
Vector chỉ phương đường đi:
\[
\overrightarrow{B'A} = (0, 0, 0) - (24, 0, 12) = ( -24, 0, -12 )
\]
Chiều dài đoạn \(B'A\):
\[
|\overrightarrow{B'A}| = \sqrt{24^2 + 12^2} = \sqrt{576 + 144} = \sqrt{720}
\]
Trong một giây con kiến tại \(M\) đi được \(\frac{2.0}{\sqrt{720}}\) lần \(\overrightarrow{B'A}\)\\
Véctơ vận tốc của con kiến tại \(M\) là:
\[
\overrightarrow{v}_M = \frac{2.0}{\sqrt{720}} \cdot ( -24, 0, -12 )
\]
Vị trí tại thời điểm \(t\):
\[
M(t) = (24, 0, 12) + t \cdot \frac{2.0}{\sqrt{720}} \cdot ( -24, 0, -12 )
= \left( 24 - \frac{1.8t}{\sqrt{720}},\; 0,\; 12 - \frac{0.9t}{\sqrt{720}} \right)
\]
Con kiến \(N\): Làm tương tự ta có:\\
Vị trí tại thời điểm \(t\):
\[
N(t) = (24, 27, 0) + t \cdot \frac{2.1}{\sqrt{720}} \cdot (-24, 0, 12)
= \left( 24 - \frac{1.9t}{\sqrt{720}},\; 27,\; \frac{0.9t}{\sqrt{720}} \right)
\]
Bước 2: Tính khoảng cách giữa hai con kiến tại thời điểm \(t\)
\[
d(t) = \sqrt{\left(24 - \frac{1.8t}{\sqrt{720}}\right)^2 + 27^2 + \left(12 - \frac{0.9t}{\sqrt{720}} - 2.1t\right)^2}
\]
Bước 3: Tìm khoảng cách nhỏ nhất
\[
d'(t)=0\Leftrightarrow t \approx 6.53 \Rightarrow d_{min}=27.0cm\
\]




Câu 6: 
Với một hình hộp chữ nhật \(ABCD.A'B'C'D'\) có \(AB=10\), \(AD=26\), \(AA'=16\) như hình vẽ. Ở cùng một thời điểm hai con kiến coi như bò chuyển động thẳng đều, con kiến \(M\) bò từ \(B'\) đến điểm \(A\) với tốc độ \(1.7\,\mathrm{cm/s}\) và con kiến \(N\) bò từ \(C\) đến \(D'\) với tốc độ bằng \(2.7\,\mathrm{cm/s}\). Hãy tính khoảng cách nhỏ nhất giữa hai con kiến theo đơn vị centimet (làm tròn kết quả đến hàng phần mười)?

\begin{tikzpicture}[scale=1.0]
% Định nghĩa các đỉnh của hình hộp với phép chiếu đúng
\coordinate (B) at (0,0);
\coordinate (C) at (4,0);
\coordinate (A) at (1.5,1.5);
\coordinate (D) at (5.5,1.5);
\coordinate (B') at (0,3);
\coordinate (C') at (4,3);
\coordinate (A') at (1.5,4.5);
\coordinate (D') at (5.5,4.5);
% Vẽ các cạnh nhìn thấy của hình hộp
\draw (B) -- (C);
\draw (B') -- (C') -- (D') -- (A') -- cycle;
\draw (B) -- (B');
\draw (C) -- (C');
\draw (C) -- (D);
\draw (D) -- (D');
% Vẽ các cạnh ẩn bằng đường đứt nét
\draw[dashed] (B) -- (A);
\draw[dashed] (A) -- (D);
\draw[dashed] (A') -- (A);
% Định nghĩa điểm M trên đường chéo AB'
\coordinate (M) at (0.75,2.25);
% Định nghĩa điểm N trên cạnh CD'
\coordinate (N) at (4.75,2.25);
% Vẽ đường chéo AB' bằng đường đứt nét đỏ (nhưng dùng màu đen theo yêu cầu)
\draw[dashed, thick] (A) -- (B');
% Vẽ đường từ C đến N đến D' (đường xanh trong gốc, nhưng dùng màu đen)
\draw[thick] (C) -- (N) -- (D');
% Vẽ các mũi tên
\draw[->, thick] (0.1,2.7) -- (0.5,2.3);  % Mũi tên từ B' về phía M
\draw[->, thick] (4.17,0.3) -- (4.57,1.4);
% Đánh dấu các điểm
\fill (A) circle (1.2pt);
\fill (B) circle (1.2pt);
\fill (C) circle (1.2pt);
\fill (D) circle (1.2pt);
\fill (A') circle (1.2pt);
\fill (B') circle (1.2pt);
\fill (C') circle (1.2pt);
\fill (D') circle (1.2pt);
\fill (M) circle (1.2pt);
\fill (N) circle (1.2pt);
% Gắn nhãn cho các điểm
\node[below right] at (A) {\(A\)};
\node[below left] at (B) {\(B\)};
\node[below right] at (C) {\(C\)};
\node[below right] at (D) {\(D\)};
\node[above right] at (A') {\(A'\)};
\node[above left] at (B') {\(B'\)};
\node[above left] at (C') {\(C'\)};
\node[above right] at (D') {\(D'\)};
\node[right] at (M) {\(M\)};
\node[right] at (N) {\(N\)};
\end{tikzpicture}


Lời giải:


Dữ kiện:\\
+ Hình hộp chữ nhật \(ABCD.A'B'C'D'\) có:\\
\[
AB = 10,\quad AD = 26,\quad AA' = 16
\]
+ Gán tọa độ:
\[
A = (0, 0, 0),\quad B = (10, 0, 0),\quad D = (0, 26, 0),\quad B' = (10, 0, 16),\quad D' = (0, 26, 16)
\]
+ Con kiến \(M\) bò từ \(B'\) đến \(A\) với tốc độ \(1.7\, \text{cm/s}\)\\
+ Con kiến \(N\) bò từ \(C\) đến \(D'\) với tốc độ \(2.7\, \text{cm/s}\)\\
Bước 1: Phương trình chuyển động\\
Con kiến \(M\):\\
Vector chỉ phương đường đi:
\[
\overrightarrow{B'A} = (0, 0, 0) - (10, 0, 16) = ( -10, 0, -16 )
\]
Chiều dài đoạn \(B'A\):
\[
|\overrightarrow{B'A}| = \sqrt{10^2 + 16^2} = \sqrt{100 + 256} = \sqrt{356}
\]
Trong một giây con kiến tại \(M\) đi được \(\frac{1.7}{\sqrt{356}}\) lần \(\overrightarrow{B'A}\)\\
Véctơ vận tốc của con kiến tại \(M\) là:
\[
\overrightarrow{v}_M = \frac{1.7}{\sqrt{356}} \cdot ( -10, 0, -16 )
\]
Vị trí tại thời điểm \(t\):
\[
M(t) = (10, 0, 16) + t \cdot \frac{1.7}{\sqrt{356}} \cdot ( -10, 0, -16 )
= \left( 10 - \frac{0.9t}{\sqrt{356}},\; 0,\; 16 - \frac{1.4t}{\sqrt{356}} \right)
\]
Con kiến \(N\): Làm tương tự ta có:\\
Vị trí tại thời điểm \(t\):
\[
N(t) = (10, 26, 0) + t \cdot \frac{2.7}{\sqrt{356}} \cdot (-10, 0, 16)
= \left( 10 - \frac{1.4t}{\sqrt{356}},\; 26,\; \frac{2.3t}{\sqrt{356}} \right)
\]
Bước 2: Tính khoảng cách giữa hai con kiến tại thời điểm \(t\)
\[
d(t) = \sqrt{\left(10 - \frac{0.9t}{\sqrt{356}}\right)^2 + 26^2 + \left(16 - \frac{1.4t}{\sqrt{356}} - 2.7t\right)^2}
\]
Bước 3: Tìm khoảng cách nhỏ nhất
\[
d'(t)=0\Leftrightarrow t \approx 4.20 \Rightarrow d_{min}=26.1cm\
\]




Câu 7: 
Với một hình hộp chữ nhật \(ABCD.A'B'C'D'\) có \(AB=25\), \(AD=28\), \(AA'=15\) như hình vẽ. Ở cùng một thời điểm hai con kiến coi như bò chuyển động thẳng đều, con kiến \(M\) bò từ \(B'\) đến điểm \(A\) với tốc độ \(2.3\,\mathrm{cm/s}\) và con kiến \(N\) bò từ \(C\) đến \(D'\) với tốc độ bằng \(2.5\,\mathrm{cm/s}\). Hãy tính khoảng cách nhỏ nhất giữa hai con kiến theo đơn vị centimet (làm tròn kết quả đến hàng phần mười)?

\begin{tikzpicture}[scale=1.0]
% Định nghĩa các đỉnh của hình hộp với phép chiếu đúng
\coordinate (B) at (0,0);
\coordinate (C) at (4,0);
\coordinate (A) at (1.5,1.5);
\coordinate (D) at (5.5,1.5);
\coordinate (B') at (0,3);
\coordinate (C') at (4,3);
\coordinate (A') at (1.5,4.5);
\coordinate (D') at (5.5,4.5);
% Vẽ các cạnh nhìn thấy của hình hộp
\draw (B) -- (C);
\draw (B') -- (C') -- (D') -- (A') -- cycle;
\draw (B) -- (B');
\draw (C) -- (C');
\draw (C) -- (D);
\draw (D) -- (D');
% Vẽ các cạnh ẩn bằng đường đứt nét
\draw[dashed] (B) -- (A);
\draw[dashed] (A) -- (D);
\draw[dashed] (A') -- (A);
% Định nghĩa điểm M trên đường chéo AB'
\coordinate (M) at (0.75,2.25);
% Định nghĩa điểm N trên cạnh CD'
\coordinate (N) at (4.75,2.25);
% Vẽ đường chéo AB' bằng đường đứt nét đỏ (nhưng dùng màu đen theo yêu cầu)
\draw[dashed, thick] (A) -- (B');
% Vẽ đường từ C đến N đến D' (đường xanh trong gốc, nhưng dùng màu đen)
\draw[thick] (C) -- (N) -- (D');
% Vẽ các mũi tên
\draw[->, thick] (0.1,2.7) -- (0.5,2.3);  % Mũi tên từ B' về phía M
\draw[->, thick] (4.17,0.3) -- (4.57,1.4);
% Đánh dấu các điểm
\fill (A) circle (1.2pt);
\fill (B) circle (1.2pt);
\fill (C) circle (1.2pt);
\fill (D) circle (1.2pt);
\fill (A') circle (1.2pt);
\fill (B') circle (1.2pt);
\fill (C') circle (1.2pt);
\fill (D') circle (1.2pt);
\fill (M) circle (1.2pt);
\fill (N) circle (1.2pt);
% Gắn nhãn cho các điểm
\node[below right] at (A) {\(A\)};
\node[below left] at (B) {\(B\)};
\node[below right] at (C) {\(C\)};
\node[below right] at (D) {\(D\)};
\node[above right] at (A') {\(A'\)};
\node[above left] at (B') {\(B'\)};
\node[above left] at (C') {\(C'\)};
\node[above right] at (D') {\(D'\)};
\node[right] at (M) {\(M\)};
\node[right] at (N) {\(N\)};
\end{tikzpicture}


Lời giải:


Dữ kiện:\\
+ Hình hộp chữ nhật \(ABCD.A'B'C'D'\) có:\\
\[
AB = 25,\quad AD = 28,\quad AA' = 15
\]
+ Gán tọa độ:
\[
A = (0, 0, 0),\quad B = (25, 0, 0),\quad D = (0, 28, 0),\quad B' = (25, 0, 15),\quad D' = (0, 28, 15)
\]
+ Con kiến \(M\) bò từ \(B'\) đến \(A\) với tốc độ \(2.3\, \text{cm/s}\)\\
+ Con kiến \(N\) bò từ \(C\) đến \(D'\) với tốc độ \(2.5\, \text{cm/s}\)\\
Bước 1: Phương trình chuyển động\\
Con kiến \(M\):\\
Vector chỉ phương đường đi:
\[
\overrightarrow{B'A} = (0, 0, 0) - (25, 0, 15) = ( -25, 0, -15 )
\]
Chiều dài đoạn \(B'A\):
\[
|\overrightarrow{B'A}| = \sqrt{25^2 + 15^2} = \sqrt{625 + 225} = \sqrt{850}
\]
Trong một giây con kiến tại \(M\) đi được \(\frac{2.3}{\sqrt{850}}\) lần \(\overrightarrow{B'A}\)\\
Véctơ vận tốc của con kiến tại \(M\) là:
\[
\overrightarrow{v}_M = \frac{2.3}{\sqrt{850}} \cdot ( -25, 0, -15 )
\]
Vị trí tại thời điểm \(t\):
\[
M(t) = (25, 0, 15) + t \cdot \frac{2.3}{\sqrt{850}} \cdot ( -25, 0, -15 )
= \left( 25 - \frac{2.0t}{\sqrt{850}},\; 0,\; 15 - \frac{1.2t}{\sqrt{850}} \right)
\]
Con kiến \(N\): Làm tương tự ta có:\\
Vị trí tại thời điểm \(t\):
\[
N(t) = (25, 28, 0) + t \cdot \frac{2.5}{\sqrt{850}} \cdot (-25, 0, 15)
= \left( 25 - \frac{2.1t}{\sqrt{850}},\; 28,\; \frac{1.3t}{\sqrt{850}} \right)
\]
Bước 2: Tính khoảng cách giữa hai con kiến tại thời điểm \(t\)
\[
d(t) = \sqrt{\left(25 - \frac{2.0t}{\sqrt{850}}\right)^2 + 28^2 + \left(15 - \frac{1.2t}{\sqrt{850}} - 2.5t\right)^2}
\]
Bước 3: Tìm khoảng cách nhỏ nhất
\[
d'(t)=0\Leftrightarrow t \approx 6.04 \Rightarrow d_{min}=28.0cm\
\]




Đáp án

1. 29.3 cm

2. 15.0 cm

3. 15.1 cm

4. 21.7 cm

5. 27.0 cm

6. 26.1 cm

7. 28.0 cm

\end{document}
\documentclass[a4paper,12pt]{article}
\usepackage[utf8]{inputenc}
\usepackage{amsmath}
\usepackage{amsfonts}
\usepackage{amssymb}
\usepackage{geometry}
\geometry{a4paper, margin=1in}
\usepackage{fontspec}
\usepackage{tikz}
\usepackage{tkz-tab}
\usepackage{tkz-euclide}
\usetikzlibrary{calc,decorations.pathmorphing,decorations.pathreplacing}
\begin{document}
\title{Câu hỏi đã sửa lỗi nghiệm}
\maketitle

Câu 1: Quãng đường di chuyển của một vật được cho bởi hàm số \(s(t) = -t^{4}+ t^{2}+ 2\) (mét) sau \(t\) giây. 
            Vận tốc tức thời của vật tại thời điểm \(t\) là \(v(t) = s'(t)\). 
            Hỏi vận tốc của vật giảm trong khoảng thời gian nào trong 10 giây đầu?

A. từ giây thứ 0 đến 1

B. từ giây thứ 0 đến 10

C. từ giây thứ 2 đến 9

*D. từ giây thứ 1 đến 10

Lời giải:

Ta có quãng đường: \(s(t) = -t^{4}+ t^{2}+ 2\) (mét).


Với điều kiện Thời gian t $\geq$ 0 (theo ngữ cảnh thực tế).

Vận tốc tức thời: \(v(t) = s'(t) = -4t^{3}+ 2t\) (m/s).

Tính đạo hàm của vận tốc (gia tốc):
\(v'(t) = -12t^{2}+ 2\)

Giải phương trình \(v'(t) = 0\), ta được các điểm tới hạn của vận tốc.



Khi đạo hàm v'(t) < 0, gia tốc âm có nghĩa là vật giảm tốc.

Lập bảng xét dấu cho \(v'(t)\) và kết luận: Vậy quãng đường giảm trong khoảng giây được chỉ ra ở đáp án.



Câu 2: Quãng đường di chuyển của một vật được cho bởi hàm số \(s(t) = t^{4}- 3t^{2}+ 4\) (mét) sau \(t\) giây. 
            Vận tốc tức thời của vật tại thời điểm \(t\) là \(v(t) = s'(t)\). 
            Hỏi vận tốc của vật giảm trong khoảng thời gian nào trong 10 giây đầu?

A. từ giây thứ 1 đến 9

B. từ giây thứ 1 đến 10

*C. từ giây thứ 0 đến 1

D. từ giây thứ 0 đến 3

Lời giải:

Ta có quãng đường: \(s(t) = t^{4}- 3t^{2}+ 4\) (mét).


Với điều kiện Thời gian t $\geq$ 0 (theo ngữ cảnh thực tế).

Vận tốc tức thời: \(v(t) = s'(t) = 4t^{3}- 6t\) (m/s).

Tính đạo hàm của vận tốc (gia tốc):
\(v'(t) = 12t^{2}- 6\)

Giải phương trình \(v'(t) = 0\), ta được các điểm tới hạn của vận tốc.



Khi đạo hàm v'(t) < 0, gia tốc âm có nghĩa là vật giảm tốc.

Lập bảng xét dấu cho \(v'(t)\) và kết luận: Vậy quãng đường giảm trong khoảng giây được chỉ ra ở đáp án.



Câu 3: Quãng đường di chuyển của một vật được cho bởi hàm số \(s(t) = t^{4}- 4t^{2}- 4\) (mét) sau \(t\) giây. 
            Vận tốc tức thời của vật tại thời điểm \(t\) là \(v(t) = s'(t)\). 
            Hỏi vận tốc của vật giảm trong khoảng thời gian nào trong 10 giây đầu?

A. từ giây thứ 0 đến 3

B. từ giây thứ 1 đến 10

*C. từ giây thứ 0 đến 1

D. từ giây thứ 1 đến 5

Lời giải:

Ta có quãng đường: \(s(t) = t^{4}- 4t^{2}- 4\) (mét).


Với điều kiện Thời gian t $\geq$ 0 (theo ngữ cảnh thực tế).

Vận tốc tức thời: \(v(t) = s'(t) = 4t^{3}- 8t\) (m/s).

Tính đạo hàm của vận tốc (gia tốc):
\(v'(t) = 12t^{2}- 8\)

Giải phương trình \(v'(t) = 0\), ta được các điểm tới hạn của vận tốc.



Khi đạo hàm v'(t) < 0, gia tốc âm có nghĩa là vật giảm tốc.

Lập bảng xét dấu cho \(v'(t)\) và kết luận: Vậy quãng đường giảm trong khoảng giây được chỉ ra ở đáp án.



Câu 4: Tốc độ phát triển của một quần thể vi khuẩn được mô tả bởi hàm số \(f(t) = -t^{3}+ 2t^{2}- 1\) (nghìn con/giờ) sau \(t\) giờ. 
            Hỏi tốc độ phát triển vi khuẩn tăng trong khoảng thời gian nào trong 10 giờ đầu?

A. từ giờ thứ 0 đến 3

B. từ giờ thứ 0 đến 5

*C. từ giờ thứ 0 đến 1

D. từ giờ thứ 1 đến 10

Lời giải:

Ta có tốc độ phát triển của quần thể vi khuẩn: \(f(t) = -t^{3}+ 2t^{2}- 1\) (nghìn con/giờ).

Tập xác định: \(D = \mathbb{R}\).
Với điều kiện Thời gian t $\geq$ 0 (theo ngữ cảnh thực tế).

Tính đạo hàm:
\(f'(t) = -3t^{2}+ 4t\)

Giải phương trình \(f'(t) = 0\), ta được các điểm tới hạn:
\(t_1 = -0.0, t_2 = 1.3333333333333333\).



Khi đạo hàm f'(t) > 0, tốc độ phát triển tăng có nghĩa là quần thể phát triển nhanh hơn.

Lập bảng xét dấu cho \(f'(t)\) và kết luận: Vậy tốc độ phát triển vi khuẩn tăng trong khoảng giờ được chỉ ra ở đáp án.



Câu 5: Tốc độ phát triển của một quần thể vi khuẩn được mô tả bởi hàm số \(f(t) = -2t^{3}+ 3t^{2}+ 5\) (nghìn con/giờ) sau \(t\) giờ. 
            Hỏi tốc độ phát triển vi khuẩn giảm trong khoảng thời gian nào trong 10 giờ đầu?

A. từ giờ thứ 0 đến 1

*B. từ giờ thứ 1 đến 10

C. từ giờ thứ 0 đến 10

D. từ giờ thứ 2 đến 9

Lời giải:

Ta có tốc độ phát triển của quần thể vi khuẩn: \(f(t) = -2t^{3}+ 3t^{2}+ 5\) (nghìn con/giờ).

Tập xác định: \(D = \mathbb{R}\).
Với điều kiện Thời gian t $\geq$ 0 (theo ngữ cảnh thực tế).

Tính đạo hàm:
\(f'(t) = -6t^{2}+ 6t\)

Giải phương trình \(f'(t) = 0\), ta được các điểm tới hạn:
\(t_1 = -0.0, t_2 = 1.0\).



Khi đạo hàm f'(t) < 0, tốc độ phát triển giảm có nghĩa là quần thể phát triển chậm lại.

Lập bảng xét dấu cho \(f'(t)\) và kết luận: Vậy tốc độ phát triển vi khuẩn giảm trong khoảng giờ được chỉ ra ở đáp án.

\end{document}
\documentclass[a4paper,12pt]{article}
\usepackage[utf8]{inputenc}
\usepackage{amsmath}
\usepackage{amsfonts}
\usepackage{amssymb}
\usepackage{geometry}
\geometry{a4paper, margin=1in}
\usepackage{fontspec}
\usepackage{tikz}
\usepackage{tkz-tab}
\usepackage{tkz-euclide}
\usetikzlibrary{calc,decorations.pathmorphing,decorations.pathreplacing}
\begin{document}
\title{Câu hỏi}
\maketitle

Câu 1: Tốc độ phát triển của một quần thể vi khuẩn được mô tả bởi hàm số \(f(t) = t^{3}- t^{2}- t\) (nghìn con/giờ) sau \(t\) giờ. 
            Hỏi tốc độ phát triển vi khuẩn tăng trong khoảng thời gian nào trong 10 giờ đầu?

*A. từ giờ thứ 1 đến 10

B. từ giờ thứ 0 đến 1

C. từ giờ thứ 2 đến 9

D. từ giờ thứ 0 đến 10

Lời giải:

Ta có tốc độ phát triển của quần thể vi khuẩn: \(f(t) = t^{3}- t^{2}- t\) (nghìn con/giờ).

Tập xác định: \(D = \mathbb{R}\).
Với điều kiện Thời gian t $\geq$ 0 (theo ngữ cảnh thực tế).

Tính đạo hàm:
\(f'(t) = 3t^{2}- 2t- 1\)

Giải phương trình \(f'(t) = 0\), ta được các điểm tới hạn:
\(t_1 = 0, t_2 = 1\).



Khi đạo hàm f'(t) > 0, tốc độ phát triển tăng có nghĩa là quần thể phát triển nhanh hơn.

Lập bảng xét dấu cho \(f'(t)\) và kết luận: Vậy tốc độ phát triển vi khuẩn tăng trong khoảng giờ được chỉ ra ở đáp án.



Câu 2: Một chiếc xe chuyển động với vận tốc được mô tả bởi hàm số \(f(t) = \frac{-4t^2-5t+3}{5t-3}\) (m/s) sau \(t\) giây. 
            Hỏi vận tốc của xe giảm trong khoảng thời gian nào trong 20 giây đầu?

A. từ giây thứ 5 đến 10

B. từ giây thứ 2 đến 19

*C. từ giây thứ 1 đến 20

D. từ giây thứ 0 đến 20

Lời giải:

Ta có vận tốc của xe: \(f(t) = \frac{-4t^2-5t+3}{5t-3}\) (m/s).

Tập xác định: \(D = \mathbb{R} \setminus \{1\}\).
Với điều kiện Thời gian t $\geq$ 0 (theo ngữ cảnh thực tế).

Tính đạo hàm:
\(f'(t) = \frac{d}{dt}\left(\frac{-4t^2-5t+3}{5t-3}\right) = \frac{-20t^2+24t}{\left(5t-3\right)^2}\)

\(f'(t) = 0 \Leftrightarrow -20t^2+24t = 0\)

\(\Leftrightarrow t_1 = 0, t_2 = 1\).

Điểm gián đoạn: \(t = 1\).

Khi đạo hàm f'(t) < 0, gia tốc âm có nghĩa là xe giảm tốc.

Lập bảng xét dấu cho \(f'(t)\) và kết luận: Vậy vận tốc của xe giảm trong khoảng giây được chỉ ra ở đáp án.



Câu 3: Tốc độ tăng dân số của một thành phố được mô tả bởi hàm số \(f(t) = -t^{3}+ 3t^{2}+ 3t- 2\) (người/năm) sau \(t\) năm. 
            Hỏi tốc độ tăng dân số tăng trong khoảng thời gian nào trong 10 năm đầu?

A. từ năm thứ 2 đến 10

*B. từ năm thứ 0 đến 2

C. từ năm thứ 0 đến 4

D. từ năm thứ 4 đến 10

Lời giải:

Ta có tốc độ tăng dân số của thành phố: \(f(t) = -t^{3}+ 3t^{2}+ 3t- 2\) (người/năm).

Tập xác định: \(D = \mathbb{R}\).
Với điều kiện Thời gian t $\geq$ 0 (theo ngữ cảnh thực tế).

Tính đạo hàm:
\(f'(t) = -3t^{2}+ 6t+ 3\)

Giải phương trình \(f'(t) = 0\), ta được các điểm tới hạn:
\(t_1 = 0, t_2 = 2\).



Khi đạo hàm f'(t) > 0, tốc độ tăng dân số tăng có nghĩa là dân số gia tăng nhanh hơn.

Lập bảng xét dấu cho \(f'(t)\) và kết luận: Vậy tốc độ tăng dân số tăng trong khoảng năm được chỉ ra ở đáp án.



Câu 4: Tốc độ tăng trưởng kinh tế của một quốc gia được mô tả bởi hàm số \(f(t) = \frac{4t^2-5t-1}{3t-5}\) (%/năm) sau \(t\) năm. 
            Hỏi tốc độ tăng trưởng kinh tế tăng trong khoảng thời gian nào trong 10 năm đầu?

A. từ năm thứ 0 đến 3

*B. từ năm thứ 0 đến 1

C. từ năm thứ 1 đến 5

D. từ năm thứ 2 đến 5

Lời giải:

Ta có tốc độ tăng trưởng kinh tế của quốc gia: \(f(t) = \frac{4t^2-5t-1}{3t-5}\) (%/năm).

Tập xác định: \(D = \mathbb{R} \setminus \{2\}\).
Với điều kiện Thời gian t $\geq$ 0 (theo ngữ cảnh thực tế).

Tính đạo hàm:
\(f'(t) = \frac{d}{dt}\left(\frac{4t^2-5t-1}{3t-5}\right) = \frac{12t^2-40t+28}{\left(3t-5\right)^2}\)

\(f'(t) = 0 \Leftrightarrow 12t^2-40t+28 = 0\)

\(\Leftrightarrow t_1 = 1, t_2 = 2\).

Điểm gián đoạn: \(t = 2\).

Khi đạo hàm f'(t) > 0, tốc độ tăng trưởng kinh tế tăng có nghĩa là nền kinh tế phát triển nhanh hơn.

Lập bảng xét dấu cho \(f'(t)\) và kết luận: Vậy tốc độ tăng trưởng kinh tế tăng trong khoảng năm được chỉ ra ở đáp án.



Câu 5: Tốc độ phát triển của một quần thể vi khuẩn được mô tả bởi hàm số \(f(t) = t^{3}- 3t^{2}+ t\) (nghìn con/giờ) sau \(t\) giờ. 
            Hỏi tốc độ phát triển vi khuẩn tăng trong khoảng thời gian nào trong 10 giờ đầu?

A. từ giờ thứ 3 đến 9

*B. từ giờ thứ 2 đến 10

C. từ giờ thứ 0 đến 10

D. từ giờ thứ 0 đến 2

Lời giải:

Ta có tốc độ phát triển của quần thể vi khuẩn: \(f(t) = t^{3}- 3t^{2}+ t\) (nghìn con/giờ).

Tập xác định: \(D = \mathbb{R}\).
Với điều kiện Thời gian t $\geq$ 0 (theo ngữ cảnh thực tế).

Tính đạo hàm:
\(f'(t) = 3t^{2}- 6t+ 1\)

Giải phương trình \(f'(t) = 0\), ta được các điểm tới hạn:
\(t_1 = 0, t_2 = 2\).



Khi đạo hàm f'(t) > 0, tốc độ phát triển tăng có nghĩa là quần thể phát triển nhanh hơn.

Lập bảng xét dấu cho \(f'(t)\) và kết luận: Vậy tốc độ phát triển vi khuẩn tăng trong khoảng giờ được chỉ ra ở đáp án.



Câu 6: Tốc độ phát triển của một quần thể vi khuẩn được mô tả bởi hàm số \(f(t) = t^{3}- 2t^{2}- 5\) (nghìn con/giờ) sau \(t\) giờ. 
            Hỏi tốc độ phát triển vi khuẩn tăng trong khoảng thời gian nào trong 10 giờ đầu?

A. từ giờ thứ 2 đến 9

B. từ giờ thứ 0 đến 1

C. từ giờ thứ 0 đến 10

*D. từ giờ thứ 1 đến 10

Lời giải:

Ta có tốc độ phát triển của quần thể vi khuẩn: \(f(t) = t^{3}- 2t^{2}- 5\) (nghìn con/giờ).

Tập xác định: \(D = \mathbb{R}\).
Với điều kiện Thời gian t $\geq$ 0 (theo ngữ cảnh thực tế).

Tính đạo hàm:
\(f'(t) = 3t^{2}- 4t\)

Giải phương trình \(f'(t) = 0\), ta được các điểm tới hạn:
\(t_1 = 0, t_2 = 1\).



Khi đạo hàm f'(t) > 0, tốc độ phát triển tăng có nghĩa là quần thể phát triển nhanh hơn.

Lập bảng xét dấu cho \(f'(t)\) và kết luận: Vậy tốc độ phát triển vi khuẩn tăng trong khoảng giờ được chỉ ra ở đáp án.



Câu 7: Tốc độ tăng dân số của một thành phố được mô tả bởi hàm số \(f(t) = 2t^{3}- 2t^{2}- 2t+ 4\) (người/năm) sau \(t\) năm. 
            Hỏi tốc độ tăng dân số tăng trong khoảng thời gian nào trong 10 năm đầu?

A. từ năm thứ 2 đến 9

B. từ năm thứ 0 đến 1

*C. từ năm thứ 1 đến 10

D. từ năm thứ 0 đến 10

Lời giải:

Ta có tốc độ tăng dân số của thành phố: \(f(t) = 2t^{3}- 2t^{2}- 2t+ 4\) (người/năm).

Tập xác định: \(D = \mathbb{R}\).
Với điều kiện Thời gian t $\geq$ 0 (theo ngữ cảnh thực tế).

Tính đạo hàm:
\(f'(t) = 6t^{2}- 4t- 2\)

Giải phương trình \(f'(t) = 0\), ta được các điểm tới hạn:
\(t_1 = 0, t_2 = 1\).



Khi đạo hàm f'(t) > 0, tốc độ tăng dân số tăng có nghĩa là dân số gia tăng nhanh hơn.

Lập bảng xét dấu cho \(f'(t)\) và kết luận: Vậy tốc độ tăng dân số tăng trong khoảng năm được chỉ ra ở đáp án.



Câu 8: Tốc độ phát triển của một quần thể vi khuẩn được mô tả bởi hàm số \(f(t) = -2t^{3}+ 3t^{2}+ t+ 4\) (nghìn con/giờ) sau \(t\) giờ. 
            Hỏi tốc độ phát triển vi khuẩn giảm trong khoảng thời gian nào trong 10 giờ đầu?

*A. từ giờ thứ 1 đến 10

B. từ giờ thứ 0 đến 1

C. từ giờ thứ 0 đến 10

D. từ giờ thứ 2 đến 9

Lời giải:

Ta có tốc độ phát triển của quần thể vi khuẩn: \(f(t) = -2t^{3}+ 3t^{2}+ t+ 4\) (nghìn con/giờ).

Tập xác định: \(D = \mathbb{R}\).
Với điều kiện Thời gian t $\geq$ 0 (theo ngữ cảnh thực tế).

Tính đạo hàm:
\(f'(t) = -6t^{2}+ 6t+ 1\)

Giải phương trình \(f'(t) = 0\), ta được các điểm tới hạn:
\(t_1 = 0, t_2 = 1\).



Khi đạo hàm f'(t) < 0, tốc độ phát triển giảm có nghĩa là quần thể phát triển chậm lại.

Lập bảng xét dấu cho \(f'(t)\) và kết luận: Vậy tốc độ phát triển vi khuẩn giảm trong khoảng giờ được chỉ ra ở đáp án.



Câu 9: Tốc độ phát triển của một quần thể vi khuẩn được mô tả bởi hàm số \(f(t) = 2t^{3}- 3t^{2}- 4t- 3\) (nghìn con/giờ) sau \(t\) giờ. 
            Hỏi tốc độ phát triển vi khuẩn tăng trong khoảng thời gian nào trong 10 giờ đầu?

A. từ giờ thứ 2 đến 9

B. từ giờ thứ 0 đến 10

C. từ giờ thứ 0 đến 1

*D. từ giờ thứ 1 đến 10

Lời giải:

Ta có tốc độ phát triển của quần thể vi khuẩn: \(f(t) = 2t^{3}- 3t^{2}- 4t- 3\) (nghìn con/giờ).

Tập xác định: \(D = \mathbb{R}\).
Với điều kiện Thời gian t $\geq$ 0 (theo ngữ cảnh thực tế).

Tính đạo hàm:
\(f'(t) = 6t^{2}- 6t- 4\)

Giải phương trình \(f'(t) = 0\), ta được các điểm tới hạn:
\(t_1 = 0, t_2 = 1\).



Khi đạo hàm f'(t) > 0, tốc độ phát triển tăng có nghĩa là quần thể phát triển nhanh hơn.

Lập bảng xét dấu cho \(f'(t)\) và kết luận: Vậy tốc độ phát triển vi khuẩn tăng trong khoảng giờ được chỉ ra ở đáp án.



Câu 10: Một chiếc xe chuyển động với vận tốc được mô tả bởi hàm số \(f(t) = \frac{-5t^2+2t-2}{2t-2}\) (m/s) sau \(t\) giây. 
            Hỏi vận tốc của xe giảm trong khoảng thời gian nào trong 20 giây đầu?

A. từ giây thứ 0 đến 1

*B. từ giây thứ 2 đến 20

C. từ giây thứ 3 đến 19

D. từ giây thứ 0 đến 20

Lời giải:

Ta có vận tốc của xe: \(f(t) = \frac{-5t^2+2t-2}{2t-2}\) (m/s).

Tập xác định: \(D = \mathbb{R} \setminus \{1\}\).
Với điều kiện Thời gian t $\geq$ 0 (theo ngữ cảnh thực tế).

Tính đạo hàm:
\(f'(t) = \frac{d}{dt}\left(\frac{-5t^2+2t-2}{2t-2}\right) = \frac{-10t^2+20t}{\left(2t-2\right)^2}\)

\(f'(t) = 0 \Leftrightarrow -10t^2+20t = 0\)

\(\Leftrightarrow t_1 = 0, t_2 = 2\).

Điểm gián đoạn: \(t = 1\).

Khi đạo hàm f'(t) < 0, gia tốc âm có nghĩa là xe giảm tốc.

Lập bảng xét dấu cho \(f'(t)\) và kết luận: Vậy vận tốc của xe giảm trong khoảng giây được chỉ ra ở đáp án.

\end{document}
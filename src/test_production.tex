\documentclass[a4paper,12pt]{article}
\usepackage{amsmath}
\usepackage{amsfonts}
\usepackage{amssymb}
\usepackage{geometry}
\geometry{a4paper, margin=1in}
\usepackage{polyglossia}
\setmainlanguage{vietnamese}
\setmainfont{Times New Roman}
\begin{document}
\title{Test ProductionOptimization}
\maketitle

Câu 1: Một cơ sở sản xuất nhựa dân dụng tại miền Trung đang cần xác định thời lượng làm việc tối ưu trong tuần nhằm đảm bảo sản lượng thực tế cao nhất. Cơ sở hiện duy trì 120 tổ lao động, mỗi tổ làm việc 42 giờ/tuần và sản xuất 140 đơn vị sản phẩm mỗi giờ. Khi mở rộng ca làm, xảy ra các biến đổi sau:\\- Cứ mỗi 4 giờ tăng thêm, giảm 1 tổ làm việc.\\- Năng suất mỗi tổ giảm 4 đơn vị mỗi giờ.\\- Số lượng sản phẩm hư hỏng phát sinh trong tuần theo công thức: \( P(x) = \dfrac{55}{4}x^2 + \dfrac{125}{8}x \).\\Bài toán yêu cầu tìm số giờ làm việc \(x\) sao cho số sản phẩm thực tế (tổng sản phẩm sản xuất trừ đi phế phẩm) đạt giá trị lớn nhất. (Đơn vị: giờ)

A. \(44\)

*B. \(65\)

C. \(42\)

D. \(40\)

Lời giải:


Gọi số giờ làm tăng thêm mỗi tuần là \(t\), \(t \in \mathbb{R}\).

Số tổ công nhân bỏ việc là \(\dfrac{t}{4}\) nên số tổ công nhân làm việc là \(120 - \dfrac{t}{4}\) (tổ).

Năng suất của tổ công nhân còn \(140 - \dfrac{4t}{4}\) sản phẩm một giờ.

Số thời gian làm việc một tuần là \(42 + t = x\) (giờ).

\(\Rightarrow\) Số phế phẩm thu được là \(P(42 + t) = \dfrac{110(42 + t)^2 + 125(42 + t)}{8}\)

Để nhà máy hoạt động được thì \(\left\{\begin{array}{l}42 + t > 0 \\ 140 - \dfrac{4t}{4} > 0\end{array}\right. \Rightarrow t \in(-42 ; 480) \\ 120 - \dfrac{t}{4} > 0\)

Số sản phẩm trong một tuần làm được:

\(S = \text{Số tổ x Năng suất x Thời gian} = \left(120 - \dfrac{t}{4}\right)\left(140 - \dfrac{4t}{4}\right)(42 + t)\).

Số sản phẩm thu được là:

\(f(t) = \left(120 - \dfrac{t}{4}\right)\left(140 - \dfrac{4t}{4}\right)(42 + t) - \dfrac{110(42 + t)^2 + 125(42 + t)}{8}\)

\(f'(t) = 1,5t^{2} - 365,5t + 7649,4\)

Ta có \(f'(t) = 0 \Leftrightarrow \left[\begin{array}{l}t \approx 23,12 \\ t \approx 220,54\end{array}\right.\).

Dựa vào bảng biến thiên ta có số lượng sản phẩm thu được lớn nhất thì thời gian làm việc trong một tuần là \(42 + (23,12) \approx 65\) giờ.


\end{document}
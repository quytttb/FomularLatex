\documentclass[a4paper,12pt]{article}
\usepackage{amsmath}
\usepackage{amsfonts}
\usepackage{amssymb}
\usepackage{geometry}
\geometry{a4paper, margin=1in}
\usepackage{polyglossia}
\setmainlanguage{vietnamese}
\setmainfont{Times New Roman}
\usepackage{tikz}
\usepackage{tkz-tab}
\usepackage{tkz-euclide}
\usetikzlibrary{calc,decorations.pathmorphing,decorations.pathreplacing}
\begin{document}
\title{Câu hỏi với tham số từ ảnh}
\maketitle

Câu 1: Trước tình hình đơn hàng xuất khẩu tăng đột biến, xí nghiệp dệt may Z cân nhắc phương án tăng giờ làm trong tuần. Tuy nhiên, mỗi thay đổi kéo theo hệ lụy:\\- Cứ tăng 2 giờ làm/tuần thì có một tổ xin nghỉ do quá tải.\\- Năng suất mỗi tổ giảm 10 áo/giờ.\\- Tổng phế phẩm hàng tuần ước tính bởi: \( P(x) = \dfrac{85}{4}x^2 + 30x \).\\Ban đầu, xí nghiệp có 125 tổ, làm 48 giờ/tuần, mỗi tổ sản xuất 125 áo/giờ. Hãy xác định số giờ làm việc \(x\) mỗi tuần để số lượng sản phẩm thu được (sau khi trừ phế phẩm) đạt giá trị lớn nhất, từ đó đảm bảo hiệu suất tối ưu. (Đơn vị: giờ)

A. \(50\)

B. \(46\)

C. \(48\)

*D. \(31\)

Lời giải:


Gọi số giờ làm tăng thêm mỗi tuần là \(t\), \(t \in \mathbb{R}\).

Số tổ công nhân bỏ việc là \(\dfrac{t}{2}\) nên số tổ công nhân làm việc là \(125 - \dfrac{t}{2}\) (tổ).

Năng suất của tổ công nhân còn \(125 - \dfrac{10t}{2}\) sản phẩm một giờ.

Số thời gian làm việc một tuần là \(48 + t = x\) (giờ).

\(\Rightarrow\) Số phế phẩm thu được là \(P(48 + t) = \dfrac{85(48 + t)^2 + 120(48 + t)}{4}\)

Để nhà máy hoạt động được thì \(\left\{\begin{array}{l}48 + t > 0 \\ 125 - \dfrac{10t}{2} > 0\end{array}\right. \Rightarrow t \in(-48 ; 250) \\ 125 - \dfrac{t}{2} > 0\)

Số sản phẩm trong một tuần làm được:

\(S = \text{Số tổ x Năng suất x Thời gian} = \left(125 - \dfrac{t}{2}\right)\left(125 - \dfrac{10t}{2}\right)(48 + t)\).

Số sản phẩm thu được là:

\(f(t) = \left(125 - \dfrac{t}{2}\right)\left(125 - \dfrac{10t}{2}\right)(48 + t) - \dfrac{85(48 + t)^2 + 120(48 + t)}{4}\)

\(f'(t) = 15t^{2} - 1062,5t - 22445\)

Ta có \(f'(t) = 0 \Leftrightarrow \left[\begin{array}{l}t \approx -17,03 \\ t \approx 87,86\end{array}\right.\).

Dựa vào bảng biến thiên ta có số lượng sản phẩm thu được lớn nhất thì thời gian làm việc trong một tuần là \(48 + (-17,03) \approx 31\) giờ.



\end{document}
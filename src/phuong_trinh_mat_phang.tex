
\documentclass[a4paper,12pt]{article}
\usepackage{amsmath}
\usepackage{mathtools}
\usepackage{amsfonts}
\usepackage{amssymb}
\usepackage{geometry}
\geometry{a4paper, margin=1in}
\usepackage{polyglossia}
\setmainlanguage{vietnamese}
\setmainfont{Times New Roman}
\usepackage{tikz}
\usepackage{tkz-tab}
\usepackage{tkz-euclide}
\usetikzlibrary{calc,decorations.pathmorphing,decorations.pathreplacing}
\begin{document}
\title{Bài tập về Phương trình mặt phẳng}
\maketitle
Câu 1: Trong các mệnh đề dưới đây, mệnh đề nào đúng?

*a) Phương trình mặt phẳng (P) qua điểm A(3;2;-1) và có VTPT \(\vec{n}=(-3;0;3)\) là (P): -3x + 3z + 12 = 0.

b) Phương trình mặt phẳng trung trực (P) của đoạn AB với A(3;-4;0), B(3;-4;3) là (P): y + 3z - 3 = 0.

c) Phương trình mặt phẳng (P) qua A(-3;-1;-4) và (P) || (Q): x + 3y - 2z = 0 là (P): x + 3y - 2z + 1 = 0.

d) Cho A(-3;2;3), B(0;0;1), C(4;-1;1). Phương trình mặt phẳng (P) qua trọng tâm G của \(\triangle ABC\) và vuông góc với BC là (P): 4x - y - 2 = 0.



Câu 2: Trong các mệnh đề dưới đây, mệnh đề nào đúng?

*a) Phương trình mặt phẳng (P) qua A(-2;3;-5) và (P) || (Q): 2x + y - 3z + 1 = 0 là (P): 2x + y - 3z - 14 = 0.

*b) Phương trình mặt phẳng trung trực (P) của đoạn AB với A(-4;5;-5), B(2;-5;3) là (P): 6x - 10y + 8z + 14 = 0.

c) Phương trình mặt phẳng (P) qua điểm A(1;4;-3) và có VTPT \(\vec{n}=(3;-1;-2)\) là (P): -3x - y - 2z - 5 = 0.

d) Cho A(-5;-4;-2), B(0;4;2), C(-1;-3;2). Phương trình mặt phẳng (P) qua trọng tâm G của \(\triangle ABC\) và vuông góc với BC là (P): -4x - 7y - 9 = 0.

\end{document}
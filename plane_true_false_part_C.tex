\documentclass[a4paper,12pt]{article}
\usepackage{amsmath,amssymb}
\usepackage{geometry}
\geometry{a4paper, margin=1in}
\usepackage{polyglossia}
\setmainlanguage{vietnamese}
\setmainfont{Times New Roman}
\begin{document}

\section*{Tương giao mặt phẳng và mặt cầu - Đúng/Sai}

Câu 1: Chọn các mệnh đề đúng.

a) Hình cầu tâm I(0;-2;1) tiếp xúc với mặt phẳng $(P): 4x + 3y - 2z + 1 = 0$ có bán kính $\dfrac{8}{\sqrt{29}}$.

*b) Mặt phẳng tiếp xúc với mặt cầu tâm I(-2;-1;-1) bán kính 5 tại M(3;-1;-1) có phương trình $(P): 5x -15 = 0$.

*c) Tồn tại mặt phẳng $(Q)$ song song với $(P): x + 2y - z -7 = 0$ và tiếp xúc với mặt cầu $x^2+y^2+z^2+4x+6y+-2z+-5=0$ có dạng $(Q): x + 2y - z + 20 = 0$.

d) Với $(P): 2x+y-2z+m=0$ và $(S)$ như trên, luôn tồn tại giao tuyến là đường tròn với mọi $m$.



Câu 2: Chọn các mệnh đề đúng.

*a) Mặt phẳng tiếp xúc với mặt cầu tâm I(-2;-2;-1) bán kính 4 tại M(2;-2;-1) có phương trình $(P): 4x -8 = 0$.

*b) Hình cầu tâm I(0;-2;1) tiếp xúc với mặt phẳng $(P): 4x + 3y - 2z + 1 = 0$ có bán kính $\dfrac{7}{\sqrt{29}}$.

*c) Với $(P): 2x+y-2z+m=0$ và mặt cầu $(S): (x+1)^2+(y-2)^2+(z-3)^2=25$, khi $m$ đủ lớn về trị tuyệt đối thì $(P)$ và $(S)$ không có điểm chung.

*d) Tồn tại mặt phẳng $(Q)$ song song với $(P): x + 2y - 2z + 9 = 0$ và tiếp xúc với mặt cầu $x^2+y^2+z^2+2x+-6y+6z+-10=0$ có dạng $(Q): x + 2y - 2z + 5 = 0$.



\end{document}
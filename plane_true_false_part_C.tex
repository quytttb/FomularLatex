\documentclass[a4paper,12pt]{article}
\usepackage{amsmath,amssymb}
\usepackage{geometry}
\geometry{a4paper, margin=1in}
\usepackage{polyglossia}
\setmainlanguage{vietnamese}
\setmainfont{Times New Roman}
\begin{document}

\section*{Tương giao mặt phẳng và mặt cầu - Đúng/Sai}

Câu 1: Chọn các mệnh đề đúng.

a) Với $(P): 2x+y-2z+m=0$ và $(S)$ như trên, luôn tồn tại giao tuyến là đường tròn với mọi $m$.

*b) Với $(P): 2x+y-2z+m=0$ và mặt cầu $(S): (x+1)^2+(y-2)^2+(z-3)^2=25$, khi $m$ đủ lớn về trị tuyệt đối thì $(P)$ và $(S)$ không có điểm chung.

*c) Với $(P): 2x+y-2z+m=0$ và mặt cầu $(S): (x+1)^2+(y-2)^2+(z-3)^2=25$, khi $m$ đủ lớn về trị tuyệt đối thì $(P)$ và $(S)$ không có điểm chung.

d) Với $(P): 2x+y-2z+m=0$ và $(S)$ như trên, luôn tồn tại giao tuyến là đường tròn với mọi $m$.



Câu 2: Chọn các mệnh đề đúng.

*a) Với $(P): 2x+y-2z+m=0$ và mặt cầu $(S): (x+1)^2+(y-2)^2+(z-3)^2=25$, khi $m$ đủ lớn về trị tuyệt đối thì $(P)$ và $(S)$ không có điểm chung.

*b) Với $(P): 2x+y-2z+m=0$ và mặt cầu $(S): (x+1)^2+(y-2)^2+(z-3)^2=25$, khi $m$ đủ lớn về trị tuyệt đối thì $(P)$ và $(S)$ không có điểm chung.

*c) Với $(P): 2x+y-2z+m=0$ và mặt cầu $(S): (x+1)^2+(y-2)^2+(z-3)^2=25$, khi $m$ đủ lớn về trị tuyệt đối thì $(P)$ và $(S)$ không có điểm chung.

*d) Với $(P): 2x+y-2z+m=0$ và mặt cầu $(S): (x+1)^2+(y-2)^2+(z-3)^2=25$, khi $m$ đủ lớn về trị tuyệt đối thì $(P)$ và $(S)$ không có điểm chung.



Câu 3: Chọn các mệnh đề đúng.

a) Với $(P): 2x+y-2z+m=0$ và $(S)$ như trên, luôn tồn tại giao tuyến là đường tròn với mọi $m$.

b) Với $(P): 2x+y-2z+m=0$ và $(S)$ như trên, luôn tồn tại giao tuyến là đường tròn với mọi $m$.

*c) Với $(P): 2x+y-2z+m=0$ và mặt cầu $(S): (x+1)^2+(y-2)^2+(z-3)^2=25$, khi $m$ đủ lớn về trị tuyệt đối thì $(P)$ và $(S)$ không có điểm chung.

*d) Với $(P): 2x+y-2z+m=0$ và mặt cầu $(S): (x+1)^2+(y-2)^2+(z-3)^2=25$, khi $m$ đủ lớn về trị tuyệt đối thì $(P)$ và $(S)$ không có điểm chung.



\end{document}
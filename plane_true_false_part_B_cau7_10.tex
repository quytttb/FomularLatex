\documentclass[a4paper,12pt]{article}
\usepackage{amsmath,amssymb}
\usepackage{geometry}
\geometry{a4paper, margin=1in}
\usepackage{polyglossia}
\setmainlanguage{vietnamese}
\setmainfont{Times New Roman}
\begin{document}

\section*{Các bài toán về viết phương trình mặt phẳng - Đúng/Sai (Câu 7-10)}

Câu 7: Chọn các mệnh đề đúng.

a) Mặt phẳng song song với (Q): x + 2y + 2z = 0 và cách điểm M(1;0;1) khoảng 3 có phương trình \(x + 2y + 2z - 14 = 0\).

b) Với M(2;3;1), mặt phẳng (P) để \(T\) nhỏ nhất có dạng \(ax+by+cz+d=0\) với \(a=b=c=0\).

*c) Mặt phẳng đi qua M(3;4;4) cắt ba trục tọa độ sao cho thể tích tứ diện OABC nhỏ nhất có phương trình \(16x + 12y + 12z - 144 = 0\).

*d) Mặt phẳng song song với (Q): x + y + z = 0 và cách (Q) khoảng 3 có phương trình \(x + y + z - 3\sqrt{3} = 0\).



Câu 8: Chọn các mệnh đề đúng.

*a) Mặt phẳng đi qua điểm G(4;2;2) và cắt các trục tọa độ tại các điểm A, B, C sao cho G là trọng tâm của tam giác ABC có phương trình \(4x + 8y + 8z - 48 = 0\).

b) Cho hai điểm C(0;0;3) và M(3;5;2). Mặt phẳng qua C, M đồng thời chắn trên các nửa trục dương Ox, Oy các đoạn thẳng bằng nhau có phương trình \(3x + 3y + 24z - 69 = 0\).

c) Mặt phẳng đi qua M(6;4;3) và cắt các trục tọa độ tại A, B, C sao cho M là trực tâm tam giác ABC có phương trình \(6x + 4y + 3z - 56 = 0\).

d) Mặt phẳng qua M(2;-3;3) và qua giao tuyến của (α): x - z - 1 = 0 và (β): 2x - y + 2z - 1 = 0 có phương trình \(6x - y - 4z - 4 = 0\).



Câu 9: Chọn các mệnh đề đúng.

*a) Mặt phẳng đi qua điểm M(1;1;3) và cắt trục tọa độ Ox, Oy, Oz tại A, B, C sao cho M là trực tâm tam giác ABC có phương trình \(x + y + 3z - 11 = 0\).

b) Mặt phẳng đi qua M(2;3;5) và cắt các trục tọa độ Ox, Oy, Oz lần lượt tại A, B, C sao cho M là trực tâm của tam giác ABC có phương trình \(2x + 3y + 5z - 34 = 0\).

c) Mặt phẳng vuông góc với (α): x + 2y - z + 2 = 0 và (β): 2x - y + z + 2 = 0, cách điểm O(2;1;0) một khoảng bằng 4 có dạng \(1x + -3y + -5z + 19.7 = 0\).

*d) Mặt phẳng đi qua M(3;1;1) cắt ba trục tọa độ sao cho thể tích tứ diện OABC nhỏ nhất có phương trình \(x + 3y + 3z - 9 = 0\).



Câu 10: Chọn các mệnh đề đúng.

a) Mặt phẳng qua M(0;3;3) và cắt các tia Ox, Oy, Oz tại A, B, C sao cho 4OA = 2OB = OC có phương trình \(4x + 2y + z - 12 = 0\).

b) Mặt phẳng đi qua M(1;3;3) cắt ba trục tọa độ sao cho tứ diện OABC có thể tích nhỏ nhất. Thể tích nhỏ nhất đó bằng \(40.5\).

*c) Có hai mặt phẳng song song với (Q): x + y + z - 2 = 0 và cách điểm M(2;1;0) khoảng 2. Một trong hai mặt phẳng đó có phương trình \(x + y + z - 3 - 2\sqrt{3} = 0\).

d) Với M(2;3;3), mặt phẳng (P) để \(T\) nhỏ nhất có dạng \(ax+by+cz+d=0\) với \(a=b=c=0\).



\end{document}
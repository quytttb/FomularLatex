\documentclass[a4paper,12pt]{article}
\usepackage{amsmath,amssymb}
\usepackage{geometry}
\geometry{a4paper, margin=1in}
\usepackage{polyglossia}
\setmainlanguage{vietnamese}
\setmainfont{Times New Roman}
\begin{document}

\section*{Các bài toán về viết phương trình mặt phẳng - Đúng/Sai (Câu 7-10)}

Câu 7: Chọn các mệnh đề đúng.

*a) Mặt phẳng song song với (Q): x + z - 1 = 0 và cách (Q) khoảng 3 có phương trình \(x + z - 1 + 3\sqrt{2} = 0\).

*b) Mặt phẳng đi qua M(4;1;4) và cắt các trục tọa độ tại A, B, C sao cho M là trực tâm tam giác ABC có phương trình \(4x + y + 4z - 33 = 0\).

*c) Mặt phẳng đi qua M(4;4;2) và cắt các trục tọa độ Ox, Oy, Oz lần lượt tại A, B, C sao cho M là trực tâm của tam giác ABC có phương trình \(4x + 4y + 2z - 36 = 0\).

d) Mặt phẳng qua M(1;-1;1) và qua giao tuyến của (α): 2x - y = 0 và (β): 2x - y + 2z - 2 = 0 có phương trình \(2x - 2z - 1 = 0\).



Câu 8: Chọn các mệnh đề đúng.

a) Mặt phẳng song song với (Q): 2x - y + 2z - 1 = 0 và cách điểm M(0;1;2) khoảng 3 có phương trình \(2x - y + 2z + 5 = 0\).

*b) Mặt phẳng vuông góc với (α): -x + 3y + z - 2 = 0 và (β): -2x + y + 2z - 3 = 0, cách điểm O(1;0;2) một khoảng bằng 3 có thể có dạng \(5x + 0y + 5z + 6.2 = 0\) hoặc \(5x + 0y + 5z + -36.2 = 0\).

*c) Mặt phẳng qua M(3;2;-1) và cắt các tia Ox, Oy, Oz tại A, B, C sao cho 4OA = 2OB = OC có phương trình \(4x + 2y + z - 15 = 0\).

*d) Mặt phẳng đi qua M(3;1;2) cắt ba trục tọa độ sao cho thể tích tứ diện OABC nhỏ nhất có phương trình \(2x + 6y + 3z - 18 = 0\).



Câu 9: Chọn các mệnh đề đúng.

*a) Mặt phẳng đi qua M(3;3;2) cắt ba trục tọa độ sao cho tứ diện OABC có thể tích nhỏ nhất. Thể tích nhỏ nhất đó bằng \(81.0\).

b) Cho hai điểm C(0;0;2) và M(1;2;0). Mặt phẳng qua C, M đồng thời chắn trên các nửa trục dương Ox, Oy các đoạn thẳng bằng nhau có phương trình \(x + y + z - 3 = 0\).

c) Mặt phẳng đi qua M(4;1;2) cắt ba trục tọa độ sao cho thể tích tứ diện OABC nhỏ nhất có phương trình \(8x + 8y + 4z - 24 = 0\).

*d) Mặt phẳng đi qua điểm G(1;-1;1) và cắt các trục tọa độ tại các điểm A, B, C sao cho G là trọng tâm của tam giác ABC có phương trình \(-x + y - z + 3 = 0\).



Câu 10: Chọn các mệnh đề đúng.

*a) Mặt phẳng đi qua điểm M(1;5;1) và cắt trục tọa độ Ox, Oy, Oz tại A, B, C sao cho M là trực tâm tam giác ABC có phương trình \(x + 5y + z - 27 = 0\).

b) Với M(3;2;1), mặt phẳng (P) để \(T\) nhỏ nhất có dạng \(ax+by+cz+d=0\) với \(a=b=c=0\).

*c) Mặt phẳng song song với (Q): x + y + z = 0 và cách (Q) khoảng 3 có phương trình \(x + y + z + 3\sqrt{3} = 0\).

*d) Mặt phẳng đi qua M(6;4;2) và cắt các trục tọa độ tại A, B, C sao cho M là trực tâm tam giác ABC có phương trình \(6x + 4y + 2z - 56 = 0\).



\end{document}
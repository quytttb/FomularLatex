\documentclass[a4paper,12pt]{article}
\usepackage{amsmath,amssymb}
\usepackage{geometry}
\geometry{a4paper, margin=1in}
\usepackage{polyglossia}
\setmainlanguage{vietnamese}
\setmainfont{Times New Roman}
\begin{document}

\section*{Các bài toán về viết phương trình mặt phẳng - Đúng/Sai (Câu 7-10)}

Câu 7: Chọn các mệnh đề đúng.

*a) Mặt phẳng đi qua M(5;4;4) và cắt các trục tọa độ tại A, B, C sao cho M là trực tâm tam giác ABC có phương trình \(5x + 4y + 4z - 57 = 0\).

*b) Mặt phẳng đi qua điểm G(4;3;-5) và cắt các trục tọa độ tại các điểm A, B, C sao cho G là trọng tâm của tam giác ABC có phương trình \(-15x - 20y + 12z + 180 = 0\).

*c) Mặt phẳng qua M(1;3;1) và cắt các tia Ox, Oy, Oz tại A, B, C sao cho 4OA = 2OB = OC có phương trình \(4x + 2y + z - 11 = 0\).

d) Mặt phẳng đi qua M(1;2;2) cắt ba trục tọa độ sao cho tứ diện OABC có thể tích nhỏ nhất. Thể tích nhỏ nhất đó bằng \(17.5\).



Câu 8: Chọn các mệnh đề đúng.

*a) Mặt phẳng song song với (Q): 2x + 1 = 0 và cách (Q) khoảng 2 có phương trình \(2x - 3 = 0\).

*b) Cho hai điểm C(0;0;4) và M(5;3;2). Mặt phẳng qua C, M đồng thời chắn trên các nửa trục dương Ox, Oy các đoạn thẳng bằng nhau có phương trình \(4x + 4y + 16z - 64 = 0\).

*c) Mặt phẳng đi qua M(5;5;3) và cắt các trục tọa độ Ox, Oy, Oz lần lượt tại A, B, C sao cho M là trực tâm của tam giác ABC có phương trình \(5x + 5y + 3z - 59 = 0\).

d) Mặt phẳng đi qua M(3;1;3) cắt ba trục tọa độ sao cho thể tích tứ diện OABC nhỏ nhất có phương trình \(3x + 9y + 3z - 25 = 0\).



Câu 9: Chọn các mệnh đề đúng.

*a) Mặt phẳng qua M(2;1;1) và qua giao tuyến của (α): 2x - y - 1 = 0 và (β): 2x - y + 2z = 0 có phương trình \(6x - 3y - 4z - 5 = 0\).

b) Mặt phẳng vuông góc với (α): x + y - 2z = 0 và (β): 3x + y - z - 1 = 0, cách điểm O(2;1;0) một khoảng bằng 4 có dạng \(1x + -5y + -2z + 29.9 = 0\).

c) Với M(4;2;1), mặt phẳng (P) để \(T\) nhỏ nhất có dạng \(ax+by+cz+d=0\) với \(a=b=c=0\).

d) Mặt phẳng song song với (Q): x + y + z - 2 = 0 và cách điểm M(2;1;0) khoảng 2 có phương trình \(x + y + z - 3 + 3\sqrt{3} = 0\).



Câu 10: Chọn các mệnh đề đúng.

*a) Mặt phẳng đi qua điểm M(4;2;2) và cắt trục tọa độ Ox, Oy, Oz tại A, B, C sao cho M là trực tâm tam giác ABC có phương trình \(4x + 2y + 2z - 24 = 0\).

b) Mặt phẳng đi qua M(1;3;2) cắt ba trục tọa độ sao cho thể tích tứ diện OABC nhỏ nhất có phương trình \(24x + 2y + 3z - 18 = 0\).

c) Mặt phẳng đi qua M(3;1;3) và cắt các trục tọa độ tại A, B, C sao cho M là trực tâm tam giác ABC có phương trình \(3x + y + 3z - 24 = 0\).

d) Mặt phẳng đi qua điểm G(2;1;-2) và cắt các trục tọa độ tại các điểm A, B, C sao cho G là trọng tâm của tam giác ABC có phương trình \(-2x + 4y + 2z + 15 = 0\).



\end{document}
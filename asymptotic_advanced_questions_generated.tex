\documentclass{article}
\usepackage{fontspec} % For XeLaTeX font support
\usepackage{amsmath} % For advanced math environments
\usepackage{amsfonts}
\usepackage{amssymb}
\usepackage{geometry} % For page layout
\geometry{a4paper, margin=1in} % Set page margins
\usepackage{polyglossia}
\setmainlanguage{vietnamese}
\setmainfont{Times New Roman}

\begin{document}

\section*{Câu 1:}

a) \(y = \frac{(2m + 4)x - 4}{nx + 4}\) có Tiệm cận đứng đi qua điểm \((-2; 0)\) và tiệm cận ngang đi qua điểm \((0; 2)\) thì giá trị biểu thức là:

A. \(-5\)

B. \(-7\)

C. \(-4\)

*D. \(-6\)


b) \(y = \frac{(-2m + 1)x - 4}{2x + 4}\) có đường tiệm cận tạo hình chữ nhật diện tích (3\. Tổng bình phương các giá trị m là:

*A. \(\frac{37}{2}\)

B. \(20\)

C. \(10\)

D. \(15\)


c) Số giá trị nguyên m trong [-10; 10] để \(y = \frac{x - 1}{x^2 + (-m + 1)x + 1}\) có 3 tiệm cận:

*A. 16

B. 17

C. 18

D. 15


d) \(\frac{x^2 + x + 2}{x + 4}\) có phương trình tiệm cận xiên:

A. \(y = -1 + \frac{6}{x + 4}\)

*B. \(y = x + 3 - \frac{10}{x + 4}\)

C. \(y = x + 1 - \frac{2}{x + 4}\)

D. \(y = x + 3 - \frac{12}{x + 4}\)


\textbf{Lời giải:}

\textbf{Lời giải cho mệnh đề a):}

Hàm số \(y = \frac{(2m + 4)x - 4}{nx + 4}\) có Tiệm cận đứng đi qua điểm \((-2; 0)\) và tiệm cận ngang đi qua điểm \((0; 2)\)

\(\Leftrightarrow\) Hàm số \(y = \frac{(2m + 4)x - 4}{nx + 4}\) có tiệm cận đứng: \(x = -2\); tiệm cận ngang: \(y = 2\)

\begin{itemize}
\item Tiệm cận đứng:
\(x = \frac{-4}{n} = -2 \Rightarrow n = \frac{-4}{-2} = 2\)
\item Tiệm cận ngang:
\(y = \frac{2m + 4}{n} = 2\)
\end{itemize}

\(\Leftrightarrow 2m + 4 = 2 \cdot 2 = 4\)

\(\Rightarrow 2m = 4 - 4 = 0\)

\(\Rightarrow m = 0\)

Vậy \(4m - 2n = 4 \cdot 0 - 2 \cdot 2 = -4\)

\textbf{Lời giải cho mệnh đề b):}

Hàm số \(y = \frac{(-2m + 1)x - 4}{2x + 4}\) có:

\begin{itemize}
\item Tiệm cận ngang: \(y = \frac{-2m + 1}{2}\)
\item Tiệm cận đứng: \(x = \frac{-4}{2} = -2\)
\end{itemize}

Diện tích hình chữ nhật tạo bởi hai đường tiệm cận và hai trục tọa độ:

\(S = \left| \frac{-2m + 1}{2} \cdot (-2) \right| = \left| \frac{-2m + 1}{2} \cdot (-2) \right| = \frac{|-2m + 1| \cdot |4|}{4}\)

Theo đề bài: \(S = 3\)

\(\Rightarrow \frac{|-2m + 1| \cdot 4}{4} = 3\)

\(\Leftrightarrow |-2m + 1| = \frac{3 \cdot 4}{4} = 6\)

\(\Leftrightarrow -2m + 1 = \pm 6\)

Trường hợp 1: \(-2m + 1 = 6 \Rightarrow m = \frac{-5}{2}\)

Trường hợp 2: \(-2m + 1 = -6 \Rightarrow m = \frac{7}{2}\)

Tổng bình phương các giá trị: \((\frac{-5}{2})^2 + (\frac{7}{2})^2 = \frac{37}{2}\)

\textbf{Lời giải cho mệnh đề c):}

\textbf{Để có 3 tiệm cận:}

\textbf{- Dễ thấy hàm số chỉ có 1 tiệm cận ngang} \(y = 0\) (vì bậc tử < bậc mẫu)

\textbf{- Để hàm số có 3 tiệm cận} \(\Rightarrow\) hàm số có 2 tiệm cận đứng

\(\Leftrightarrow \begin{cases}
f(1) \neq 0 \\
\Delta > 0
\end{cases}\)

\(\Leftrightarrow \begin{cases}
m \neq 3 \\
m^2 - 2m - 3 > 0
\end{cases}\)

\(\Leftrightarrow m \in (-\infty, 1 - 2\sqrt{4}) \cup (1 + 2\sqrt{4}, +\infty)\)

\textbf{Kết quả:} Có \(16\) giá trị nguyên thỏa mãn

\textbf{Lời giải cho mệnh đề d):}

\textbf{Giải:}

Ta có: \(y = \frac{x^2 + x + 2}{x + 4} = x + 3 - \frac{10}{x + 4}\)

\(\Rightarrow \displaystyle\lim_{x \to +\infty} \left(\left(\frac{x^2 + x + 2}{x + 4}\right) - \left(x + 3 - \frac{10}{x + 4}\right)\right) = \displaystyle\lim_{x \to +\infty} \frac{-10}{x + 4} = \displaystyle\lim_{x \to +\infty} \frac{\frac{-10}{x}}{-1 + \frac{4}{x}} = 0\)

\(\Rightarrow\) Tiệm cận xiên: \(y = x + 3 - \frac{10}{x + 4}\)



\newpage

\section*{Câu 2:}

a) \(y = \frac{-mx - 1}{nx + 1}\) có Tiệm cận đứng: \(x = 3\); tiệm cận ngang: \(y = 1\) thì giá trị biểu thức là:

*A. \(\frac{4}{3}\)

B. \(3\)

C. \(1\)

D. \(2\)


b) \(y = \frac{(2m - 2)x + 3}{-2x - 4}\) có đường tiệm cận tạo hình chữ nhật diện tích (5\. Tổng bình phương các giá trị m là:

A. \(45\)

B. \(48\)

C. \(49\)

*D. \(47\)


c) Số giá trị nguyên m trong [-10; 10] để \(y = \frac{x - 1}{x^2 + (-2m + 2)x - 2}\) có 2 tiệm cận đứng:

A. 20

B. 22

C. 23

*D. 21


d) \(\frac{2x^2 - 2x - 4}{x + 2}\) có phương trình tiệm cận xiên:

*A. \(y = 2x + 6 - \frac{16}{x + 2}\)

B. \(y = 3x + 8 - \frac{20}{x + 2}\)

C. \(y = 2x + 5 - \frac{14}{x + 2}\)

D. \(y = 2x + 6 - \frac{15}{x + 2}\)


\textbf{Lời giải:}

\textbf{Lời giải cho mệnh đề a):}

Hàm số \(y = \frac{-mx - 1}{nx + 1}\) có:

\begin{itemize}
\item Tiệm cận đứng:
\(x = \frac{-1}{n} = 3 \Rightarrow n = \frac{-1}{3}\)
\item Tiệm cận ngang:
\(y = \frac{-m}{n} = 1\)
\end{itemize}

\(\Leftrightarrow -m = (\frac{-1}{3}) = \frac{-1}{3}\)

\(\Rightarrow -m = \frac{-1}{3}\)

\(\Rightarrow m = \frac{1}{3}\)

Vậy \(3m + 2n = 3 \cdot \frac{1}{3} + 2 \cdot (\frac{-1}{3}) = \frac{1}{3}\)

\textbf{Lời giải cho mệnh đề b):}

Hàm số \(y = \frac{(2m - 2)x + 3}{-2x - 4}\) có:

\begin{itemize}
\item Tiệm cận ngang: \(y = \frac{2m - 2}{-2}\)
\item Tiệm cận đứng: \(x = \frac{4}{-2} = -2\)
\end{itemize}

Diện tích hình chữ nhật tạo bởi hai đường tiệm cận và hai trục tọa độ:

\(S = \left| \frac{2m - 2}{-2} \cdot (-2) \right| = \left| \frac{2m - 2}{-2} \cdot (-2) \right| = \frac{|2m - 2| \cdot |-4|}{4}\)

Theo đề bài: \(S = 5\)

\(\Rightarrow \frac{|2m - 2| \cdot 4}{4} = 5\)

\(\Leftrightarrow |2m - 2| = \frac{5 \cdot 4}{4} = 10\)

\(\Leftrightarrow 2m - 2 = \pm 10\)

Trường hợp 1: \(2m - 2 = 10 \Rightarrow m = 6\)

Trường hợp 2: \(2m - 2 = -10 \Rightarrow m = -4\)

Tổng bình phương các giá trị: \((6)^2 + (-4)^2 = 52\)

\textbf{Lời giải cho mệnh đề c):}

\textbf{Để có 2 tiệm cận đứng:}

Điều kiện: \(f(1) \neq 0\) và mẫu số có 2 nghiệm phân biệt

\(\Leftrightarrow \begin{cases}
f(1) \neq 0 \\
\Delta > 0
\end{cases}\)

\(\Leftrightarrow \begin{cases}
m \neq \frac{1}{2} \\
4m^2 - 8m + 12 > 0
\end{cases}\)

\(\Leftrightarrow m \in \mathbb{R} \setminus \{\frac{1}{2}\}\)

\textbf{Kết quả:} Có \(21\) giá trị nguyên thỏa mãn

\textbf{Lời giải cho mệnh đề d):}

\textbf{Giải:}

Ta có: \(y = \frac{2x^2 - 2x - 4}{x + 2} = 2x + 6 - \frac{16}{x + 2}\)

\(\Rightarrow \displaystyle\lim_{x \to +\infty} \left(\left(\frac{2x^2 - 2x - 4}{x + 2}\right) - \left(2x + 6 - \frac{16}{x + 2}\right)\right) = \displaystyle\lim_{x \to +\infty} \frac{-16}{x + 2} = \displaystyle\lim_{x \to +\infty} \frac{\frac{-16}{x}}{-1 + \frac{2}{x}} = 0\)

\(\Rightarrow\) Tiệm cận xiên: \(y = 2x + 6 - \frac{16}{x + 2}\)



\newpage

\section*{Câu 3:}

a) \(y = \frac{(2m + 1)x - 5}{nx + 4}\) có Tiệm cận đứng và ngang giao nhau tại điểm \((-1; 2)\) thì giá trị biểu thức là:

A. \(3\)

B. \(6\)

*C. \(4\)

D. \(5\)


b) \(y = \frac{(-2m + 4)x + 1}{x - 2}\) có đường tiệm cận tạo hình chữ nhật diện tích (6\. Tổng bình phương các giá trị m là:

A. \(19\)

*B. \(21\)

C. \(23\)

D. \(22\)


c) Số giá trị nguyên m trong [-10; 10] để \(y = \frac{x - 3}{x^2 + (-m - 3)x + 6}\) có 2 tiệm cận đứng:

A. 10

*B. 11

C. 13

D. 12


d) \(\frac{2x^2 - 2x - 3}{x + 3}\) có phương trình tiệm cận xiên:

A. \(y = -2x + 3 - \frac{12}{x + 3}\)

B. \(y = -2x + 4 - \frac{13}{x + 3}\)

*C. \(y = -2x + 4 - \frac{15}{x + 3}\)

D. \(y = -2x + 4 - \frac{15}{x + 3}\)


\textbf{Lời giải:}

\textbf{Lời giải cho mệnh đề a):}

Hàm số \(y = \frac{(2m + 1)x - 5}{nx + 4}\) có Tiệm cận đứng và ngang giao nhau tại điểm \((-1; 2)\)

\(\Leftrightarrow\) Hàm số \(y = \frac{(2m + 1)x - 5}{nx + 4}\) có tiệm cận đứng: \(x = -1\); tiệm cận ngang: \(y = 2\)

\begin{itemize}
\item Tiệm cận đứng:
\(x = \frac{-4}{n} = -1 \Rightarrow n = \frac{-4}{-1} = 4\)
\item Tiệm cận ngang:
\(y = \frac{2m + 1}{n} = 2\)
\end{itemize}

\(\Leftrightarrow 2m + 1 = 2 \cdot 4 = 8\)

\(\Rightarrow 2m = 8 - 1 = 7\)

\(\Rightarrow m = \frac{7}{2}\)

Vậy \(4m - 3n = 4 \cdot \frac{7}{2} - 3 \cdot 4 = 2\)

\textbf{Lời giải cho mệnh đề b):}

Hàm số \(y = \frac{(-2m + 4)x + 1}{x - 2}\) có:

\begin{itemize}
\item Tiệm cận ngang: \(y = \frac{-2m + 4}{1}\)
\item Tiệm cận đứng: \(x = \frac{2}{1} = 2\)
\end{itemize}

Diện tích hình chữ nhật tạo bởi hai đường tiệm cận và hai trục tọa độ:

\(S = \left| \frac{-2m + 4}{1} \cdot 2 \right| = \left| \frac{-2m + 4}{1} \cdot 2 \right| = \frac{|-2m + 4| \cdot |-2|}{1}\)

Theo đề bài: \(S = 6\)

\(\Rightarrow \frac{|-2m + 4| \cdot 2}{1} = 6\)

\(\Leftrightarrow |-2m + 4| = \frac{6}{2} = 6\)

\(\Leftrightarrow -2m + 4 = \pm 6\)

Trường hợp 1: \(-2m + 4 = 6 \Rightarrow m = -1\)

Trường hợp 2: \(-2m + 4 = -6 \Rightarrow m = 5\)

Tổng bình phương các giá trị: \((-1)^2 + (5)^2 = 26\)

\textbf{Lời giải cho mệnh đề c):}

\textbf{Để có 2 tiệm cận đứng:}

Điều kiện: \(f(3) \neq 0\) và mẫu số có 2 nghiệm phân biệt

\(\Leftrightarrow \begin{cases}
f(3) \neq 0 \\
\Delta > 0
\end{cases}\)

\(\Leftrightarrow \begin{cases}
m \neq 2 \\
m^2 + 6m - 15 > 0
\end{cases}\)

\(\Leftrightarrow m \in (-\infty, -3 - 4\sqrt{6}) \cup (-3 + 4\sqrt{6}, +\infty) \setminus \{2\}\)

\textbf{Kết quả:} Có \(11\) giá trị nguyên thỏa mãn

\textbf{Lời giải cho mệnh đề d):}

\textbf{Giải:}

Ta có: \(y = \frac{2x^2 - 2x - 3}{x + 3} = -2x + 4 - \frac{15}{x + 3}\)

\(\Rightarrow \displaystyle\lim_{x \to +\infty} \left(\left(\frac{2x^2 - 2x - 3}{x + 3}\right) - \left(-2x + 4 - \frac{15}{x + 3}\right)\right) = \displaystyle\lim_{x \to +\infty} \frac{-15}{x + 3} = \displaystyle\lim_{x \to +\infty} \frac{\frac{-15}{x}}{1 + \frac{3}{x}} = 0\)

\(\Rightarrow\) Tiệm cận xiên: \(y = -2x + 4 - \frac{15}{x + 3}\)



\newpage

\section*{Câu 4:}

a) \(y = \frac{(3m - 5)x}{nx - 4}\) có Tiệm cận đứng và ngang giao nhau tại điểm \((1; 3)\) thì giá trị biểu thức là:

A. \(1\)

B. \(3\)

*C. \(\frac{-5}{3}\)

D. \(2\)


b) \(y = \frac{(2m + 4)x + 1}{2x + 2}\) có đường tiệm cận tạo hình chữ nhật diện tích (3\. Tổng bình phương các giá trị m là:

*A. \(80\)

B. \(82\)

C. \(78\)

D. \(81\)


c) Số giá trị nguyên m trong [-10; 10] để \(y = \frac{x - 3}{x^2 + (m - 2)x + 6}\) có 2 tiệm cận đứng:

A. 10

B. 9

*C. 8

D. 7


d) \(\frac{2x^2 + 4x - 3}{2x + 4}\) có phương trình tiệm cận xiên:

A. \(y = \frac{-3}{2}x - 5 + \frac{17}{2x + 4}\)

B. \(y = -x - 5 + \frac{17}{2x + 4}\)

*C. \(y = -x - 4 + \frac{13}{2x + 4}\)

D. \(y = -x - 4 + \frac{12}{2x + 4}\)


\textbf{Lời giải:}

\textbf{Lời giải cho mệnh đề a):}

Hàm số \(y = \frac{(3m - 5)x}{nx - 4}\) có Tiệm cận đứng và ngang giao nhau tại điểm \((1; 3)\)

\(\Leftrightarrow\) Hàm số \(y = \frac{(3m - 5)x}{nx - 4}\) có tiệm cận đứng: \(x = 1\); tiệm cận ngang: \(y = 3\)

\begin{itemize}
\item Tiệm cận đứng:
\(x = \frac{4}{n} = 1 \Rightarrow n = \frac{4}{1} = 4\)
\item Tiệm cận ngang:
\(y = \frac{3m - 5}{n} = 3\)
\end{itemize}

\(\Leftrightarrow 3m - 5 = 3 \cdot 4 = 12\)

\(\Rightarrow 3m = 12 + 5 = 17\)

\(\Rightarrow m = \frac{17}{3}\)

Vậy \(-m + n = (-1) \cdot \frac{17}{3} + 4 = \frac{-5}{3}\)

\textbf{Lời giải cho mệnh đề b):}

Hàm số \(y = \frac{(2m + 4)x + 1}{2x + 2}\) có:

\begin{itemize}
\item Tiệm cận ngang: \(y = \frac{2m + 4}{2}\)
\item Tiệm cận đứng: \(x = \frac{-2}{2} = -1\)
\end{itemize}

Diện tích hình chữ nhật tạo bởi hai đường tiệm cận và hai trục tọa độ:

\(S = \left| \frac{2m + 4}{2} \cdot (-1) \right| = \left| \frac{2m + 4}{2} \cdot (-1) \right| = \frac{|2m + 4| \cdot |2|}{4}\)

Theo đề bài: \(S = 3\)

\(\Rightarrow \frac{|2m + 4| \cdot 2}{4} = 3\)

\(\Leftrightarrow |2m + 4| = \frac{3 \cdot 4}{2} = 12\)

\(\Leftrightarrow 2m + 4 = \pm 12\)

Trường hợp 1: \(2m + 4 = 12 \Rightarrow m = 4\)

Trường hợp 2: \(2m + 4 = -12 \Rightarrow m = -8\)

Tổng bình phương các giá trị: \((4)^2 + (-8)^2 = 80\)

\textbf{Lời giải cho mệnh đề c):}

\textbf{Để có 2 tiệm cận đứng:}

Điều kiện: \(f(3) \neq 0\) và mẫu số có 2 nghiệm phân biệt

\(\Leftrightarrow \begin{cases}
f(3) \neq 0 \\
\Delta > 0
\end{cases}\)

\(\Leftrightarrow \begin{cases}
m \neq -3 \\
m^2 - 4m - 20 > 0
\end{cases}\)

\(\Leftrightarrow m \in (-\infty, 2 - 4\sqrt{6}) \cup (2 + 4\sqrt{6}, +\infty) \setminus \{-3\}\)

\textbf{Kết quả:} Có \(8\) giá trị nguyên thỏa mãn

\textbf{Lời giải cho mệnh đề d):}

\textbf{Giải:}

Ta có: \(y = \frac{2x^2 + 4x - 3}{2x + 4} = -x - 4 + \frac{13}{2x + 4}\)

\(\Rightarrow \displaystyle\lim_{x \to +\infty} \left(\left(\frac{2x^2 + 4x - 3}{2x + 4}\right) - \left(-x - 4 + \frac{13}{2x + 4}\right)\right) = \displaystyle\lim_{x \to +\infty} \frac{13}{2x + 4} = \displaystyle\lim_{x \to +\infty} \frac{\frac{13}{x}}{-2 + \frac{4}{x}} = 0\)

\(\Rightarrow\) Tiệm cận xiên: \(y = -x - 4 + \frac{13}{2x + 4}\)



\newpage

\section*{Câu 5:}

a) \(y = \frac{(m - 4)x + 1}{nx - 1}\) có Tiệm cận đứng đi qua điểm \((3; 0)\) và tiệm cận ngang đi qua điểm \((0; 2)\) thì giá trị biểu thức là:

*A. \(15\)

B. \(17\)

C. \(14\)

D. \(16\)


b) \(y = \frac{(m - 2)x}{x + 3}\) có đường tiệm cận tạo hình chữ nhật diện tích (8\. Tổng bình phương các giá trị m là:

A. \(20\)

B. \(10\)

C. \(15\)

*D. \(\frac{584}{9}\)


c) Số giá trị nguyên m trong [-10; 10] để \(y = \frac{x - 3}{x^2 + (-2m + 3)x + 12}\) có 3 tiệm cận:

A. 13

*B. 14

C. 15

D. 16


d) \(\frac{2x^2 - x}{x - 3}\) có phương trình tiệm cận xiên:

*A. \(y = -2x + 7 + \frac{21}{x - 3}\)

B. \(y = -2x + 7 + \frac{20}{x - 3}\)

C. \(y = -2x + 6 + \frac{18}{x - 3}\)

D. \(y = -x + 4 + \frac{12}{x - 3}\)


\textbf{Lời giải:}

\textbf{Lời giải cho mệnh đề a):}

Hàm số \(y = \frac{(m - 4)x + 1}{nx - 1}\) có Tiệm cận đứng đi qua điểm \((3; 0)\) và tiệm cận ngang đi qua điểm \((0; 2)\)

\(\Leftrightarrow\) Hàm số \(y = \frac{(m - 4)x + 1}{nx - 1}\) có tiệm cận đứng: \(x = 3\); tiệm cận ngang: \(y = 2\)

\begin{itemize}
\item Tiệm cận đứng:
\(x = \frac{1}{n} = 3 \Rightarrow n = \frac{1}{3}\)
\item Tiệm cận ngang:
\(y = \frac{m - 4}{n} = 2\)
\end{itemize}

\(\Leftrightarrow m - 4 = 2 \cdot \frac{1}{3}\)

\(\Rightarrow m = \frac{2}{3} + 4 = \frac{14}{3}\)

\(\Rightarrow m = \frac{14}{3}\)

Vậy \(3m + 3n = 3 \cdot \frac{14}{3} + 3 \cdot \frac{1}{3} = 15\)

\textbf{Lời giải cho mệnh đề b):}

Hàm số \(y = \frac{(m - 2)x}{x + 3}\) có:

\begin{itemize}
\item Tiệm cận ngang: \(y = \frac{m - 2}{1}\)
\item Tiệm cận đứng: \(x = \frac{-3}{1} = -3\)
\end{itemize}

Diện tích hình chữ nhật tạo bởi hai đường tiệm cận và hai trục tọa độ:

\(S = \left| \frac{m - 2}{1} \cdot (-3) \right| = \left| \frac{m - 2}{1} \cdot (-3) \right| = \frac{|m - 2| \cdot |3|}{1}\)

Theo đề bài: \(S = 8\)

\(\Rightarrow \frac{|m - 2| \cdot 3}{1} = 8\)

\(\Leftrightarrow |m - 2| = \frac{8}{3}\)

\(\Leftrightarrow m - 2 = \pm \frac{16}{3}\)

Trường hợp 1: \(m - 2 = \frac{16}{3} \Rightarrow m = \frac{22}{3}\)

Trường hợp 2: \(m - 2 = \frac{-16}{3} \Rightarrow m = \frac{-10}{3}\)

Tổng bình phương các giá trị: \((\frac{22}{3})^2 + (\frac{-10}{3})^2 = \frac{584}{9}\)

\textbf{Lời giải cho mệnh đề c):}

\textbf{Để có 3 tiệm cận:}

\textbf{- Dễ thấy hàm số chỉ có 1 tiệm cận ngang} \(y = 0\) (vì bậc tử < bậc mẫu)

\textbf{- Để hàm số có 3 tiệm cận} \(\Rightarrow\) hàm số có 2 tiệm cận đứng

\(\Leftrightarrow \begin{cases}
f(3) \neq 0 \\
\Delta > 0
\end{cases}\)

\(\Leftrightarrow \begin{cases}
m \neq 5 \\
4m^2 - 12m - 39 > 0
\end{cases}\)

\(\Leftrightarrow m \in (-\infty, \frac{12 - 8\sqrt{12}}{8}) \cup (\frac{12 + 8\sqrt{12}}{8}, +\infty) \setminus \{5\}\)

\textbf{Kết quả:} Có \(14\) giá trị nguyên thỏa mãn

\textbf{Lời giải cho mệnh đề d):}

\textbf{Giải:}

Ta có: \(y = \frac{2x^2 - x}{x - 3} = -2x + 7 + \frac{21}{x - 3}\)

\(\Rightarrow \displaystyle\lim_{x \to +\infty} \left(\left(\frac{2x^2 - x}{x - 3}\right) - \left(-2x + 7 + \frac{21}{x - 3}\right)\right) = \displaystyle\lim_{x \to +\infty} \frac{21}{x - 3} = \displaystyle\lim_{x \to +\infty} \frac{\frac{21}{x}}{-1 + \frac{-3}{x}} = 0\)

\(\Rightarrow\) Tiệm cận xiên: \(y = -2x + 7 + \frac{21}{x - 3}\)



\end{document}
\documentclass{article}
\usepackage{fontspec} % For XeLaTeX font support
\usepackage{amsmath} % For advanced math environments
\usepackage{amsfonts}
\usepackage{amssymb}
\usepackage{geometry} % For page layout
\geometry{a4paper, margin=1in} % Set page margins
\usepackage{polyglossia}
\setmainlanguage{vietnamese}
\setmainfont{Times New Roman}

\begin{document}

Câu 1: Cho hàm số \(y = \frac{(-m + 2)x}{-x + 1}\) có đường tiệm cận tạo hình chữ nhật diện tích 6. Tính tổng bình phương của các giá trị \(m\):

A. \(16\)

B. \(26\)

*C. \(80\)

D. \(16\)

Lời giải:

Hàm số \(y = \frac{(-m + 2)x}{-x + 1}\) có:

\begin{itemize}
\item Tiệm cận ngang: \(y = \frac{-m + 2}{-1}\)
\item Tiệm cận đứng: \(x = \frac{-1}{-1} = 1\)
\end{itemize}

Diện tích hình chữ nhật tạo bởi hai đường tiệm cận và hai trục tọa độ:

\(S = \left| \frac{-m + 2}{-1} \right| = \left| \frac{-m + 2}{-1} \right| = \frac{|-m + 2| \cdot |1|}{1}\)

Theo đề bài: \(S = 6\)

\(\Rightarrow \frac{|-m + 2|}{1} = 6\)

\(\Leftrightarrow |-m + 2| = \frac{6}{1} = 6\)

\(\Leftrightarrow -m + 2 = \pm 6\)

Trường hợp 1: \(-m + 2 = 6 \Rightarrow m = -4\)

Trường hợp 2: \(-m + 2 = -6 \Rightarrow m = 8\)

Tổng bình phương các giá trị: \((-4)^2 + (8)^2 = 80\)



Câu 2: Số giá trị nguyên \(m\) trong \([-10; 10]\) để hàm số \(y = \frac{x - 1}{x^2 + (m - 2)x + 1}\) có 2 tiệm cận đứng:

*A. 16

B. 15

C. 17

D. 16

Lời giải:

Để có 2 tiệm cận đứng:

Điều kiện: \(f(1) \neq 0\) và mẫu số có 2 nghiệm phân biệt

\(\Leftrightarrow \begin{cases}
f(1) \neq 0 \\
\Delta > 0
\end{cases}\)

\(\Leftrightarrow \begin{cases}
m \neq 0 \\
m^2 - 4m > 0
\end{cases}\)

\(\Leftrightarrow m \in (-\infty, 0) \cup (4, +\infty)\)

Kết quả: Có \(16\) giá trị nguyên thỏa mãn



Câu 3: Cho hàm số \(y = \frac{2x^2 + 3x + 2}{x + 1}\). Phương trình đường tiệm cận xiên của đồ thị hàm số này là:

A. \(\frac{3}{2}x + 1\)

*B. \(2x + 1\)

C. \(2x + 2\)

D. \(2x + 3\)

Lời giải:

Giải:

Ta có: \(y = \frac{2x{2} + 3x + 2}{x + 1} = 2x + 1 + \frac{1}{x + 1}\)

\(\Rightarrow \displaystyle\lim_{x \to +\infty} \left(\left(\frac{2x{2} + 3x + 2}{x + 1}\right) - \left(2x + 1 + \frac{1}{x + 1}\right)\right) = \displaystyle\lim_{x \to +\infty} \frac{1}{x + 1} = \displaystyle\lim_{x \to +\infty} \frac{1}{x \cdot 1 + \frac{1}{x}} = 0\)

\(\Rightarrow\) Tiệm cận xiên: \(y = 2x + 1\)



\end{document}